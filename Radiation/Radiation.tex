% !TeX program = lualatex
% !TeX encoding = utf8
% !TeX spellcheck = uk_UA
% !TeX root =../ClassicalEletrodynamics.tex

%=========================================================
\Opensolutionfile{answer}[\currfilebase/\currfilebase-Answers]
\chapter{Випромінювання}\label{\currfilebase}
%=========================================================


Перед тим, як перейти до конкретних задач, пов’язаних з вивченням випромінювання електромагнітних хвиль, зауважимо, що при цьому нас цікавить передусім
характер поля на великих відстанях: \( r \gg \lambda \), де \( \lambda = \frac{2\pi c}{\omega} \) --- довжина хвилі; також вважаємо, що \( r \) значно
більше за розміри системи, що випромінює. Цю область називають хвильовою зоною. Тут, далеко від джерел, поле можна подати як суперпозицію сферичних
хвиль \( \sim \frac{\exp\{i(kr - \omega t)\}}{r} \), де \( k = \frac{2\pi}{\lambda} = \frac{\omega}{c} \). Похідні від хвильового поля в основному
визначаються експоненціальним співмножинком. Зокрема:

\begin{equation}
	\left( \frac{e^{i(kr - \omega t)}}{r} \right) \approx \left( \frac{e^{i(kr - \omega t)}}{r} \right) \left( 1 + \frac{1}{r^2} \right),
	\label{eq:wave_field}
\end{equation}

де позначено:

\begin{equation}
	\vect{k} = nk = n\frac{\omega}{c}, \quad \vect{n} = \frac{\vect{r}}{r}.
	\label{eq:wave_vector}
\end{equation}

Відповідно до цього, на поверхні сфери радіуса \( r \gg \lambda \) на невеликих, порівняно із \( r \), ділянках можна наближено вважати хвилі плоскими.
Можна показати, що поправки до цього наближення дають малі внески в потік енергії, які прямують до нуля за \( r \to \infty \).

Враховуючи ці обставини, з рівнянь Максвелла легко отримати співвідношення, аналогічні формулам для плоских хвиль:

\begin{equation}
	\Bfield = [\vect{n} \times \Efield], \quad \Efield = -[\vect{n} \times \Bfield].
	\label{eq:maxwell_relations}
\end{equation}

Ці співвідношення справедливі в хвильовій зоні будь-якої обмеженої системи з точністю до членів \( \frac{1}{r} \) включно; лише ці члени дають
ненульовий внесок при обчисленні потоку енергії за \( r \to \infty \).

Як буде видно далі, складові напруженостей поля, що спадають як \( \frac{1}{r} \), з'являються лише тоді, коли в системі є прискорені заряди. Завдяки
цим складовим маємо ненульовий потік енергії випромінювання.

%% --------------------------------------------------------
\section{Поле заряду, що рухається з прискоренням}
%% --------------------------------------------------------

%% --------------------------------------------------------
\subsection*{Потенціали Ліенара-Віхерта}
%% --------------------------------------------------------

Для точкового заряду \( q \) з траєкторією \( \vect{r} = \vect{r}_q(t) \) густина заряду і густина струму мають вигляд:

\begin{equation}\label{eq:4.1.1}
	\rho(t, \vect{r}) = q \delta^3(\vect{r} - \vect{r}_q(t)),
\end{equation}

\begin{equation}
	\vect{j}(t, \vect{r}) = q \dot{\vect{r}}_q(t) \delta^3(\vect{r} - \vect{r}_q(t)),
	\label{eq:current_density}
\end{equation}

де \( \delta^3 \) --- тривимірна \( \delta \)-функція.


За загальними формулами \eqref{eq:D}, \eqref{eq:2.2.18}, \eqref{eq:2.2.19} для потенціалів:

\begin{equation}
	\varphi(t, \vect{r}) = \frac{2}{c} \int dt' \int d^3\vect{r}' \delta \left[ (t - t')^2 - \frac{1}{c^2} (\vect{r} - \vect{r}')^2 \right] \theta (t - t')
	\rho (t', \vect{r}'),
	\label{eq:general_potential}
\end{equation}

Враховуючи \eqref{eq:4.1.1}, дістанемо:

\begin{equation}
	\varphi(t, \vect{r}) = \frac{2q}{c} \int dt' \delta \left[ (t - t')^2 - \frac{1}{c^2} (\vect{r} - \vect{r}_q (t'))^2 \right] \theta (t - t').
	\label{eq:potential_with_delta}
\end{equation}

Підінтегральна функція відмінна від нуля при \( t = t_q (t, \vect{r}) \), де \( t_q \) є розв’язком рівняння:

\begin{equation}\label{eq:retarded_time}
	c(t - t_q) = R_q, \quad R_q = |\vect{R}_q|, \quad \vect{R}_q = \vect{r} - \vect{r}_q (t_q),
\end{equation}

яке пов'язує час випромінювання сигналу \( t_q \) (запізнюючий час) у точці \( \vect{r}_q (t_q) \) та час \( t \) прийому сигналу у точці \( \vect{r}
\). За властивістю \(\delta\)-функції від складного аргументу:

\begin{multline}\label{eq:4.1.4}
	\varphi(t, \vect{r}) = \frac{2q}{c} \left[ \frac{d}{dt'} \left( (t - t')^2 - \frac{1}{c^2} (\vect{r} - \vect{r}_q (t'))^2 \right) \right]_{t' =
	t_q}^{-1} = \\
    = \frac{q}{c} \left[ \left( t - t') - \frac{1}{c^2} \dot{\vect{r}}_q (t') \cdot (\vect{r} - \vect{r}_q (t')) \right) \right]_{t' = t_q}^{-1}.
\end{multline}

Перепишемо це у більш компактному виді, врахувавши зв'язок \eqref{eq:retarded_time}:

\begin{equation}
	\varphi(t, \vect{r}) = \frac{q}{R_q - (\vect{R}_q \cdot \vect{v}_q)/c},
	\label{eq:compact_potential}
\end{equation}

тут і далі позначено \( \vect{v}_q = \dot{\vect{r}}_q (t_q) \).

Аналогічно отримуємо вектор-потенціал:

\begin{equation}
	\vect{A}(t, \vect{r}) = \frac{q \vect{v}_q (t_q)}{c \left[ R_q - (\vect{R}_q \cdot \vect{v}_q)/c \right]}.
	\label{eq:vector_potential}
\end{equation}

Формули \eqref{eq:compact_potential}), \eqref{eq:vector_potential} подають \textit{потенціали Лієнара-Віхерта} точкового заряду. Звідси, за формулами
\eqref{eq:potB}, \eqref{eq:potE} отримуємо напруженості полів.

Якщо заряд рухається зі сталою швидкістю \( \vect{v}_q = \text{const} \), обчислення напруженостей полів виходячи з (\ref{eq:compact_potential}),
(\ref{eq:vector_potential}) дає нульовий потік на нескінченності, оскільки поля спадають, як \( 1/r^2 \). Це найбільш очевидно в системі спокою заряду.
Таким чином, випромінювання можливе лише для прискореного руху заряду.

	\begin{figure}[h!]
		\begin{tikzpicture}[scale=1]
			\coordinate (O) at (-4,-3.5);
			\draw[-latex] (O) -- ([yshift=3cm]O) node[left] {$z$};
			\draw[-latex] (O) -- ([xshift=3cm]O)  node[below] {$y$};
			\draw[-latex] (O) -- ([shift={(225:3)}]O)  node[left] {$x$};
			\draw[red, thick] (-2,2) .. controls (-1,0) and (1,0) .. (2,-2)  coordinate[pos=0.2] (a) coordinate[pos=0.8] (b);
			\node[circle, inner sep=0, ball color=red, text=white, font=\tiny\bfseries] (A) at (a) {$+$};
			\draw[-latex] (A) -- +(-45:1.5) node[below] {$\vect{v}_q$};
			\node[circle, inner sep=0, ball color=red, text=white, font=\tiny\bfseries] (B) at (b) {$+$};
			\draw[-latex] (A) -- node[above] {$\vect{R}_q(t_q)$} (5,1.5) coordinate[pos=0.1] (C1) coordinate (C);
			%	\draw[-latex] (A) -- (C1) ;
			\node[left, text width=10em] (1) at (-3, 1) {\itshape\small Положення частинки в момент $t_q = t - R_q/c$};
			\node[below, text width=10em] (2) at (3, -2) {\itshape\small Положення частинки в момент $t$};
            \draw[->] (1) to[bend left] (A);
            \draw[->] (2) to[bend left] (B.west);
			\fill[black] (C) circle (1pt);
			\draw[-latex] (O) -- node[above=5pt] {$\vect{r}_q$} (A);
			\draw[-latex] (O) -- node[pos=0.7, above] {$\vect{r}$} (C);
%			\node[below right, text width=5em] at (C) ;
%			\draw[-latex, dashed] (B) -- node[below=5pt] {$\vect{R}_q(t)$}  (C);
		\end{tikzpicture}
		\captionof{figure}{До розрахунку запізнюючих потенціалів}
		\label{pic:retardion}
	\end{figure}

%% --------------------------------------------------------
\subsection*{Випромінювання}
%% --------------------------------------------------------

Нас цікавить лише та частина напруженостей полів, що спадає як \( \frac{1}{r} \) на нескінченності. Це значно полегшує обчислення за формулами
\eqref{eq:potB}, \eqref{eq:potE}. Але обчислювати \( \Efield \) за формулою \eqref{eq:potE} немає необхідності, оскільки в хвильовій зоні:

\begin{equation*}
	\Efield = [\Bfield \times \vect{n}], \quad \vect{\Pi} = \frac{c}{4\pi} B^2 \vect{n},
\end{equation*}
де вектор \( \vect{n} = \frac{\vect{r}}{r} \) вказує напрямок випромінювання.

Обчислимо індукцію магнітного поля:

\begin{equation*}
	\Bfield = \Rot\vect{A}.
\end{equation*}
Ненульовий внесок в члени \( \sim \frac{1}{r} \) дає лише величина \( \vect{v}(t_q) \), яка залежить від \( \vect{r} \) через час запізнення \( t_q =
t_q(t, \vect{r}) \). Похідні від \( t_q \) визначаються з рівняння \eqref{eq:retarded_time}, що задає цю функцію неявно:

\begin{equation}
	\nabla t_q = \frac{\vect{n}_q}{c Z_q},
	\label{eq:nabla_tq}
\end{equation}

де позначено:

\[
	Z_q = 1 - (\vect{n}_q \cdot \vect{v}_q), \quad \vect{n}_q = \frac{\vect{R}_q}{R_q}.
\]

На великих відстанях, відкидаючи члени порядку \( \sim \frac{1}{r^2} \), дістанемо:

\begin{equation}
	\Bfield \approx -\frac{q}{c} \left[ \frac{\vect{n}_q \times (\vect{n}_q \times \dot{\vect{v}}_q)}{Z_q^3 R_q} + \frac{\vect{n}_q \times \vect{v}_q}{c
			Z_q^2 R_q} \right],
	\label{eq:magnetic_field}
\end{equation}
де \( \dot{\vect{v}}_q = \frac{d\vect{v}_q}{dt_q} \).

Інтенсивність випромінювання в тілесний кут \( d\Omega \) у фіксований запізнюючий момент часу \( t_q \):

\begin{equation*}
	dI = \vect{\Pi} \cdot \vect{n} R_q^2 d\Omega = \frac{c}{4\pi} |\Bfield|^2 R_q^2 d\omega, \quad R_q \to \infty.
\end{equation*}

Після дещо громіздких обчислень з використанням (\ref{eq:magnetic_field}) отримуємо потужність випромінювання в тілесний кут \( d\Omega \) в напрямку \(
\vect{n}_q \):

\begin{equation}
	dI = \frac{q^2}{4\pi c^3} \left[ \frac{|\vect{n}_q \times (\vect{n}_q \times \dot{\vect{v}}_q)|^2}{Z_q^5} + \frac{|\vect{n}_q \times \vect{v}_q|^2}{c^2
			Z_q^4} \right] do.
	\label{eq:radiation_power}
\end{equation}

Нагадаємо, що всі величини на траєкторії частинки (що входять з індексом \( q \)) обчислюються в момент часу запізнення \( t_q \).

Як було зазначено вище, випромінює лише прискорений заряд. Це також очевидно з формули (\ref{eq:radiation_power}).

У випадку нерелятивістських рухів \( |\vect{v}_q| \ll c \):

\begin{equation*}
	dI(\vect{n}) = \frac{q^2}{4\pi c^3} \left[ \dot{\vect{v}}_q^2 - (\vect{n} \cdot \dot{\vect{v}}_q)^2 \right] d\Omega = \frac{q^2}{4\pi c^3} \sin^2 \theta
	\, \dot{\vect{v}}_q^2 \, d\Omega.
\end{equation*}

Повна потужність випромінювання в усіх напрямках:

\begin{equation}
	N = \int_{4\pi} dI(\vect{n}) = \frac{2q^2 \dot{\vect{v}}_q^2}{3c^3}.
	\label{eq:total_power_nonrel}
\end{equation}

В ультрарелятивістському випадку, коли швидкість частинки, що випромінює, близька до швидкості світла, значну роль відіграє вираз \( 1 -
\frac{\vect{v}_q \cdot \vect{n}}{c} \) в знаменнику формули (\ref{eq:radiation_power}). Якщо \( 1 - \frac{v_q}{c} \ll 1 \), завдяки цьому виразу
інтенсивність зосереджена в області малих кутів \( \theta \) між напрямком швидкості \( \vect{v}_q(t_q) \) і напрямком випромінювання \( \vect{n}_q \).
Якщо записати:

\begin{equation*}
	1 - \frac{\vect{n}_q \cdot \vect{v}_q}{c} = 1 - \frac{v_q \cos \theta}{c} \approx 1 - \frac{v_q}{c} + \frac{\theta^2}{2}, \quad v_q = |\vect{v}_q|,
\end{equation*}

маємо оцінку кутів, для яких випромінювання найбільш ефективне:

\begin{equation}
	\theta \approx \left( 1 - \frac{v_q}{c} \right)^{1/2}.
	\label{eq:effective_angle}
\end{equation}

Для паралельних \( \vect{v}_q \) та \( \dot{\vect{v}}_q \) маємо:

\begin{equation*}
	\Bfield \approx -\frac{q}{c^2 R_q Z_q^3} \left[ \vect{n}_q \times \dot{\vect{v}}_q \right],
\end{equation*}

\begin{equation}
	dI = \frac{q^2}{4\pi c^3} \frac{\dot{\vect{v}}_q^2 \sin^2 \theta}{Z_q^6} do.
	\label{eq:radiation_power_parallel}
\end{equation}

В області малих кутів (\ref{eq:effective_angle}) маємо:

\begin{equation*}
	dI \approx \frac{q^2}{4\pi c^3} \dot{\vect{v}}_q^2 \left( 1 - \frac{v_q}{c} + \frac{\theta^2}{2} \right)^{-6} \theta^2 do.
\end{equation*}

Якщо швидкість та прискорення взаємно перпендикулярні \( \vect{v}_q \perp \dot{\vect{v}}_q \) (наприклад, рух по колу), тоді з
(\ref{eq:radiation_power}):


\begin{equation}\label{eq:4.1.14}
	dI = \frac{q^2}{4\pi c^3} \left[ \frac{\dot{\vect{v}}_q^2}{Z^4} - \frac{\left(1 - \frac{v_q^2}{c^2}\right) (\vect{n} \cdot \dot{\vect{v}}_q)^2}{Z^6}
		\right] do,
\end{equation}
де

\begin{equation*}
	Z = 1 - \frac{\vect{v}_q \cdot \vect{n}}{c}.
\end{equation*}

%% --------------------------------------------------------
\subsection*{Коловий рух}
%% --------------------------------------------------------

Ця формула описує середню інтенсивність випромінювання для релятивістської частинки, що рухається по колу в магнітному полі.

Формулу \eqref{eq:radiation_power} можна застосувати у разі руху частинки в одній площині по колу під дією однорідного магнітного поля. В цьому разі
випромінювання
концентрується у площині руху в межах кутів, що визначаються формулою \eqref{eq:effective_angle}. Внаслідок того, що релятивістська частинка випромінює
переважно в
напрямку руху, випромінювання складатиметься з дуже коротких імпульсів, період повторення яких дорівнює періоду обертання.

Для колового руху в магнітному полі \( \Bfield \) частота обертання:

\begin{equation*}
	\omega = \frac{eB}{\gamma m c},
\end{equation*}
де \( \gamma \) --- релятивістський фактор, \( m \) --- маса частинки, \( c \) --- швидкість світла.

Прискорення:

\begin{equation*}
	\dot{\vect{v}}_q = \omega \vect{v}_q \times \vect{n}_B,
\end{equation*}
де \( \vect{n}_B \) --- одиничний вектор у напрямку магнітного поля.

Систему координат виберемо таким чином, що:

\begin{equation*}
	\vect{n}_B = (0, 0, 1), \quad \vect{n}_q = (\sin \theta, 0, \cos \theta),
\end{equation*}
і зафіксуємо напрямок \( \vect{n}_q = \{\sin \theta, 0, \cos \theta\} \).

Середнє значення інтенсивності випромінювання можна отримати за допомогою інтегрування формули \eqref{eq:4.1.14} по \( dt \), причому слід перейти до
інтегрування по часу запізнення:

\begin{equation*}
	t_q = t_q(t),
\end{equation*}
де \( t_q = t_q(t) \) неявно задано формулою \eqref{eq:4.1.4}.

Після тривалих обчислень дістанемо середнє (за часом) значення інтенсивності в тілесний кут \( d\Omega \) в напрямку під кутом \( \theta \) до площини
орбіти:


\begin{equation*}
	\left\langle dI \right\rangle  = \frac{q^2 B^2 v_q}{8\pi m^2c^5}\left( 1 - \frac{v_q^2}{c^2}\right)  \left[
		\frac{2-\cos^2\theta - \frac{v_q^2}{4c^2}\left( 1 + \frac{3v_q^2}{c^2}\right)\cos^4\theta  }{ \left( 1- \frac{v_q^2}{c^2}\cos^2\theta\right)^{7/2} }
		\right] d\omega.
\end{equation*}


%% --------------------------------------------------------
\section{Довгохвильове випромінювання}
%% --------------------------------------------------------

\subsection*{Дипольне випромінювання}

Розглянемо систему зарядів, розмір якої \( L \) значно менший за довжину хвилі:

\begin{equation}
	\lambda \gg L.
	\label{eq:long_wavelength_condition}
\end{equation}

За цією умовою сигнал встигає поширитися всередині системи значно раніше, ніж відбудуться суттєві зміни у розподілі зарядів. Обчислимо потік енергії,
виходячи з формули для вектор-потенціалу:

\begin{equation}\label{eq:4.2.2}
	\vect{A}(t, \vect{r}) = \frac{1}{c} \int \frac{\vect{j}\left(t - \frac{|\vect{r} - \vect{r}'|}{c}, \vect{r}'\right)}{|\vect{r} - \vect{r}'|} \, dV',
\end{equation}
де вважаємо, що початок координат знаходиться в області, де розташована система струмів.

Густина потоку енергії подається вектором Пойнтінга:

\begin{equation*}
	\vect{\Pi} = \frac{c}{4\pi} [\Efield \times \Bfield].
\end{equation*}

На великій відстані від системи зарядів, що випромінює, можна користуватися співвідношеннями для плоских хвиль, тому:

\begin{equation}\label{eq:poynting_vector_far_field}
	\vect{\Pi} = \frac{c}{4\pi} B^2 \vect{n},
\end{equation}
де \( \vect{n} = \frac{\vect{r}}{r} \) --- напрямок поширення хвилі.

Наша мета --- оцінити головні внески в напруженості поля, що спадають як \( \frac{1}{r} \), завдяки чому маємо ненульовий потік енергії через сферу
великого радіусу \( r \to \infty \):

\begin{equation*}
	|\vect{\Pi}| = \frac{c}{4\pi} |\Bfield|^2 \sim \frac{1}{r^2}, \quad \int \vect{\Pi} \cdot d\vect{S} = \frac{c}{4\pi} \int d\Omega \, r^2 |\Bfield|^2
	\neq
	0 \quad \text{при} \quad r \to \infty.
\end{equation*}

Практично умова \( r \to \infty \) означає, що обчислення виконують в зоні випромінювання, яка визначена умовою \( r \gg \lambda \). За умови \( r \gg
\lambda \gg L \) маємо:

\begin{equation*}
	|\vect{r} - \vect{r}'| = r - \vect{n} \cdot \vect{r}' + O\left(\frac{1}{r}\right),
\end{equation*}

звідки

\begin{equation*}
	\vect{A}(t, \vect{r}) = \frac1{cr} \int  \vect{j}\left( t - \frac{r}c + \frac{\vect{n} \cdot \vect{r}'}c, \vect{r}'\right)  dV'.
\end{equation*}

Розкладемо запізнюючий аргумент по малій величині \( \frac{|\vect{r}'|}{r} \):

\begin{equation}
	\vect{A} = \vect{A}_0 + \vect{A}_1,
	\label{eq:vector_potential_expansion}
\end{equation}
де утримано два перших члени розкладу:

\begin{equation}
	\vect{A}_0 = \frac{1}{c r} \int \vect{j}(t_r, \vect{r}') \, dV',
	\label{eq:vector_potential_zero_order}
\end{equation}
де \( t_r = t - \frac{r}{c} \) не залежить від змінної інтегрування \( \vect{r}' \),

\begin{equation}\label{eq:A1}
	\vect{A}_1 = \frac{1}{c r} \int \vect{j}(t_r, \vect{r}')(\vect{n} \cdot \vect{r}') \, dV'.
\end{equation}

Перетворимо (\ref{eq:vector_potential_zero_order}), користуючись тотожністю \( \nabla \cdot (x_j \vect{j}) = J_j + x_j \nabla \cdot \vect{j} \) і
законом збереження заряду \( \nabla \cdot \vect{j} = -\frac{\partial \rho}{\partial t} \). Звідси:

\begin{equation*}
	j_j = \frac{\partial}{\partial t}(x_j \rho) - \nabla \cdot (x_j \vect{j}).
\end{equation*}

Проінтегруємо це співвідношення:

\begin{equation*}
	\int \vect{j} \, dV = \frac{d}{dt} \int \vect{r} \rho \, dV,
\end{equation*}

де враховано, що поза межами системи, що розглядається, \( \vect{j} = 0 \). Підстановка цього співвідношення в (\ref{eq:vector_potential_zero_order})
дає:

\begin{equation}
	\vect{A}_0 = \frac{\dot{\vect{d}}(t_r)}{c r},
	\label{eq:dipole_potential}
\end{equation}

де \( \vect{d}(t) = \int \vect{r} \rho(t, \vect{r}) \, dV \) – дипольний момент системи.

Оскільки \( \nabla t_r = \nabla \left(t - \frac{r}{c}\right) = -\frac{\vect{n}}{c} \), звідси:

\begin{equation*}
	\Bfield = \nabla \times \vect{A}_0 = -\frac{[\vect{n} \times \dot{\vect{d}}(t_r)]}{c r}.
\end{equation*}

З урахуванням \eqref{eq:poynting_vector_far_field} дістанемо потік енергії системи через сферу радіусу \( r \gg \lambda \):

\begin{equation*}
	\frac{d\epsilon}{dt} = \int \vect{\Pi} \cdot d\vect{S} = \frac{1}{4\pi c^3} \int\limits_0^{2\pi} d\varphi \int\limits_0^\pi d\theta \, \ddot{\vect{d}}^2
	\sin^3
	\theta
	=
	\frac{2}{3c} \ddot{\vect{d}}^2.
\end{equation*}

Потік енергії системи через сферу радіусу \( r \gg \lambda \) задається формулою:

\begin{equation}
	\frac{d\epsilon}{dt} = \frac{2}{3c^3} \dot{\vect{d}}^2(t_r).
	\label{eq:energy_flux}
\end{equation}

Якщо випромінює система з \( \vect{d} = \vect{d}_0 \cos(\omega t) \), тоді усереднена за часом потужність випромінювання пропорційна четвертій степені
частоти:

\begin{equation*}
	\left\langle \frac{d\epsilon}{dt} \right\rangle = \frac{\omega^4 \vect{d}_0^2}{3c^3}.
\end{equation*}

\subsection*{Квадрупольне і магнітодипольне випромінювання}

Якщо дипольний момент системи відсутній, основний внесок у вектор-потенціал \eqref{eq:vector_potential_expansion} дає наступний після дипольного член
розкладу \eqref{eq:A1}.

Скористаємося співвідношенням:

\begin{equation*}
	\Div\{x_i x_j \vect{j}\} = x_i J_j + x_j J_i + x_i x_j \Div \vect{j} = x_i J_j + x_j J_i - x_i x_j \frac{\partial \rho}{\partial t}.
\end{equation*}

Для обмеженої системи струмів інтеграл від лівої частини дорівнює нулю. Звідси:

\begin{equation*}
	\int dV (x_i J_j + x_j J_i) = \int dV x_i x_j \frac{\partial \rho}{\partial t} \Rightarrow \int dV (x_i J_j + x_j J_i) = \int dV x_i x_j
	\frac{\partial^2 \rho}{\partial t^2}.
\end{equation*}

Тоді:

\begin{equation*}
	\int dV \cdot j_i x_j = \frac{1}{2} \int dV (j_i x_j - j_j x_i) + \frac{1}{2} \int dV (j_i x_j + j_j x_i).
\end{equation*}

Зауважимо, що перший доданок містить компоненти магнітного моменту:

\begin{equation*}
	\vect{m} = \frac{1}{2c} \int [\vect{r} \times \vect{j}] \, dV.
\end{equation*}

З отриманих співвідношень та формули \eqref{eq:A1} випливає:

\begin{equation*}
	\vect{A} = \vect{A}_\text{md} + \vect{A}_\text{Q},
\end{equation*}
де:

\begin{equation}
	\vect{A}_\text{md} = \frac{[\vect{m} \times \vect{n}]}{c r},
	\label{eq:magnetic_dipole_potential}
\end{equation}

\begin{equation}
	\vect{A}_\text{Q} = \{A_{\text{Q},i}\}, \quad A_{\text{Q},i} = \frac{1}{2c^2} \frac{\partial^2}{\partial t^2} \int x_i (\vect{x} \cdot \vect{n})
	\rho \, dV',
	\label{eq:quadrupole_potential}
\end{equation}

де штрихом позначено величини, що залежать від змінної інтегрування \( \vect{x}' \). Останній доданок можна виразити через тензор квадрупольного моменту:

\begin{equation}
	Q_{ij} = \int \{3x_i x_j - r^2 \delta_{ij}\} \rho \, dV, \quad \sum_{i=1}^3 Q_{ii} = 0.
	\label{eq:quadrupole_tensor}
\end{equation}

Тоді можна записати:

\begin{equation}\label{eq:Quadrupole_potential}
\vect{A}_\text{Q} = \frac{1}{6c^2 r} \tilde{\vect{D}}(t_r) + \frac{\vect{n}}{6c^2 r^2} \frac{\partial^2}{\partial t^2} \int r^2 \rho \, dV,
\end{equation}

де введено позначення:

\begin{equation}
D_i = Q_{ij} n_j, \quad \vect{D} = \{D_i\}.
\label{eq:quadrupole_vector}
\end{equation}

Другий доданок в \eqref{eq:Quadrupole_potential} має вигляд \( \vect{n} f(r) \), тобто це градієнт від деякої функції; він не впливає на обчислення
напруженості \( \Bfield = \nabla \times \vect{A} \). Цей доданок можна знищити за допомогою калібрувального перетворення, тоді:

\begin{equation}
\vect{A}_{Q} = \frac{\vect{D}(t_r)}{6c^2 r}.
\label{eq:quadrupole_potential_simplified}
\end{equation}

Повний вектор-потенціал \eqref{eq:vector_potential_zero_order} є сумою дипольної, магніто-дипольної і квадрупольної складової:

\begin{equation*}
\vect{A} = \vect{A}_\text{дип} + \vect{A}_\text{мд} + \vect{A}_Q,
\end{equation*}

де відповідні співвідношення подано формулами \eqref{eq:dipole_potential}, \eqref{eq:magnetic_dipole_potential}, \eqref{eq:Quadrupole_potential}.

Нехтуючи членами більш високого порядку при диференціюванні \eqref{eq:dipole_potential}, \eqref{eq:magnetic_dipole_potential},
\eqref{eq:Quadrupole_potential}, маємо:

\begin{equation}\label{eq:4.2.12}
\Bfield = -\frac{1}{c^2 r} \left[ \vect{n} \times \ddot{\vect{d}} \right] + \left[ \vect{n} \times \left[ \vect{n} \times \ddot{\vect{m}} \right]
\right] + \frac{1}{6c} \left[ \vect{n} \times \dddot{\vect{D}} \right].
\end{equation}

В цій формулі залежність від кутів входить лише через \( n_i \), в тому числі в \( D_i = Q_{ij} n_j \), де \( \{n_i\} = \{\sin \theta \cos \varphi, \sin
\theta \sin \varphi, \cos \theta\} \) --- одиничний вектор. Інтегруючи по кутових змінних (\( d\Omega = \sin \theta \, d\theta \, d\varphi \)) дістанемо,
в силу симетрії:

\begin{equation*}
\int d\Omega \cdot n_i n_j = 0, \quad i \neq j; \quad \int d\Omega \cdot n_1^2 = \int d\Omega \cdot n_2^2 = \int d\Omega \cdot n_3^2.
\end{equation*}

Оскільки \( n_1^2 + n_2^2 + n_3^2 = 1 \), інтегруючи по \( d\Omega \) отримаємо суму трьох однакових доданків \( = 4\pi \); звідси:

\begin{equation}\label{eq:4.2.13}
\int d\Omega \cdot n_i n_j = \frac{4\pi}{3} \delta_{ij}.
\end{equation}

Це легко перевірити і прямим обчисленням.

Очевидно,

\begin{equation}\label{eq:4.2.14}
\int d\Omega \cdot n_i = 0, \quad \int d\Omega \cdot n_i n_j n_k = 0
\end{equation}

(підінтегральний вираз міняє знак при зміні напрямку координатних осей, а результат інтегрування по кутах має бути незалежним від такої зміни).

\subsection*{Квадрупольне і магнітодипольне випромінювання}

Обчислимо тензор:

\begin{equation}
X_{ijkl} = \int d\Omega \cdot n_i n_j n_k n_l,
\label{eq:tensor_X}
\end{equation}

що є симетричний за усіма індексами.

Очевидно, тут можливі лише два ненульових значення компонент: коли усі індекси співпадають, наприклад \( X_{3333} = \frac{4\pi}{5} \), та коли серед
індексів є дві різні пари, наприклад \( X_{1133} = \frac{4\pi}{15} \).

Тензор, що задовольняє цим умовам, легко побудувати шляхом симетризації квадратичних комбінацій символів Кронекера:

\begin{equation}\label{eq:4.2.16}
X_{ijkl} = \frac{4\pi}{15} (\delta_{ij}\delta_{kl} + \delta_{ik}\delta_{jl} + \delta_{il}\delta_{jk}).
\end{equation}

Для перевірки відзначимо, що подвійна згортка дає:

\begin{equation*}
X_{ijlj} = 4\pi = \int d\Omega \cdot 1.
\end{equation*}

Обчислимо потік через сферу великого радіусу \( r \), виходячи з \eqref{eq:4.2.12}. Квадрат першого (дипольного) доданку в \eqref{eq:4.2.12} дає внесок,
що вже обчислено
і подано формулою \eqref{eq:energy_flux}. При обчисленні \( \vect{B}^2 \) добуток першого (дипольного) доданку в \eqref{eq:4.2.12} на інші в кінцевому
результаті
дасть нуль в
силу \eqref{eq:4.2.14}, оскільки тут під інтегралом з’являються лише непарні комбінації компонент \( n_i \) (нагадаємо, що квадрупольний доданок в
\eqref{eq:4.2.12}
містить квадратичні комбінації цих компонент).

Також:

\begin{multline*}
\int [\vect{n} \times [\ddot{\vect{m}}\times\vect{n}] \cdot [\vect{n} \times \dddot{\vect{D}}] \, d\Omega = \int (m_i - n_i n_j m_i)
(\epsilon_{ipq} n_p \tilde{\dddot{Q}}_{qr} n_r) \, d\Omega = \\ = \int \dddot{m}_i\epsilon_{ipq} n_p \dddot{Q}_{qr} n_r =
\frac{4\pi}c \epsilon_{ipq} n_p \dddot{Q}_{qp}\dddot{m}_i = 0
\end{multline*}

де враховано симетрію \( Q_{qp} \) та формули \eqref{eq:4.2.13} та \eqref{eq:4.2.16}. Таким чином, потужність є сумою окремих потужностей дипольного,
магнітодипольного та
квадрупольного внесків.

Далі маємо:

\begin{equation*}
\int [\vect{n} \times [\ddot{\vect{m}}\times\vect{n}]]^2 \, d\Omega = \int (\ddot{\vect{m}}^2 - (\vect{n} \cdot \ddot{\vect{m}})^2) \, d\Omega =
\frac{8\pi}{3} \ddot{\vect{m}}^2.
\end{equation*}

За допомогою \eqref{eq:4.2.13} та \eqref{eq:4.2.16}  дістанемо:

\begin{equation*}
\int [\vect{n} \times \dddot{\vect{D}}]^2 \, d\Omega = \int [\dddot{\vect{D}}^2 - (\vect{n} \cdot \dddot{\vect{D}})^2] \, d\Omega = \frac{4\pi}{5}
\dddot{Q}_{kl} \dddot{Q}_{kl}.
\end{equation*}

Звідси потужність випромінювання:
\begin{equation*}
    \frac{d\epsilon}{dt} = \frac2{3c^3}\ddot{\vect{d}}^2+
    \frac2{3c^3}\ddot{\vect{m}}^2+
    \frac1{180c^5}\dddot{Q}_{kl} \dddot{Q}_{kl}.
\end{equation*}

\section{Розсіювання електромагнітних хвиль}

\subsection*{Загальні поняття}

Коли електромагнітна хвиля падає на систему зарядів, вони починають рухатися і стають джерелами вторинного випромінювання. Оскільки загальна енергія
зберігається, це означає, що частина енергії хвилі, що падає, переходить в енергію вторинних хвиль. Цей процес називають \textit{розсіюванням}. Він може
також супроводжуватися \textit{поглинанням} енергії, коли частина енергії хвилі, що падає, втрачається в розсіювачі, трансформуючись в інші види енергії.

Для опису цих процесів вводять \textit{диференціальний переріз розсіювання в тілесний кут} \( d\Omega \):

\begin{equation}
d\sigma = \frac{dI}{\Pi},
\label{eq:differential_cross_section}
\end{equation}
де \( dI \) --- середня (за часом) енергія вторинного випромінювання за одиницю часу в тілесний кут \( d\Omega \), \( \Pi \) — середня густина потоку
енергії хвилі, що падає, яка визначається вектором Пойнтінга.

\textit{Повний переріз розсіювання} є:

\begin{equation}
\sigma = \int d\sigma,
\label{eq:total_cross_section}
\end{equation}
де інтеграл береться по усіх можливих напрямках розсіяного випромінювання.

\textit{Перерізом поглинання} називають величину:

\begin{equation}
\sigma_a = \frac{Q}{\Pi},
\label{eq:absorption_cross_section}
\end{equation}
де \( Q \) --- середня енергія, що поглинається системою за одиницю часу.

\subsection*{Розсіювання електромагнітних хвиль точковим вільним зарядом}

Задача опису процесу розсіювання розбивається на дві: визначити рух заряду під дією електромагнітної хвилі, а потім визначити вторинне, тобто розсіяне,
випромінювання.

Нехай електричне поле хвилі, що падає, є:

\begin{equation}
\vect{E} = \vect{E}_0 \cos(\omega t - \vect{k} \cdot \vect{r}),
\label{eq:incident_wave}
\end{equation}

в точці, де розташований заряд.

\textit{Будемо вважати рух заряду нерелятивістським} (\( V \ll c \)), що дає змогу знехтувати магнітним полем в формулі для сили Лоренца, що діє на
заряд з боку електромагнітної хвилі. Силою реакції розсіяного випромінювання також знехтуємо.

Якщо інших сил тут немає, основний рух заряду буде періодичним з амплітудою:

\begin{equation*}
r \ll \frac{c}{\omega}.
\end{equation*}

Ця умова дозволяє відкинути доданок \( \vect{k} \cdot \vect{r} \ll 1 \) під аргументом косінуса в (\ref{eq:incident_wave}). Звідси з рівнянь руху:

\begin{equation*}
m \ddot{\vect{r}} = q \vect{E}_0 \cos(\omega t),
\end{equation*}

отримуємо:

\begin{equation*}
\ddot{\vect{r}} = -\frac{q \vect{E}_0}{m \omega^2} \cos(\omega t).
\end{equation*}

Дипольний момент системи:

\begin{equation*}
\vect{d} = q \vect{r} = -\frac{q^2 \vect{E}_0}{m \omega^2} \cos(\omega t),
\end{equation*}

причому:

\begin{equation}
\ddot{\vect{d}} = \frac{q^2 \vect{E}}{m}.
\label{eq:dipole_moment}
\end{equation}

Для нерелятивістського руху можна скористатися співвідношеннями для дипольного випромінювання. Це дає інтенсивність, випромінену в напрямку \( \vect{n}
\) в тілесний кут \( d\Omega \):

\begin{equation}
dI = \frac{1}{4\pi c^3} |\ddot{\vect{d}}|^2 \sin^2 \theta \, d\Omega,
\label{eq:intensity_dipole}
\end{equation}
де \( \theta \) --- кут між \( \ddot{\vect{d}} \) та напрямком спостереження \( \vect{n} \). Враховуючи (\ref{eq:dipole_moment}):

\begin{equation*}
dI = \frac{q^4 |\vect{E}_0|^2}{4\pi m^2 c^3} \sin^2 \theta \, d\Omega.
\end{equation*}

Оскільки вектор Пойнтінга хвилі, що падає, є:

\begin{equation*}
\vect{\Pi} = \frac{c}{4\pi} [\vect{E} \times \vect{B}],
\end{equation*}
де \( \Pi_0 \) --- напрямок хвилі, що падає, за визначенням перерізу \eqref{eq:differential_cross_section} маємо:

\begin{equation}
d\sigma = \left( \frac{q^2}{mc^2} \right)^2 \sin^2 \theta \, d\Omega.
\label{eq:differential_cross_section_thomson}
\end{equation}

Цікаво, що цей вираз не залежить від частоти. Повний переріз отримаємо, інтегруючи (\ref{eq:differential_cross_section_thomson}) по усіх напрямках:

\begin{equation}
\sigma_T = \frac{8\pi}{3} \left( \frac{q^2}{mc^2} \right)^2.
\label{eq:thomson_cross_section}
\end{equation}
(формула Томсона). Ця формула втрачає силу, коли \( \hbar \omega > mc^2 \), в цьому разі потрібно враховувати квантово-електродинамічні ефекти.


%% --------------------------------------------------------
\subsection*{Неполяризована хвиля}
%% --------------------------------------------------------

Формула (\ref{eq:intensity}) дає диференціальний переріз розсіювання для фіксованого напрямку поляризації хвилі, що падає. У природному світлі присутні
усі напрямки \( \vect{E} \). Будемо припускати, що усі ці напрямки еквівалентні, а значить, згідно з (\ref{eq:dipole_moment}), рівноймовірними є усі
напрямки дипольного моменту з (\ref{eq:intensity}).

Зафіксуємо напрямки спостереження:

\begin{equation*}
\vect{n} = \{\sin \theta_0 \cos \phi, \sin \theta_0 \sin \phi, \cos \theta_0\}
\end{equation*}

та напрямок падіння хвилі:

\begin{equation*}
\vect{n}_0 = \{0, 0, 1\}.
\end{equation*}

Відповідно, напрямок \( \vect{d} = \vect{n}_d |\vect{d}| \) визначимо формулою:

\begin{equation*}
\vect{n}_d = \{\cos \phi, \sin \phi, 0\},
\end{equation*}

де усі значення \( \phi \in [0, 2\pi] \) рівноймовірні.

Для кута \( \theta \) між \( \vect{n}_d \) та \( \vect{n} \) маємо:

\begin{equation*}
\sin^2 \theta = 1 - \cos^2 \theta = 1 - \sin^2 \theta_0 \cos^2 (\phi - \phi_0) = \frac{1 + \cos^2 \theta_0}{2}.
\end{equation*}

Звідси з \eqref{eq:intensity} та за визначенням \eqref{eq:differential_cross_section}:

\begin{equation*}
\left\langle d\sigma \right\rangle = \frac{1}{2} \left( \frac{q^2}{mc^2} \right)^2 (1 + \cos^2 \theta_0) \, d\Omega,
\end{equation*}

де \( d\Omega = \sin \theta_0 \, d\theta_0 \, d\phi \).


%% --------------------------------------------------------
\subsection*{Розсіювання на малих макроскопічних частинках}
%% --------------------------------------------------------

Якщо довжина плоскої хвилі \( \lambda \), що падає, значно більша за розміри макроскопічних частинок, можна вважати поле в околі частинок однорідним.
Припустимо також, що частота хвилі \( \omega \) значно менша за резонансні частоти усіх складових частинки. Це дає змогу обчислювати поляризованість \(
\alpha \) частинки, яка дає зв’язок між дипольним моментом:

\begin{equation}
\vect{d} = \alpha \vect{E},
\label{eq:polarizability}
\end{equation}

та зовнішнім однорідним полем \( \vect{E} \) в наближенні електростатики, тобто \( \alpha \) не залежить від \( \omega \). Тоді за формулами дипольного
випромінювання:

\begin{equation}
d\sigma = \frac{\omega^4 \alpha^2}{c^4} \sin^2 \theta \, d\Omega,
\label{eq:scattering_cross_section}
\end{equation}

\begin{equation}
\sigma = \frac{8\pi}{3} \frac{\omega^4 \alpha^2}{c^4}.
\label{eq:total_scattering_cross_section}
\end{equation}

Якщо поле хвилі можна вважати однорідним, але умова квазістатичності не виконується, поляризовність \( \alpha \) залежатиме від частоти і може бути
комплексною. В цьому разі:

\begin{equation*}
\sigma = \frac{8\pi}{3} \frac{|\alpha|^2 \omega^4}{c^4}.
\end{equation*}