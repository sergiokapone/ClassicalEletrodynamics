% !TeX program = lualatex
% !TeX encoding = utf8
% !TeX spellcheck = uk_UA
% !TeX root =../ClassicalEletrodynamics.tex

%% --------------------------------------------------------
\section{Основні формули векторного аналізу}\label{Vanaliz}
%% --------------------------------------------------------

%% --------------------------------------------------------
\subsection{Диференціальні операції в різних системах координат}
%% --------------------------------------------------------

%% --------------------------------------------------------
\subsubsection{Декартова система координат}
%% --------------------------------------------------------

\begin{align}\label{cartesian}
	\mathrm{grad}\,\psi \equiv \vect{\nabla}\psi & = \frac{\partial \psi}{\partial x} \vect{e}_x + \frac{\partial \psi}{\partial y} \vect{e}_y +
	\frac{\partial \psi}{\partial z} \vect{e}_z \\
	\mathrm{div}\,\left(\mathrm{grad}\,\psi\right) \equiv \Laplasian\psi    & = \frac{\partial^2 \psi}{\partial x^2} + \frac{\partial^2 \psi}{\partial y^2} + \frac{\partial^2 \psi }{\partial z^2}               \\
	\mathrm{div}\,\vect{A} \equiv \divg\vect{A}     & = \frac{\partial A_x}{\partial x}  + \frac{\partial A_y}{\partial y} + \frac{\partial A_z}{\partial z}                              \\
	\mathrm{rot}\,\vect{A} \equiv  \rot\vect{A}      & = \left( \frac{\partial A_z}{\partial y}  - \frac{\partial A_y}{\partial z}\right)  \vect{e}_x +
	\left( \frac{\partial A_x}{\partial z}  - \frac{\partial A_z}{\partial x}\right)  \vect{e}_y +
	\left( \frac{\partial A_y}{\partial x}  - \frac{\partial A_x}{\partial y}\right)  \vect{e}_z
\end{align}

%% --------------------------------------------------------
\subsubsection{Циліндрична система координат}
%% --------------------------------------------------------

\begin{align}\label{cylindric}
		\vect{\nabla}\psi & = \frac{\partial \psi}{\partial \rho} \vect{e}_{\rho} + \frac{1}{\rho}\frac{\partial \psi}{\partial \phi} \vect{e}_{\phi} + \frac{\partial \psi}{\partial z} \vect{e}_{z}                                                                                       \\
	\Laplasian\psi    & = \frac{1}{r} \frac{\partial }{\partial r} \left( r \frac{\partial \psi}{\partial r} \right) + \frac{1}{r^2} \frac{\partial^2 \psi}{\partial \phi^2} + \frac{\partial^2 \psi}{\partial z^2}                                                                   \\
	\divg\vect{A}     & = \frac{1}{\rho}\frac{\partial \left(\rho A_{\rho }\right)}{\partial \rho }+\frac{1}{\rho }\frac{\partial A_{\phi } }{\partial \phi }+\frac{\partial A_{z}}{\partial z}                                                                                    \\
	\rot\vect{A}      & =\left({\frac {1}{\rho }}{\frac {\partial A_{z}}{\partial \phi }}-{\frac {\partial A_{\phi }}{\partial z}}\right)\vect{e}_{\rho}+\left({\frac {\partial A_{\rho }}{\partial z}}-{\frac {\partial A_{z}}{\partial \rho }}\right)\vect{e}_{\phi} + \nonumber \\
	                  & +
	{\frac {1}{\rho }}\left({\frac {\partial (\rho A_{\phi })}{\partial \rho }}-{\frac {\partial A_{\rho }}{\partial \phi }}\right)\vect{e}_{z}
\end{align}
Орти циліндричної системи координат зв'язані з декартовими ортами як:
\begin{align}\label{}
    \vect{e}_r &= \cos\phi\, \vect{e}_x + \sin\phi\, \vect{e}_y, \\
    \vect{e}_{\phi} &= -\sin\phi\, \vect{e}_x + \cos\phi\, \vect{e}_y, \\
    \vect{e}_{z} &= \vect{e}_z.
\end{align}

%% --------------------------------------------------------
\subsubsection{Сферична система координат}
%% --------------------------------------------------------

\begin{align}\label{spheric}
    \vect{\nabla}\psi & = \frac{\partial \psi}{\partial r} \vect{e}_{r} + \frac{1}{r}\frac{\partial \psi}{\partial \theta} \vect{e}_{\theta} + \frac{1}{r\sin\theta}\frac{\partial \psi}{\partial \phi} \vect{e}_{\phi}                                                                                                            \\
	\Laplasian\psi    & = \frac{1}{r^2} \frac{\partial}{\partial r} \left( r^2 \frac{\partial \psi}{\partial r} \right) + \frac{1}{r^2 \sin \theta} \frac{\partial \psi}{\partial \theta} \left( \sin \theta \frac{\partial \psi}{\partial \theta} \right) + \frac{1}{r^2\sin^2 \theta} \frac{\partial^2 \psi}{\partial \phi^2} \\
	\divg\vect{A}     & =\frac{1}{r^2}\frac{\partial \left(r^2A_r\right)}{\partial r}
	+
	\frac{1}{r\sin \theta }\frac{\partial}{\partial \theta }\left(A_{\theta }\sin \theta \right)
	+
	\frac{1}{r\sin\theta}\frac{\partial A_{\phi}}{\partial \phi }                                                                                                                                                                                                                                                            \\
	\rot\vect{A}      & = \frac{1}{r\sin \theta }\left(\frac{\partial}{\partial \theta }\left(A_{\phi }\sin \theta \right)-\frac{\partial A_{\theta }}{\partial \phi }\right)\vect{e}_{r}
	+
	\frac{1}{r}\left(\frac{1}{\sin \theta }\frac{\partial A_{r}}{\partial \phi }-\frac{\partial}{\partial r}\left(rA_{\phi }\right)\right)\vect{e}_{\theta}
	+ \nonumber                                                                                                                                                                                                                                                                                                                    \\
	                  & + \frac{1}{r}\left(\frac{\partial}{\partial r}\left(rA_{\theta }\right)-\frac{\partial A_{r}}{\partial \theta }\right)\vect{e}_{\phi}
\end{align}
Орти сферичної системи координат зв'язані з декартовими ортами як:
\begin{align}\label{}
    \vect{e}_r &= \sin\theta\cos\phi\, \vect{e}_x + \sin\theta\sin\phi\, \vect{e}_y + \cos\theta\, \vect{e}_z, \\
    \vect{e}_{\theta} &= \cos\theta\cos\phi\, \vect{e}_x + \cos\theta\sin\phi\, \vect{e}_y - \sin\theta\, \vect{e}_z, \\
    \vect{e}_{\phi} &= -\sin\phi\, \vect{e}_x + \cos\phi\, \vect{e}_y. \\
\end{align}



%% --------------------------------------------------------
\subsection{Другі похідні}
%% --------------------------------------------------------

\begin{align}
	\mathrm{rot}\,\mathrm{grad}\,\phi    & = \rot(\vect{\nabla}\phi)  = 0                                             \\
	\mathrm{div}\,\mathrm{rot}\,\vect{A} & = \divg(\rot\vect{A})  = 0                                                 \\
	\mathrm{rot}\,\mathrm{rot}\,\vect{A} & = \rot(\rot\vect{A})  = \vect{\nabla}(\divg\vect{A}) - \Laplasian \vect{A}
\end{align}

%% --------------------------------------------------------
\subsection{Похідні від добутків}
%% --------------------------------------------------------

\begin{align}
	\mathrm{grad}\,(\phi \psi)             & = \psi\,\mathrm{grad}\,\phi +\phi\, \mathrm{grad}\,\psi                                                                                                                           \\
	\mathrm{div}\,(\phi \vect{A})          & = \phi\,\mathrm{div}\,\vect{A} + \vect{A}\,\mathrm{grad}\,\phi                                                                                                                    \\
	\mathrm{rot}\,(\phi \vect{A})          & = \phi\,\mathrm{rot}\,\vect{A} + \mathrm{grad}\,\phi \times \vect{A}                                                                                                              \\
	\mathrm{grad}\,(\vect{A}\cdot\vect{B}) & = \vect{B}\times\mathrm{rot}\,\vect{A} + \vect{A}\times\mathrm{rot}\,\vect{B} + \left( \vect{B}\vect{\nabla}\right)\vect{A} + \left( \vect{A}\vect{\nabla}\right)\vect{B}         \\
	\mathrm{div}\,(\vect{A}\times\vect{B}) & = \vect{B}\cdot\mathrm{rot}\,\vect{A} - \vect{A}\cdot\mathrm{rot}\,\vect{B}                                                                                                       \\
	\mathrm{rot}\,(\vect{A}\times\vect{B}) & = \left( \vect{B}\vect{\nabla}\right)\vect{A} - \left( \vect{A}\vect{\nabla}\right)\vect{B} + 	\vect{A}\,\mathrm{div}\,\vect{B} - \vect{B}\,\mathrm{div}\,\vect{A} \label{rotvect} \\
	\frac12\mathrm{grad}\,A^2              & =  \left( \vect{A}\vect{\nabla}\right)\vect{A} + \vect{A}\times\mathrm{rot}\,\vect{A}
\end{align}



%% --------------------------------------------------------
\subsection{Індексна нотація формул векторного аналізу}
%% --------------------------------------------------------

Нагадаємо співвідношення з векторного аналізу, що будуть потрібні далі.
Латинські індекси $i$, $j$, $k$ пробігатимуть допустимі значення $1$,$2$,$3$. Якщо індекси
у виразі повторюються, це означатиме суму по цих індексах (правило
Ейнштейна).

%% --------------------------------------------------------
\subsubsection{Тривимірний символ Леві-Чівіти}\label{Levi-Chiv}
%% --------------------------------------------------------

Тривимірний символ Леві-Чівіти визначений співвідношеннями
\begin{equation}\label{eq:Levi-Chiv}
\epsilon_{ijk} =  - \epsilon_{jik} =  - \epsilon_{ikj}.
\end{equation}

Деякі співвідношення з цим символом:
\begin{align}
    \epsilon_{ijk}\epsilon_{kpq} &= \delta_{ip}\delta_{jq} - \delta_{iq}\delta_{jp}, \\
    \epsilon_{iqk}\epsilon_{pqk} &= 2\delta _{ip}.
\end{align}

Ротор та векторний добуток у декартових координатах $\{x,y,z\} = \{x_1, x_2, x_3\}$:
\begin{align}
\left[\rot\vect{A} \right]_i &= \epsilon_{ijk}{\partial_j}{A_k}, \\
\left[ \vect{A} \times \vect{B} \right]_i &= \epsilon_{ijk}{A_j}{B_k},
\end{align}
всі індекси пробігають значення $1$, $2$, $3$.

%% --------------------------------------------------------
\subsubsection{Диференціальні операції в декартових координатах}
%% --------------------------------------------------------


У декартових координатах \( x_1, x_2, x_3 \) дивергенція:

\begin{equation*}
\Div \vect{A} \equiv \partial_j A_j;
\end{equation*}

градієнт:

\begin{equation*}
(\grad F)_i \equiv \partial_i F,
\end{equation*}

тут і надалі \((X)_i = X_i\) означає \(i\)-ту компоненту вектора \(\vect{X}\);

та оператор Лапласа:

\begin{equation*}
\Delta F \equiv \partial_j \partial_j F.
\end{equation*}



За допомогою \eqref{eq:Levi-Chiv} можна записати \(i\)-ту компоненту векторного добутку:

\begin{equation*}
[\vect{A} \times \vect{B}]_i = \varepsilon_{ijk} A_j B_k
\end{equation*}

\((A_i, B_i\) --- компоненти векторів \(\vect{A}\) та \(\vect{B})\), а також ротор:

\begin{equation*}
\Rot(\vect{A})_i = \varepsilon_{ijk} \partial_j A_k.
\end{equation*}

%% --------------------------------------------------------
\subsubsection{Добуток скалярної та векторної функцій}
%% --------------------------------------------------------

Легко перевірити такі співвідношення для добутку скалярної функції \( F(\vect{r}) \) та векторної \( \vect{A}(\vect{r}) \):

\begin{equation*}
\Div(F \vect{A}) = \partial_i (F A_i) = A_i \partial_i F + F \partial_i A_i = \vect{A} \cdot \grad F + F \Div \vect{A};
\end{equation*}

\begin{equation*}
\Rot(F \vect{A})_i = \varepsilon_{ijk} \partial_j (F A_k) = \varepsilon_{ijk} (\grad F)_j A_k + F \varepsilon_{ijk} \partial_j A_k,
\end{equation*}

або

\begin{equation*}
\Rot(F \vect{A}) = [\grad F \times \vect{A}] + F \Rot \vect{A}.
\end{equation*}

\subsubsection{Дивергенція векторного добутку}

Дивергенція векторного добутку:

\begin{multline*}
\Div[\vect{A} \times \vect{B}] = \partial_i (\varepsilon_{ijk} A_j B_k) = B_k \varepsilon_{ijk} \partial_i A_j + A_j \varepsilon_{ijk} \partial_i B_k =
B_k \varepsilon_{kij} \partial_i A_j - A_j \varepsilon_{ijk} \partial_i B_k = \\ = \vect{B} \cdot \Rot \vect{A} - \vect{A} \cdot \Rot \vect{B}.
\end{multline*}

%% --------------------------------------------------------
\subsubsection{Згортка символів \texorpdfstring{\(\epsilon_{ijk}\)}{}}
%% --------------------------------------------------------

Далі будуть потрібні формули для \emph{згортки двох символів} \(\varepsilon_{ijk}\). Користуючись властивостями визначника, легко перевірити, що:

\begin{equation*}
\varepsilon_{ijk} =
\begin{vmatrix}
\delta_{i1} & \delta_{i2} & \delta_{i3} \\
\delta_{j1} & \delta_{j2} & \delta_{j3} \\
\delta_{k1} & \delta_{k2} & \delta_{k3}
\end{vmatrix},
\end{equation*}

де \(\delta_{ij}\) --- символ Кронекера, а також:

\begin{equation*}
\varepsilon_{ijk} \varepsilon_{pqr} =
\begin{vmatrix}
\delta_{ip} & \delta_{iq} & \delta_{ir} \\
\delta_{jp} & \delta_{jq} & \delta_{jr} \\
\delta_{kp} & \delta_{kq} & \delta_{kr}
\end{vmatrix}
\end{equation*}

Якщо в останньому співвідношенні розкрити визначник за правилом трикутників:

\begin{equation*}
\varepsilon_{ijk} \varepsilon_{pqr} = \delta_{ip} \delta_{jq} \delta_{kr} + \delta_{iq} \delta_{jr} \delta_{kp} + \delta_{ir} \delta_{jp} \delta_{kq} -
\delta_{ip} \delta_{jr} \delta_{kq} - \delta_{iq} \delta_{jp} \delta_{kr} - \delta_{ir} \delta_{jq} \delta_{kp},
\end{equation*}

та взяти суму по \( k = r \), отримаємо формулу згортки:

\begin{equation}
\varepsilon_{ijk} \varepsilon_{pqk} = \delta_{ip} \delta_{jq} - \delta_{iq} \delta_{jp}.
\label{eq:epsilon_contraction}
\end{equation}

%% --------------------------------------------------------
\subsubsection{Подвійний векторний добуток}
%% --------------------------------------------------------

Отримаємо за допомогою \eqref{eq:epsilon_contraction} правило для подвійного векторного добутку:

\begin{equation*}
([\vect{A} \times [\vect{B} \times \vect{C}]])_i \equiv \varepsilon_{ijk} A_j ([\vect{B} \times \vect{C}])_k \equiv \varepsilon_{ijk} A_j
\varepsilon_{kpq} B_p C_q.
\end{equation*}

Завдяки \eqref{eq:epsilon_contraction}:

\begin{equation*}
([\vect{A} \times [\vect{B} \times \vect{C}]])_i \equiv (\delta_{ip} \delta_{jq} - \delta_{iq} \delta_{jp}) A_j B_p C_q = B_i A_q C_q - C_i A_p B_p,
\end{equation*}

тобто маємо відому формулу:

\begin{equation*}
[\vect{A} \times [\vect{B} \times \vect{C}]] = \vect{B}(\vect{A} \cdot \vect{C}) - \vect{C}(\vect{A} \cdot \vect{B}).
\end{equation*}


%% --------------------------------------------------------
\subsubsection{Подвійний ротор}
%% --------------------------------------------------------

Обчислимо за допомогою \eqref{eq:epsilon_contraction} подвійний ротор:

\begin{equation*}
(\Rot \Rot \vect{A})_i = \varepsilon_{ijk} \partial_j (\Rot \vect{A})_k = \varepsilon_{ijk} \partial_j \varepsilon_{kpq} \partial_p A_q.
\end{equation*}

Завдяки \eqref{eq:epsilon_contraction}:

\begin{equation*}
(\Rot \Rot \vect{A})_i = (\delta_{ip} \delta_{jq} - \delta_{iq} \delta_{jp}) \partial_j \partial_p A_q = \partial_i \partial_j A_j - \partial_j
\partial_j A_i.
\end{equation*}

За означенням, \(\partial_j A_j \equiv \Div \vect{A}\) та \(\partial_j \partial_j A_i \equiv \Delta A_i\). Звідси маємо:

\begin{equation*}
\Rot \Rot \vect{A} = \grad \Div \vect{A} - \Delta \vect{A}.
\end{equation*}

%% --------------------------------------------------------
\subsection{Інтегральні характеристики та теореми}
%% --------------------------------------------------------

Теорема Ос\-тро\-град\-сько\-го-Гаусса:
\begin{equation}\label{OGTheorem}
	\oiint\limits_{\partial \Omega} \vect{A}\cdot d\vect{S} = \iiint\limits_{\Omega} \divg\vect{A} dV,
\end{equation}
де $\Omega$~--- об'єм, $\partial\Omega$ --- його межа.

Теорема Стокса:
\begin{equation}\label{Stoksheorem}
	\oint\limits_{\partial S} \vect{A}\cdot d\vect{l} = \iint\limits_S \rot\vect{A} \cdot d\vect{S},
\end{equation}
де $S$~--- поверхня, що спирається на контур $\partial S$.

Теорема Гріна:
\begin{equation}\label{Grin}
	\iiint\limits_{\Omega}(\phi \nabla ^{2}\psi -\psi \nabla ^{2}\phi )\ dV=\iint \limits_{\partial \Omega}(\phi \vect{\nabla} \psi -\psi \vect{\nabla}
	\phi )\cdot d\vect{S}.
\end{equation}

%% --------------------------------------------------------
\section{Поліноми Лежандра}\label{Polinoms}
%% --------------------------------------------------------

Поліноми Лежандра застосовуються у теорії потенціалу при розкладанні виразу в околі точки $\vect{r}$:
\begin{equation*}\label{}
    \frac{1}{|\vect{r} - \vect{r}_0|} = \frac{1}{\sqrt{r^2 - 2rr_0 \cos\chi + r_0^2 }}  =  \sum\limits_{l = 0}^{\infty} \frac{r^l_<}{r^{l+1}_>} P_l(\cos\chi),
\end{equation*}
де $r_>$, $r_<$~--- більша і менша із величин $|\vect{r}|$ та $\vect{r}_0$, відповідно, $\cos\chi$~--- кут між векторами $\vect{r}$ та $|\vect{r}_0|$.

\medskip%
\textbf{Деякі поліноми Лежандра}

\begin{align*}
P_{0}(\cos\chi)  = 1, &\quad P_{1}(\cos\chi)  = \cos\chi, \\
P_{2}(\cos\chi)  = \frac {1}{2}(3\cos^2\chi-1), &\quad P_{3}(\cos\chi)  = \frac {1}{2}(5\cos^2\chi-3\cos\chi).
\end{align*}

%% --------------------------------------------------------
\section{Сферичні гармоніки}\label{Spherical_Harmonics}
%% --------------------------------------------------------

Сферичні функції, що залежать від полярних кутів визначаються формулою:
\begin{equation*}
    Y_{lm}(\theta,\phi) = (-1)^{(m+|m|)/2} \sqrt{\frac{2l+1}{4\pi}\frac{(l-m)!}{(l+m)!}}P_l^{|m|}(\cos \theta) e^{im\phi},
\end{equation*}
де $l = 0,1,2, \ldots$, $m$ пробігає значення від $-l$ до $l$, а $P_l^{|m|}(x)$~--- приєднані функції Лежандра.

Вони утворюють повну ортонормовану систему функцій:
\[
    \int\limits_{\theta=0}^{\pi} \int\limits_{\phi=0}^{2\pi} Y^*_{l',m'}(\theta, \phi) Y_{l,m}(\theta, \phi) \sin\theta d\theta d\phi  = \delta_{l,l'} \delta_{m,m'}.
\]

Деякі сферичні гармоніки:
\begin{align*}
	Y_{0,0}(\theta,\phi)     & =\sqrt{1\over 4\pi},                                            \\
	Y_{1,\pm 1}(\theta,\phi) & =\sqrt{3\over 8\pi} \, \sin\theta \, e^{\pm i\phi},             \\
	Y_{1,0}(\theta,\phi)     & =\sqrt{3\over 4\pi}\, \cos\theta ,                              \\
	Y_{2,0}(\theta,\phi)     & =\sqrt{5\over 16\pi}\, (3\cos^{2}\theta-1),                     \\
	Y_{2,\pm 1}(\theta,\phi) & =\sqrt{15\over 8\pi}\, \sin\theta\, \cos\theta\, e^{\pm i\phi}, \\
	Y_{2,\pm 2}(\theta,\phi) & =\sqrt{15\over 32\pi} \, \sin^{2}\theta \, e^{\pm2i\phi}.
\end{align*}

%% --------------------------------------------------------
\section{Циліндричні функції}\setcounter{equation}{0}
%% --------------------------------------------------------

Рівняння, що виникають в задачах з циліндричною симетрією, мають вигляд:
	\begin{equation}\label{eq:Bessel_eq}
		\frac{d^2y}{dx^2} + \frac1x\frac{dy}{dx} + \left(1 - \frac{m^2}{x^2} \right) y = 0,
	\end{equation}
	розв'язок яких можна представити за допомогою функцій Бесселя $J_m(x)$ та Неймана $N_m$ у вигляді лінійної комбінації $y(x) = A J_m(x) + B N_m(x)$ або у вигляді лінійної комбінації $y(x) = A H^{(1)}_m(x) + B H^{(2)}_m(x)$, де функції $H_m^{(1,2)} =J_m \pm i N_m$~--- називаються функціями Ганкеля 1-го та 2-го роду, відповідно. Доцільність введення функцій Ганкеля обумовлена тим, що вони мають прості асимптотичні розкладання при $|x| \gg 1$ і зручні для задач, пов'язаних з поширенням хвиль.

Для $m = 0,1,2,\ldots$ функції Неймана нескінченні в точці $x = 0$, тобто $\lim\limits_{x\to0}N_m(x) = -\infty$.

Функції Бесселя можна представити за допомогою ряду (в околі точки $x = 0$ для цілих, або невід'ємних $m$):
\begin{equation}\label{eq:J}
    J_m(x) = \sum_{n=0}^\infty \frac{(-1)^n}{n! \Gamma(n+m+1)} {\left({\frac{x}{2}}\right)}^{2n+m},
\end{equation}
де $ \Gamma$~--- \href{https://en.wikipedia.org/wiki/Gamma_function}{гамма-функція}. Для $m \in \mathbb{Z}$ має місце рівність $J_{-m}(x)  = (-1)^m J_m(x)$.

\begin{center}
  \begin{tikzpicture}
    \begin{axis}[axis lines = middle,
			axis line style={-stealth},
			minor grid style = {line width=.1pt,draw=gray!10},
            width=\textwidth, height=0.5*\textwidth, xlabel=$x$,
            xtick=\empty,
%            ytick={-0.5,0.5,1},
            legend style={draw=none},
    ]
    \addplot+[id=parable,domain=-20:20, samples=500, mark=none, width=2pt, color=red, thick]
    gnuplot{besj0(x)};% node[pin=95:{$J_0(x)$}]{};
    \addplot+[id=parable,domain=-20:20, samples=500, mark=none, width=2pt, color=blue, thick]
    gnuplot{besj1(x)};% node[pin=130:{$J_1(x)$}]{};
    \addplot+[id=parable2,domain=-20:20, samples=500, mark=none, width=2pt, color=green!50!black, thick]
    gnuplot{2*1/x*besj1(x)-besj0(x)};% node[pin=-140:{$J_2(x)$}]{};
    \legend{$J_0(x)$,$J_1(x)$,$J_2(x)$}
   \end{axis}
  \end{tikzpicture}

    {Графіки функцій Бесселя $J_m$ для $m = 0,1,2$.}
\end{center}

\begin{center}
  \begin{tikzpicture}
    \begin{axis}[axis lines = middle,
			axis line style={-stealth},
			minor grid style = {line width=.1pt,draw=gray!10},
            width=\textwidth, height=0.5*\textwidth, xlabel=$x$,
            xtick=\empty,
%            ytick={-0.5,0.5,1},
            legend style={at={(current axis.south east)}, anchor = south east, draw=none},
    ]
    \addplot+[id=parable,domain=0:40, restrict y to domain=-1.5:1, samples=500, mark=none, width=2pt, color=red, thick]
    gnuplot{besy0(x)};% node[pin=95:{$J_0(x)$}]{};
    \addplot+[id=parable,domain=0:40, restrict y to domain=-1.5:1, samples=1000, mark=none, width=2pt, color=blue, thick]
    gnuplot{besy1(x)};% node[pin=130:{$J_1(x)$}]{};
    \addplot+[id=parable2,domain=0:40, restrict y to domain=-1.5:1, samples=1000, mark=none, width=2pt, color=green!50!black, thick]
    gnuplot{2*1/x*besy1(x)-besy0(x)};% node[pin=-140:{$J_2(x)$}]{};
    \legend{$N_0(x)$,$N_1(x)$,$N_2(x)$}
   \end{axis}
  \end{tikzpicture}

    {Графіки функцій Неймана $N_m$ для $m = 0,1,2$.}
\end{center}

Функції Неймана визначаються через функції Бесселя як:
\begin{equation}\label{eq:NG}
    N_m(x) = \frac{J_m(x)\cos m\pi - J_{-m}(x)}{\sin m\pi}.
\end{equation}


Деякі рекурентні співвідношення:
\begin{equation}
    J_{m + 1}(x) + J_{m - 1}(x) = \frac{2m}{x}J_m(x).
\end{equation}

Деякі диференціальні та інтегральні співвідношення для нецілих $m$ (для цілих $m$ ці функції можна визначити за допомогою граничного переходу).:
\begin{align}
    \frac{d}{dx} J_0(x) = - J_1(x), \\
    \frac{d}{dx} \left( x^{-m}J_m(x)\right)  = - x^{-m}J_{m+1}(x), \\
     \int\limits_0^{x} x^{\prime m+1} J_m(x') dx' &= x^{m+1}J_{m+1}. \label{eq:recInt}
\end{align}

Інтеграли від добутків:
\begin{equation}
    \int\limits_0^x  J_m(k_1x')J_m(k_2x') x' dx' = \frac{x\left( k_2J_m(k_1x)J'_m(k_2x) - k_1J_m(k_2x)J'_m(k_1x)\right)}{k_1^2-k_2^{2}}   \label{eq:JJ0*}.
\end{equation}

В задачах, зазвичай, часто необхідно знайти наближений вигляд циліндричних функцій при малих та великих значеннях аргументу $x$:

при $|x| \ll 1$ з~\eqref{eq:J}

\begin{equation}
    J_0(x) \approx 1 - \frac{x^2}{4}, \quad
    J_m \approx \frac{x^m}{2^m m!}, \ m \ge 1,\, m \in \mathbb{N};
\end{equation}

при $|x| \gg 1$

\begin{align}
    J_m &\approx \sqrt{\frac{2}{\pi x}}  \cos\left( x - m\frac{\pi}{2} - \frac{\pi}{4}\right),
      \label{eq:Jxgg1}\\
    N_m &\approx \sqrt{\frac{2}{\pi x}}  \sin\left( x - m\frac{\pi}{2} - \frac{\pi}{4}\right),
  \label{eq:Yxgg1}\\
    H_m^{(1,2)} &\approx \sqrt{\frac{2}{\pi x}}  e^{\pm i \left( x - m\frac{\pi}{2} - \frac{\pi}{4}\right) }. \label{eq:Hxgg1}
\end{align}

Співвідношення Якобі-Ангера (розкладання за функціями Бесселя):
\begin{equation}\label{eq:JacobiAnger}
    e^{ix\cos\theta} = \sum\limits_{m= - \infty}^{\infty} i^mJ_m(x)e^{im\theta}, \quad
    e^{ix\sin\theta} = \sum\limits_{m= - \infty}^{\infty} J_m(x)e^{im\theta}. \\
\end{equation}

%% --------------------------------------------------------
\section{Узагальнені функції}\setcounter{equation}{0}
%% --------------------------------------------------------

%Дельта-функція Дірака (або $\delta$-функція) є узагальненою функцією і була введена фізиком Полем Діраком для моделювання густини ідеалізованої
%точкової маси або точкового заряду.
%
%На <<фізичному рівні строгості>> можна визначити $\delta$-функцію формальним співвідношенням:
%\begin{equation}
%    \int\limits_{-\infty}^{+\infty}f(x)\delta(x-x_0)\,dx=f(x_0),
%\end{equation}
%У випадку інтегрування по скінченному об'єму $V$:
%\begin{equation}\label{eq:delta3D}
%    \int\limits_{V}f(\vect{r})\delta(\vect{r} - \vect{r}_0) dV=f(\vect{r}_0),
%\end{equation}
%де точка $\vect{r}_0$ знаходиться всередині об'єму $V$.

%$\delta$-функцію однієї дійсної змінної можна визначити як функцію, що задовольняє наступним умовам:
%\begin{equation}\label{eq:delta}
%\delta(x - x_0)=\left\{\begin{matrix}
%   +\infty, & x=x_0, \\
%   0, & x \neq x_0; \\
%\end{matrix}\right.
%\end{equation}
%\begin{equation}\label{eq:deltaProp0}
%    \int\limits_{-\infty}^{+\infty}\delta(x-x_0)dx=1.
%\end{equation}
%Тобто ця функція не дорівнює нулю тільки в точці $x = x_0$, де вона перетворюється в нескінченність таким чином, щоб її інтеграл в будь-якому околі точки $x = x_0$ дорівнює $1$.
%
%Аналітичне представлення $\delta$-функції:
%\begin{equation}\label{eq:deltaanalit}
%    \delta(x) = \frac{1}{2\pi}\int\limits_{-\infty}^{\infty} e^{ikx}dk.
%\end{equation}
%\bigskip\noindent%
%\textbf{Властивості дельта-функції}
%\bigskip
%
%\begin{enumerate}[label=\alph*)]
%\item Дельта-функція парна $\delta(-x) = \delta(x)$,
%\item $x\delta(x) = 0$,
%\item $\delta(ax) = \frac{1}{|a|}\delta(x)$,
%\item  $x\delta^\prime(x)=-\delta(x)$,
%\item $\delta(f(x))=\sum\limits_k\frac{\delta(x-x_k)}{|f'(x_k)|}$, де $x_k$~--- нулі функції $f(x)$,
%
%\end{enumerate}



В задачах електродинаміки часто виникає необхідність розглядати заряджені тіла, розміри яких дуже малі у порівнянні з іншими просторовими масштабами. З
цим пов’язана модель точкового (сферичного) заряду, густину якого подають у вигляді:

\begin{equation}
\rho(\vect{r}) = q \delta(\vect{r} - \vect{r}_0),
\end{equation}

де \( q \) --- величина заряду, \( \vect{r} \) --- його положення, \( \delta \) --- функція Дірака, що визначається умовою:

\begin{equation}
\int dV \, \delta(\vect{r} - \vect{r}_0) \chi(\vect{r}) = \chi(\vect{r}_0)
\label{eq:dirac_definition}
\end{equation}

для будь-якої неперервної функції \( \chi \).

Очевидно, що визначення \eqref{eq:dirac_definition} \(\delta\)-функції не є математично коректним, якщо вважати, що в \eqref{eq:dirac_definition} маємо
справу із звичайним інтегруванням, наприклад, за Лебегом. Цьому співвідношенню можна надати математичний зміст, якщо розглядати \(\delta\)-функцію як
слабку границю деякої послідовності звичайних функцій. Якщо покласти (в одновимірному варіанті):

\begin{equation}
\delta_{\varepsilon}(x) = \frac{1}{\varepsilon \sqrt{\pi}} e^{-x^2 / \varepsilon^2},
\label{eq:delta_sequence}
\end{equation}

то границя:

\begin{equation}
\lim_{\varepsilon \to 0} \int dx \, \delta_{\varepsilon}(x) \chi(x) = \chi(0)
\label{eq:delta_limit}
\end{equation}
існує для будь-якої послідовності \( \varepsilon_n \to 0 \), \( n \to \infty \) та будь-якої функції \( \chi(x) \), що задовольняє досить широким умовам
(наприклад, якщо \( \chi(x) \) --- гладка та обмежена). На практиці більш зручно розглядати подібні границі за більш жорстких обмежень на \( \chi(x) \).

Таке визначення цілком відповідає фізичним уявленням про точковий заряд, зосереджений в нескінченно малій області. Насправді, для багатьох застосувань
істотно лише те, що розміри заряду значно менші за відстань до нього. Разом із тим треба пам’ятати, що ці міркування не проходять, коли треба розглядати
співвідношення, нелінійні за густиною заряду або за напруженостями полів; наприклад, в формулах для енергії.

\subsubsection{Узагальнені та основні функції (\( x \in \mathbb{R}^n \))}

Узагальнену \(\delta\)-функцію можна визначити за допомогою іншої, відмінної від \eqref{eq:delta_sequence}, послідовності, тобто означення
\eqref{eq:delta_sequence}, \eqref{eq:delta_limit} не є єдиним (див. приклади далі). Але усі подібні співвідношення визначають лінійний неперервний
функціонал, що зіставляє функції \( \chi(x) \) число \( \chi(0) \). Далі розглянемо лінійні неперервні функціонали на деякій множині функцій, які будемо
називати \textit{основними}.

Нехай множина основних функцій --- це множина \( D \) фінітних (тобто відмінних від нуля у деяких обмежених областях) нескінченно диференційовних
функцій. Збіжність в \( D \) визначають так: послідовність \( \{\phi_k\} \subset D \) збігається до \( \phi \in D \), якщо всі \( \phi_k \) можуть бути
відмінними від нуля лише у деякій спільній обмеженій області та усі похідні рівномірно збігаються до відповідних похідних від \( \phi \).

Узагальнена функція --- це лінійний неперервний функціонал, означений на \( D \). Множину таких узагальнених функцій позначають через \( D' \).

Існує багато способів регуляризації дельта-функції Дірака; зокрема, використовуються наступні наближення:

\begin{equation}
\delta_{\varepsilon}(x) = \frac{1}{\pi} \frac{\sin(x/\varepsilon)}{x},
\label{eq:delta_sin}
\end{equation}

\begin{equation}
\delta_{\varepsilon}(x) = \frac{1}{\pi} \frac{\varepsilon}{x^2 + \varepsilon^2},
\label{eq:delta_lorentz}
\end{equation}

\begin{equation}
\delta_{\varepsilon}(x) = \frac{1}{\pi} \frac{\sin^2(x/\varepsilon)}{x^2},
\label{eq:delta_sin2}
\end{equation}
де \( x \in \mathbb{R} \).

Далі будемо позначати значення функціоналу, що відповідає узагальненій функції \( F \in D' \) на основній функції \( \chi \in D \), як \( (F, \chi) \).
Наприклад, співвідношення \( (\delta, \chi) = \chi(0) \) визначає \(\delta\)-функцію Дірака. Узагальнену функцію \( F(x) \) (\( x \in \mathbb{R}^n \))
називають \textit{регулярною}, якщо існує інтегровна функція \( f(x) \), така, що відповідний функціонал можна подати у вигляді звичайного інтегралу
Лебега:

\begin{equation}
(F, \chi) = \int dx \, f(x) \chi(x).
\label{eq:regular_functional}
\end{equation}

Якщо це неможливо, \( F \) називають \textit{сингулярною узагальненою функцією}. Прикладом такої функції є \(\delta\)-функція Дірака. Тим не менш, в
фізичній літературі для сингулярних функцій також використовують запис \eqref{eq:regular_functional}, маючи на увазі певний граничний перехід --- типу
\eqref{eq:delta_limit} або інший. Можна показати, що будь-яку узагальнену функцію можна подати як слабку границю послідовності основних функцій з~\( D
\).

Далі ми також будемо застосовувати формальний запис \eqref{eq:regular_functional}, пам’ятаючи про вказані застереження.

\subsubsection{Диференціювання і фундаментальні розв’язки}

Визначимо похідну \( \partial_i F(x) \) від узагальненої функції \( F \) співвідношенням:

\begin{equation}
(\partial_i F, \chi) = - (F, \partial_i \chi), \quad x \in \mathbb{R}^n.
\label{eq:derivative_definition}
\end{equation}

Для регулярної диференційовної функції \( F \) це співвідношення є очевидним наслідком інтегрування за частинами:

\begin{equation*}
\int dx \, \partial_i F(x) \chi(x) = - \int dx \, F(x) \partial_i \chi(x)
\end{equation*}

з урахуванням того, що \( \chi(x) \) є фінітною.

Нехай \( A \) --- диференціальний оператор скінченого порядку із сталими коефіцієнтами, \( x \in \mathbb{R}^n \).

\emph{Фундаментальним розв’язком оператора \( \hat{L} \)} називають узагальнену функцію \( G \), таку, що:

\begin{equation}
\hat{L} G = \delta(x).
\label{eq:fundamental_solution}
\end{equation}


\subsubsection{Фундаментальний розв’язок оператора Лапласа}

Нехай  \( x = \vect{r} = (x_1, x_2, x_3) \in \mathbb{R}^3 \), покажемо, що:

\begin{equation}
\Delta \left( -\frac{1}{r} \right) = -4\pi \delta(\vect{r}),
\label{eq:laplace_fundamental}
\end{equation}

де в правій частині фігурує тривимірна \(\delta\)-функція,

\begin{equation*}
r = |\vect{r}| = \sqrt{x_1^2 + x_2^2 + x_3^2}.
\end{equation*}

За означенням похідної:

\begin{equation*}
\int \Delta \left( \frac{1}{r} \right) \chi(\vect{r}) dV = \int \frac{1}{r} \Delta \chi dV
\end{equation*}

Права частина означена як звичайний невласний інтеграл, який можна обчислювати в сферичних координатах:

\begin{multline*}
\int \frac{\Delta \chi}{r}  r^2 \sin \theta \, d\theta \, d\varphi \, dr = \\
=
\int \frac{1}{r} \left[ \frac{\partial}{\partial r} \left( r^2
\frac{\partial \chi}{\partial r} \right) + \frac{1}{\sin \theta} \frac{\partial}{\partial \theta} \left( \sin \theta \frac{\partial \chi}{\partial
\theta} \right) + \frac{1}{\sin^2 \theta} \frac{\partial^2 \chi}{\partial \varphi^2} \right] \sin \theta \, d\theta \, d\varphi =
\\=
\int \frac{1}{r}
\frac{\partial}{\partial r} \left( r^2 \frac{\partial \chi}{\partial r} \right) dr d\Omega, \quad d\Omega = \sin \theta \, d\theta \, d\varphi.
\end{multline*}

Тут враховано, що:

\begin{equation*}
\int\limits_0^\pi d\theta \, \frac{\partial}{\partial \theta} \left( \sin \theta \frac{\partial \chi}{\partial \theta} \right) = \left( \sin \theta
\frac{\partial \chi}{\partial \theta} \right)_0^\pi = 0, \quad \int\limits_0^{2\pi} d\varphi \, \frac{\partial^2 \chi}{\partial \varphi^2} =
\frac{\partial
\chi}{\partial \varphi} \bigg|_0^{2\pi} = 0.
\end{equation*}

Залишається інтеграл, що береться по \( r \) при фіксованих \( \varphi \), \( \theta \):

\begin{multline*}
\int\limits_0^\infty d\Omega \int\limits_0^r dr \, \frac{1}{r} \frac{\partial}{\partial r} \left( r^2 \frac{\partial \chi}{\partial r} \right) =
\int\limits_0^\infty d\Omega
\int\limits_0^r dr \, \frac{\partial}{\partial r} \left( r \frac{\partial \chi}{\partial r} + \chi \right) = \left.\left( r \frac{\partial
\chi}{\partial r} +
\chi
\right)\right|_0^\infty = \\ =- \int d\Omega \, \chi(0) = -4\pi \chi(0).
\end{multline*}

Отриманий вираз доводить співвідношення \eqref{eq:laplace_fundamental}.

\subsubsection{Фундаментальний розв’язок оператора Даламбера}

Нехай \( c = 1 \), \( \Dalambertian = \frac{\partial^2}{\partial t^2} - \Delta \), \( x = (t, x_1, x_2, x_3) \in \mathbb{R}^4 \).

Визначимо функцію \( G(x) \) співвідношенням:

\begin{equation}
G(x) = \frac{1}{2\pi} \delta \left[ t^2 - x^2 \right] \theta(t) = \frac{1}{4\pi r} \delta \left( t - r \right),
\label{eq:dalambert_fundamental}
\end{equation}

де \( r = |\vect{x}| = \sqrt{x_1^2 + x_2^2 + x_3^2} \).

%Покажіть, що \( (\Dalambertian G, \chi) = \chi(0) \).
%
%Вказівка: \( (\Dalambertian G, \chi) = (\Dalambertian G, \nabla) \). Перейдіть до сферичних координат і покажіть, що внесок кутової частини оператора
%Лапласа дає
%нуль; потім врахуйте тотожність
%
%\begin{equation*}
%r (\chi_{tt}(r, r) - \chi_{rr}(r, r)) - 2 \chi_r(r, r) = \frac{d}{dr} \left\{ r (\chi_t(r, r) - \chi_r(r, r)) - \chi(r, r) \right\},
%\end{equation*}
%де \( \chi = \chi(t, r) \).

\subsubsection{Фундаментальний розв’язок оператора Гельмгольца}

Покажіть, що:

\begin{equation}
\left( \Delta + k^2 \right) \frac{e^{ikr}}{r}, \quad \chi = \int \frac{e^{ikr}}{r} (\Delta + k^2) \chi \, dV = -4\pi \chi(0).
\label{eq:helmholtz_fundamental}
\end{equation}

\subsubsection{Згортка}

Згортка \( \zeta = F * \chi \) узагальненої функції \( F \) з основною \( \chi \) є за визначенням:

\begin{equation}
\zeta(y) = \int dx \, F(x) \chi(y - x).
\label{eq:convolution}
\end{equation}

За допомогою згортки та фундаментального розв’язку можна будувати розв’язки рівнянь виду:

\begin{equation}
\hat{L} \phi = f, \quad f \in D.
\end{equation}

Дійсно, якщо покласти \( \varphi = G * f \), де \( \hat{L} G = \delta(x) \), маємо:

\begin{equation*}
\hat{L}\phi(y) = \int dx \, G(x) \hat{L}_y f(y - x) = \int dx \, \hat{L}_x G(x) f(y - x) = (\delta(x), f(y - x)) = f(y)
\end{equation*}
(тут введено індекси, щоб підкреслити, що \( \hat{L}_x \) діє на змінну \( x \), \( \hat{L}_y \) --- на \( y \)). Цей результат за певних умов можна
розширити на
випадок \( f \in D \).

Звідси маємо такі розв’язки рівнянь:

\begin{equation}
\Delta \phi = -4\pi \rho \implies \phi = \int dV' \, \frac{\rho(\vect{r}')}{|\vect{r} - \vect{r}'|}, \quad \phi = \varphi(\vect{r}),
\end{equation}

\begin{equation}
\Dalambertian \phi = 4\pi \rho \implies \phi = \int \frac{\rho(t - |\vect{r} - \vect{r}'|, \vect{r}')}{|\vect{r} - \vect{r}'|} dV' ,
\quad
\phi = \phi(t, \vect{r}'),
\end{equation}

\begin{equation}
\Delta \phi + k^2 \phi = -4\pi \rho \implies \varphi(\vect{r}) = \int \frac{e^{ik|\vect{r} - \vect{r}'|}}{|\vect{r} - \vect{r}'|}
\rho{\vect{r}'}
dV'.
\end{equation}

