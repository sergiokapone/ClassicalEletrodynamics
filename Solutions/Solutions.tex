% !TeX program = lualatex
% !TeX encoding = utf8
% !TeX spellcheck = uk_UA
% !TeX root =../ClassicalEletrodynamics.tex

%=========================================================
\chapter{Розв’язки рівнянь Максвелла}\label{\currfilebase}
%=========================================================

%% --------------------------------------------------------
\section{Потенціали електромагнітного поля}
%% --------------------------------------------------------

Існує досить багато методів аналізу рівнянь Максвелла \eqref{eq:M1D} -- \eqref{eq:M4D}, серед яких одним з найбільш поширених є введення потенціалів, що
дозволяють дещо
зменшити число невідомих функцій.

З умови соленоїдальності магнітного поля \eqref{eq:M2D} випливає, що існує деяке векторне поле $\vect{A}$, таке, що:
\begin{equation}\label{eq:potB}
	 \Bfield = \Rot\vect{A}
\end{equation}
Підставляючи це в \eqref{eq:M3D}, після елементарних перетворень маємо:
\begin{equation*}
	\Rot \left( \Efield + \frac{1}{c} \frac{\partial \vect{A}}{\partial t}\right)  = 0,
\end{equation*}
звідки випливає існування скалярного поля $\phi$, такого, що
\begin{equation}\label{eq:potE}
	\Efield = -\nabla \phi - \frac{1}{c} \frac{\partial \vect{A}}{\partial t}.
\end{equation}

Таким чином, якщо шукати електромагнітне поле у вигляді \eqref{eq:potB}, \eqref{eq:potE}, рівняння \eqref{eq:M2D}, \eqref{eq:M3D} виконуються
автоматично.

%% --------------------------------------------------------
\subsection*{Калібрувальна інваріантність}
%% --------------------------------------------------------

Формули \eqref{eq:potB}, \eqref{eq:potE} не визначають потенціали однозначно. Розглянемо перетворення $(\phi, \vect{A}) \to (\phi', \vect{A}')$:
\begin{equation}\label{eq:A'}
	\vect{A}' = \vect{A} + \nabla \chi,
\end{equation}
\begin{equation}\label{eq:phi'}
	\phi' = \phi - \frac{1}{c} \frac{\partial \chi}{\partial t}.
\end{equation}
Підставляючи в \eqref{eq:potB}, \eqref{eq:potE}, бачимо, що напруженості електромагнітного
поля виражаються через $\phi'$, $\vect{A}'$ подібно до $\phi$, $\vect{A}$:

\begin{equation}
	\Bfield = \Rot \vect{A}',
\end{equation}
\begin{equation}
	\Efield = -\nabla \phi' - \frac{1}{c} \frac{\partial \vect{A}'}{\partial t}.
\end{equation}

Таким чином, потенціали $(\phi, \vect{A})$ містять деякі степені свободи, що ніяк не
впливають на фізичну ситуацію. Цю обставину називають \emph{калібрувальною
інваріантністю} рівнянь поля, а перетворення \eqref{eq:A'}, \eqref{eq:phi'}, або інші, що не
впливають на спостережувані величини $(\Efield, \Bfield)$, називають \emph{калібрувальними
перетвореннями}.


%% --------------------------------------------------------
\subsection*{Калібрувальна умова Лоренца}
%% --------------------------------------------------------

Калібрувальна інваріантність дозволяє накладати додаткові умови на
потенціали $(\phi, \vect{A})$, за допомогою яких можна привести рівняння до більш
зручного вигляду.

Розглянемо умову Лоренца:
\begin{equation}\label{eq:LorenzCondition}
	\Div \vect{A} + \frac{1}{c} \frac{\partial \phi}{\partial t} = 0.
\end{equation}
Цю умову можна задовольнити за допомогою перетворень \eqref{eq:A'}, \eqref{eq:phi'}.
Дійсно, припустимо, що:
\begin{equation*}
	\Div \vect{A} + \frac{1}{c} \frac{\partial \phi}{\partial t} = f \neq 0?
\end{equation*}
та перейдемо до нових потенціалів $(\phi', \vect{A}')$ за формулами \eqref{eq:A'}, \eqref{eq:phi'}. Тоді:
\begin{equation*}
    \Div \vect{A}' + \frac{1}{c} \frac{\partial \phi'}{\partial t} =
	f - \frac{1}{c^2} \pparttime{\chi} + \nabla^2\chi.
\end{equation*}
Підбираючи функцію $\chi$ таким чином, щоб $f = \mdlgwhtsquare\  \chi$
(де $\mdlgwhtsquare\  = \nabla^2 - \frac{1}{c^2} \frac{\partial^2}{\partial t^2}$ – оператор Даламбера), бачимо, що нові потенціали $(\phi',
	\vect{A}')$
задовольняють калібрувальній умові Лоренца \eqref{eq:LorenzCondition}.

Отримаємо рівняння для $(\phi, \vect{A})$, припускаючи тепер, що умова \eqref{eq:LorenzCondition}
виконана. З подальшого буде видно, що розв’язки рівнянь, з якими
матимемо справу, дійсно задовольняють цій умові. З урахуванням
співвідношень \eqref{eq:potE} та \eqref{eq:LorenzCondition} маємо:

\begin{equation*}
    \Div\Efield = \Div\left( -\nabla\phi - \frac1c \parttime{\vect{A}}\right) = - \nabla^2\phi - \frac1c \parttime{}\Div\vect{A} =
    \frac1{c^2}\pparttime{\phi} - \nabla^2\phi.
\end{equation*}
Тоді з рівняння \eqref{eq:M1D} маємо:
\begin{equation}\label{eq:Dphi}
	\mdlgwhtsquare\  \phi = -4\pi \rho.
\end{equation}

Підставимо \eqref{eq:potB} та \eqref{eq:potE} в \eqref{eq:M4D}:
\begin{equation*}
    \Rot\Rot\vect{A} = \nabla \Div\vect{A} - \nabla^2\vect{A} = -\frac1c \nabla \parttime\phi - \nabla^2\vect{A}.
\end{equation*}
де враховано умову Лоренца \eqref{eq:LorenzCondition}. Оскільки члени з скалярним потенціалом
$\phi$ в останній формулі скорочуються, отримуємо:
\begin{equation}\label{eq:DA}
	\mdlgwhtsquare\  \vect{A} = -\frac{4\pi}{c} \vect{j}.
\end{equation}

Перевіримо, чи сумісні рівняння \eqref{eq:Dphi} та \eqref{eq:DA} з умовою Лоренца \eqref{eq:LorenzCondition}.
Комбінування рівнянь $\frac1c \parttime{} \eqref{eq:Dphi} + \Div\eqref{eq:DA}$ дає:
\begin{equation}
	\mdlgwhtsquare\ \left( \frac1c \parttime{\phi} + \Div\vect{A} \right) = \frac{4\pi}c \left( \parttime{\rho} + \Div\vect{j}\right)  .
\end{equation}
За умови Лоренца права частина дорівнює нулю, тобто закон збереження
заряду є необхідною умовою існування розв’язку. Навпаки, якщо цей закон
виконується, то:
\begin{equation*}
	\mdlgwhtsquare\ \left( \frac1c \parttime{\phi} + \Div\vect{A} \right) = 0.
\end{equation*}

Останнє співвідношення, якщо його розглядати як рівняння для:
\begin{equation}
	f = \Div \vect{A} + \frac{1}{c} \frac{\partial \phi}{\partial t},
\end{equation}
само по собі не гарантує $f \equiv 0$, оскільки розв`язок
рівняння:
\begin{equation}\label{eq:Df}
	\mdlgwhtsquare\ f = 0
\end{equation}
не є єдиним (воно має, наприклад, хвильові розв`язки). Але за умови
відсутності зовнішнього випромінювання, коли розглядається обмежена
система зарядів і струмів, рівняння \eqref{eq:Df} має тільки тривіальний розв'язок
$f=0$. Ця ситуація відповідає запізнюючим розв`язкам рівнянь\eqref{eq:Dphi} та \eqref{eq:DA}, що
розглядаються нижче.


%% --------------------------------------------------------
\subsection*{Калібрування Гамільтона}
%% --------------------------------------------------------

Розглянемо іншу калібрувальну умову Гамільтона:
\begin{equation}\label{eq:HamiltonCondition}
	\phi = 0.
\end{equation}

Це співвідношення також завжди можна задовольнити за допомогою
калібрувальних перетворень \eqref{eq:A'}, \eqref{eq:phi'}. Тоді за умови \eqref{eq:HamiltonCondition} рівняння
поля \eqref{eq:M1D} перепишеться, з урахуванням \eqref{eq:potE}, так:
\begin{equation}
	-\frac1c \parttime{} (\Div \vect{A}) = -4\pi \rho.
\end{equation}

Рівняння \eqref{eq:M4D} перепишеться, з урахуванням \eqref{eq:potB} та \eqref{eq:potE}, так:
\begin{equation}\label{eq:DA_for_Hamilton1}
	\mdlgwhtsquare\ \vect{A}  + \nabla \Div\vect{A} = -\frac{4\pi}{c} \vect{j}.
\end{equation}

Перевіримо, чи сумісні рівняння ці рівняння. Застосування
дивергенції до лівої частини останнього рівняння дає:
\begin{equation*}
	\Div(\mdlgwhtsquare\ \vect{A}  - \nabla \Div\vect{A}) = \frac1{c^2} \pparttime{} \Div\vect{A} - \nabla^2\Div\vect{A} + \nabla^2\Div\vect{A},
\end{equation*}
або
\begin{equation}\label{eq:DA_for_Hamilton2}
	-\frac1c \parttime{} \left[ \frac1c \parttime{} (\Div\vect{A})\right] + \frac{4\pi}c \Div\vect{j} = 0.
\end{equation}
Якщо врахувати \eqref{eq:DA_for_Hamilton1}, маємо:
\begin{equation*}
	\Div \vect{j} + \frac{\partial \rho}{\partial t} = 0,
\end{equation*}
тобто закон збереження заряду \eqref{eq:charge_conservation_low_diff} є необхідною умовою розв`язку
\eqref{eq:DA_for_Hamilton1}, \eqref{eq:DA_for_Hamilton2} за умови \eqref{eq:HamiltonCondition}.


%% --------------------------------------------------------
\subsection*{Калібрування Кулона}
%% --------------------------------------------------------

Кулонівське калібрування накладає умову:
\begin{equation}\label{eq:ColumbCondition}
	\Div \vect{A} = 0,
\end{equation}
також може бути виконана за допомогою підбору калібрувального
перетворення.

За умовою \eqref{eq:ColumbCondition} з \eqref{eq:M1D} маємо рівняння Пуассона для потенціалу~$\phi$:
\begin{equation}\label{eq:ColumbCondition_Puasson}
	\nabla^2 \phi = -4\pi \rho
\end{equation}
аналогічно електростатиці.

З іншого рівняння Максвелла \eqref{eq:M4D} дістаємо:
\begin{equation}\label{eq:ColumbCondition_DA}
	\mdlgwhtsquare\ \vect{A} = \frac{4\pi}c \vect{j} - \frac1c \nabla \parttime{\phi}.
\end{equation}

Для перевірки сумісності \eqref{eq:ColumbCondition_Puasson} та \eqref{eq:ColumbCondition_DA} обчислимо з останнього рівняння з
урахуванням \eqref{eq:ColumbCondition}:
\begin{equation*}
	\mdlgwhtsquare\Div\vect{A} = \frac{4\pi}c \Div\vect{j} - \frac1c \nabla^2 \parttime{\phi} = \frac{4\pi}c\left( \Div\vect{j} +
	\parttime{\phi}\right),
\end{equation*}
де підставлено $\nabla^2\phi$ з \eqref{eq:ColumbCondition_Puasson}. Знову рівняння неперервності --- закон збереження
заряду --- виступає як необхідна умова існування розв’язку при заданій
калібрувальній умові. Навпаки, якщо виконується рівняння неперервності, для
розв’язку \eqref{eq:ColumbCondition_DA} маємо $\mdlgwhtsquare\Div\vect{A} = 0$ і за відповідних граничних умов
дістанемо \eqref{eq:ColumbCondition}.

%% --------------------------------------------------------
\section{Потенціали ізольованої системи зарядів і струмів}
%% --------------------------------------------------------



В цьому розділі нас цікавитиме поле обмеженої системи, яка є ізольована. Це означає, що протягом усієї історії системи, починаючи з нескінченного
минулого, зовнішніх впливів немає; тобто немає джерел на нескінченності чи якогось зовнішнього випромінювання. На відміну від задачі Коші , коли поля
розглядають за $t >0$, а при $t =0$ задають початкові умови (разом з граничними умовами на нескінченності), у разі ізольованої системи будемо розглядати
поля за усіх часів, припускаючи, що функції $\rho(t, \vect{r})$ та $\vect{j}(t, \vect{r})$ задані на усій дійсній осі $t$. Це, зокрема, дозволяє
оперувати з перетворенням Фур’є цих функцій за часом. Для ізольованої системи буде отримано розв’язок рівнянь \eqref{eq:Dphi}, \eqref{eq:DA} у вигляді
запізнюючих потенціалів, який застосовується для розгляду різноманітних задач теорії випромінювання.


%% --------------------------------------------------------
\subsection*{Перетворення Фур’є та рівняння Гельмгольца}
%% --------------------------------------------------------

Позначимо:
\begin{equation}\label{eq:tilde_phi}
	\tilde{\phi}(\omega, \vec{r}) = \frac{1}{\sqrt{2\pi}} \int\limits_{-\infty}^{+\infty} dt \, e^{i \omega t} \phi(t, \vect{r}).
\end{equation}
\begin{equation}\label{eq:tilde_A}
	\tilde{\vect{A}}(\omega, \vect{r}) = \frac{1}{\sqrt{2\pi}} \int\limits_{-\infty}^{+\infty} dt \, e^{i \omega t} \vect{A}(t, \vect{r}).
\end{equation}
--- перетворення Фур’є для потенціалів.
Оскільки диференціювання за часом індукує множення фур’є-образів на \((-i \omega)\), аналог калібрувальної умови Лоренца \eqref{eq:LorenzCondition} має
вигляд:
\begin{equation}
	-\frac{i \omega}{c} \tilde{\phi} + \Div \tilde{\vect{A}} = 0.
\end{equation}

З закону збереження заряду \eqref{eq:charge_conservation_low_diff} маємо:
\begin{equation}
	-i \omega \tilde{\rho} + \Div \tilde{\vect{j}} = 0,
\end{equation}
де
\begin{equation}
	\tilde{\rho}(\omega, \vect{r}) = \frac{1}{\sqrt{2\pi}} \int\limits_{-\infty}^{+\infty} dt \, e^{i \omega t} \rho(t, \vect{r}),
\end{equation}
\begin{equation}
	\tilde{\vect{j}}(\omega, \vect{r}) = \frac{1}{\sqrt{2\pi}} \int\limits_{-\infty}^{+\infty} dt \, e^{i \omega t} \vect{j}(t, \vect{r}).
\end{equation}

Далі розглянемо рівняння для потенціалів саме за умови Лоренца \eqref{eq:LorenzCondition} або \eqref{eq:A'}. З рівнянь \eqref{eq:Dphi}, \eqref{eq:DA},
де друга похідна за часом індукує множення на \( -\omega^2 \) фур’є-образів, отримаємо \emph{рівняння Гельмгольца}:
\begin{equation}\label{eq:Dphi_Fourier}
	\Delta \tilde{\phi} + k^2 \tilde{\varphi} = -4\pi \tilde{\rho}, \quad \text{де} \quad k = \frac{\omega}{c}.
\end{equation}
\begin{equation}\label{eq:DA_Fourier}
	\Delta \tilde{\vect{A}} + k^2 \tilde{\vect{A}} = -\frac{4\pi}{c} \tilde{\vect{j}}.
\end{equation}



%% --------------------------------------------------------
\subsection*{Умова випромінювання для ізольованої системи}
%% --------------------------------------------------------


Систему зарядів і струмів називатимемо ізольованою, якщо вона зосереджена в обмеженій області за відсутності зовнішнього випромінювання, що йде з
нескінченності. Зосередимось на пошуку розв’язку рівняння \eqref{eq:Dphi} та його фур’є-образу \eqref{eq:Dphi_Fourier}для скалярного потенціалу $\phi$.
Зараз ми зацікавлені у знаходженні розв’язку, що описує поле ізольованої системи джерел. Розглянемо спочатку розв’язок, що відповідає сферично
симетричному точковому джерелу, та задовольняє рівнянню:
\begin{equation*}
	\mdlgwhtsquare\ \phi = \delta(\vect{r} - \vect{r}')\chi(t, \vect{r}'),
\end{equation*}
де $\chi(t, \vect{r}') = 4\pi\rho(t, \vect{r}_0)$. Очевидно, розв’язок \eqref{eq:Dphi} можна подати, як суперпозицію
таких розв’язків з різними $\vect{r}'$.

Нехай $\vect{r}' = 0$. Поле, що створюється точковим джерелом у точці $\vect{r}$, є сферично-симетричним $\phi = \phi(t, r)$, $r = |\vect{r}|$. Завдяки
сферичній симетрії:
\begin{equation*}
	\mdlgwhtsquare\ \phi = \frac1{c^2}\pparttime{\phi} - \frac1{r^2} \diff{}{r}\left(r^2 \diff{\phi}{r} \right) .
\end{equation*}
Покладемо $\phi = \frac{\psi}{r}$, тоді за $r>0$:
\begin{equation*}
	\mdlgwhtsquare\ \phi = \frac1r \left( \frac1{c^2}\pparttime{\psi} - \ddiff{\psi}{r}\right) =0 .
\end{equation*}
Це одновимірне хвильове рівняння, яке має загальний розв’язок:
\begin{equation*}
	\psi = f_1\left( t - \frac{r}c \right) + f_2\left( t + \frac{r}c \right),
\end{equation*}
де $f_1$ та $f_2$ --- довільні функції однієї змінної. Тут $f_1$ описує хвилі, що
випромінюються джерелом, а $f_2$ --- хвилі, що приходять з нескінченності. \emph{За
	відсутності зовнішнього випромінювання} слід покласти $f_2=0$. Звідси:
\begin{equation*}
	\phi = \frac1r f_1\left( t - \frac{r}c \right).
\end{equation*}

Якщо джерело знаходиться у точці $\vect{r}' \neq 0$, очевидно:
\begin{equation}\label{eq:rad_phi}
	\phi(t, \vect{r}) = \frac1{|\vect{r} - \vect{r}'|} f_1\left( t - \frac{|\vect{r} - \vect{r}'|}c \right).
\end{equation}

%% --------------------------------------------------------
\subsection*{Умова випромінювання і рівняння Гельмгольца}
%% --------------------------------------------------------


Для фур'є-образів розв’язку \eqref{eq:rad_phi} маємо:
\begin{equation}
	\tilde{\phi} (\omega, \vect{r}) = \frac{1}{\sqrt{2\pi}} \int dt \, e^{i\omega t} \frac{1}{|\vect{r} - \vect{r}_0|}
	\exp \left[ i \frac{\omega}{c} |\vect{r} - \vect{r}_0| \right] \tilde{f}_1(\omega),
\end{equation}
де
\begin{equation}
	\tilde{f}_1 (\omega) \equiv \frac{1}{\sqrt{2\pi}} \int d\xi \, e^{i\omega \xi} f_1(\xi).
\end{equation}

На великих відстанях
\begin{equation}
	\tilde{\phi} (\omega, \vect{r}) \sim \frac{e^{i\omega (r - \vect{n} \vect{r}_0)/c}}{r} + O(r^{-2}).
\end{equation}

Очевидно, для будь-якого сферично-симетричного розв’язку рівняння Гельмгольца \eqref{eq:Dphi_Fourier} зовні області, де права частина цього рівняння
відмінна від нуля, за
умови відсутності зовнішнього випромінювання
\begin{equation}\label{eq:rad_phi_sph}
	\tilde{\phi} (\omega, \vect{r}) \sim \frac{e^{ikr}}{r}.
\end{equation}

Розв’язок рівняння Гельмгольца, що задовольняє умовам випромінювання, у випадку обмеженої системи джерел має бути суперпозицією розв’язків типу
\eqref{eq:rad_phi_sph} з
різними $\vect{r}'$ і мати асимптотику
\begin{equation}
	\tilde{\phi} (\omega, \vect{r}) \approx C(\vect{n}) \frac{e^{ikr}}{r} + O(r^{-2}),
\end{equation}
де амплітуда $C(\vect{n}) $ залежить лише від кутів.


%% --------------------------------------------------------
\subsection*{Запізнюючі потенціали}
%% --------------------------------------------------------


Умови випромінювання однозначно задають поля ізольованої системи струмів і розв’язки хвильових рівнянь \eqref{eq:Dphi}, \eqref{eq:DA} для потенціалів, а
також рівнянь
Гельмгольца \eqref{eq:Dphi_Fourier}, \eqref{eq:DA_Fourier} для їх Фур’є-перетворень. Розв’язки \eqref{eq:Dphi}, \eqref{eq:DA} можна отримати
безпосередньо,
використовуючи сферично-симетричний
розв’язок \eqref{eq:rad_phi}. Але ми проведемо аналогічний розгляд з використанням сферично-симетричного розв’язку рівнянь Гельмгольца, а потім
перейдемо до
розв’язків \eqref{eq:Dphi}, \eqref{eq:DA} через перетворення Фур’є. Розглянемо рівняння для скалярного потенціалу \eqref{eq:Dphi_Fourier} і шукатимемо
фундаментальний
розв’язок
оператора в лівій частині \eqref{eq:Dphi_Fourier}:
\begin{equation}
	\nabla^2\tilde{G} + k^2\tilde{G} = \delta(\vect{r})
\end{equation}
Поле, створюване сферично-симетричним точковим джерелом, також є сферично-симетричним, тому можна покласти
\begin{equation}
	\tilde{G} (\vect{r}) = \frac{g(r)}{r}
\end{equation}

Тоді з \eqref{eq:rad_phi_sph} за $r>0$ маємо

\begin{equation}
	\frac{d^2 g}{dr^2} + k^2 g = 0 \quad \Rightarrow \quad g = C_1(k) e^{ikr} + C_2(k) e^{-ikr}.
\end{equation}

Враховуючи умову випромінювання, слід покласти $C_2(k)=0$:
\begin{equation}
	\tilde{G}(r) = C_1(k) \frac{e^{ikr}}{r}, \quad \text{за} \quad r>0.
\end{equation}

Залишається визначити $C_1(k)$. Коли $r \to 0$, поведінку розв’язку визначає співмножник $C_1/r$, а в лівій частині рівняння \eqref{eq:rad_phi_sph}
домінує доданок
$\Delta \varphi$. Тому асимптотика розв’язку за $r \to 0$ повинна збігатися з розв’язком рівняння Пуассона для точкового заряду:
\begin{equation}
	\Delta \left( \frac{q}{r} \right) = -4\pi q \delta (\vect{r}).
\end{equation}

Співставлення за $r > 0$ дає:
\begin{equation}\label{eq:tildeGr}
	\tilde{G}(r) = \frac{e^{ikr}}{4\pi r}.
\end{equation}

Більш послідовний розгляд фундаментальних розв’язків операторів Даламбера та Гельмгольца з точки зору узагальнених функцій див. Додаток 1.

З урахуванням \eqref{eq:tildeGr}, за принципом суперпозиції розв’язки рівнянь \eqref{eq:tilde_phi}, \eqref{eq:tilde_A} за умови випромінювання можна
подати згортками:
\begin{equation}\label{eq:tilde_phi_bundle}
	\tilde{\varphi} (\omega, \vect{r}) = \int \frac{dV'}{|\vect{r} - \vect{r}'|} \exp \left[ i \frac{\omega}{c} |\vect{r} - \vect{r}'| \right]
	\tilde{\rho} (\omega, \vect{r}'),
\end{equation}
\begin{equation}\label{eq:tilde_A_bundle}
	\tilde{\vect{A}}(\omega, \vect{r}) = \int \frac{dV'}{c|\vect{r} - \vect{r}'|} \exp \left[ i \frac{\omega}{c} |\vect{r} - \vect{r}'| \right]
	\tilde{\vect{j}} (\omega, \vect{r}').
\end{equation}

За допомогою оберненого до \eqref{eq:tilde_phi},\eqref{eq:tilde_A} перетворення Фур'є маємо:
\begin{equation}
	\varphi (t, \vect{r}) = \frac{1}{\sqrt{2\pi}} \int e^{-i\omega t} \tilde{\varphi} (\omega, \vect{r}) d\omega,
\end{equation}
\begin{equation}
	\rho (t, \vect{r}) = \frac{1}{\sqrt{2\pi}} \int e^{-i\omega t} \tilde{\rho} (\omega, \vect{r}) d\omega.
\end{equation}

Звідси та з \eqref{eq:tilde_phi_bundle}
\begin{equation}
	\phi (t, \vect{r}) = \int \frac{dV'}{|\vect{r} - \vect{r}'|} \int \frac{d\omega}{\sqrt{2\pi}} e^{-i\omega t} \tilde{\rho} (\omega, \vect{r}')
	= \int \frac{dV'}{|\vect{r} - \vect{r}'|} \rho (t_{\text{ret}}, \vect{r}'),
\end{equation}
де $t_{\text{ret}} = t - \frac{1}{c} |\vect{r} - \vect{r}'|$.

Остаточно запишемо:
\begin{equation}\label{eq:phi_ret}
	\varphi (t, \vect{r}) = \int \frac{dV'}{|\vect{r} - \vect{r}'|} \rho \left( t - \frac{1}{c} |\vect{r} - \vect{r}'|, \vect{r}' \right),
\end{equation}
а також, аналогічно, для розв’язку \eqref{eq:DA}:
\begin{equation}\label{eq:A_ret}
	\vect{A} (t, \vect{r}) = \frac{1}{c} \int \frac{dV'}{|\vect{r} - \vect{r}'|} \vect{j} \left( t - \frac{1}{c} |\vect{r} - \vect{r}'|,
	\vect{r}' \right).
\end{equation}

Формули \eqref{eq:phi_ret}, \eqref{eq:A_ret} подають \textit{запізнюючі розв’язки} рівнянь \eqref{eq:Dphi}, \eqref{eq:DA}, що задовольняють умовам
випромінювання в разі
обмеженої
ізольованої системи зарядів та струмів.

Зауважимо, що \eqref{eq:phi_ret}, \eqref{eq:A_ret} можна записати у вигляді згортки фундаментального розв’язку оператора Даламбера з правими частинами
рівнянь
\eqref{eq:Dphi},
\eqref{eq:DA} (див. Додаток 1). Цей фундаментальний розв’язок має вигляд:
\begin{equation}\label{eq:D}
	D(t, \vect{r}) = \frac{1}{2\pi c} \delta \left( t^2 - r^2 / c^2 \right) \theta(t).
\end{equation}

Згортка з правою частиною \eqref{eq:Dphi}:
\begin{equation}\label{eq:2.2.18}
	\phi (t, \vect{r}) = 4\pi \int dt' dV' D(t - t', \vect{r} - \vect{r}') \rho (t', \vect{r}').
\end{equation}

Підставимо \eqref{eq:D}:
\begin{multline*}
	\varphi (t, \vect{r}) = \frac{2}{c} \int dt' dV' \delta \left[ \left( t - t' \right)^2 - \frac{(r - r')^2}{c^2} \right] \theta (t - t') \rho (t',
	\vect{r}') = \\
	= \frac{1}{c} \int dV' \int \delta \left[ t - t' - \frac{|\vect{r} - \vect{r}'|}{c} \right] \rho (t', \vect{r}') dt',
\end{multline*}
що збігається з \eqref{eq:phi_ret} після інтегрування по $t'$.

Аналогічно:
\begin{equation}\label{eq:2.2.19}
	\vect{A'} (t, \vect{r}) = \frac{4\pi}{c} \int dt' dV' G(t - t', \vect{r} - \vect{r}') \vect{j} (t', \vect{r}').
\end{equation}

Перевіримо виконання калібрувальної умови Лоренца для розв’язків \eqref{eq:2.2.18}, \eqref{eq:2.2.19}:

\begin{multline*}
\frac{1}{c} \frac{\partial \varphi}{\partial t} + \Div\vect{A} = \\
= \frac{4\pi}{c} \int dt' dV' \left[ \left( \frac{\partial}{\partial t'} G(t - t', \vect{r} - \vect{r}') \right) \rho (t, \vect{r}) + \left(
\frac{\partial}{\partial x_i} G(t - t', \vect{r} - \vect{r}') \right) J_i (t', \vect{r}') \right] = \\
= \frac{4\pi}{c} \int dt' dV' \left[ - \left( \frac{\partial}{\partial t'} G(t - t', \vect{r} - \vect{r}') \right) \rho (t', \vect{r}') - \left(
\frac{\partial}{\partial x_i} G(t - t', \vect{r} - \vect{r}') \right) J_i (t', \vect{r}') \right] = \\
= \frac{4\pi}{c} \int dt' dV' \left[ G(t - t', \vect{r} - \vect{r}') \frac{\partial}{\partial t'} \rho (t', \vect{r}') + G(t - t', \vect{r} -
\vect{r}') \frac{\partial}{\partial x_i} J_i (t', \vect{r}') \right] = \\
= \frac{4\pi}{c} \int dt' dV' G(t - t', \vect{r} - \vect{r}') \left[ \frac{\partial}{\partial t'} \rho (t', \vect{r}') + \Div\vect{j}
(t', \vect{r}') \right] = 0
\end{multline*}
де \( \vect{r} = \{ x_i \} \), \( \vect{r}' = \{ x'_i \} \), в останньому перетворенні проведено інтегрування частинами по \( t' \) та по \( x'_i \) з
урахуванням обмеженості області, де густини зарядів та струмів відмінні від нуля.

Таким чином, виконання умови Лоренца для \eqref{eq:2.2.18}, \eqref{eq:2.2.19} забезпечено законом збереження заряду.

%% --------------------------------------------------------
\section{Задача Коші для рівнянь Максвелла}\label{sec:Cauchi}
%% --------------------------------------------------------

Як зазначено на початку попереднього розділу, задача Коші, на відміну від розгляду ізольованої системи, оперує лише з полями \( t \geq t_0 \). Поведінка
джерел та полів за \( t < t_0 \), яка може впливати на стан системи за \( t \geq t_0 \), є невідомою. Тому, на відміну від п. 2.2, окрім задання функцій
\( \rho(t, \vect{r}) \), \( \vect{j}(t, \vect{r}) \) за \( t \geq t_0 \) необхідно задавати певні початкові умови.

Далі для простоти виберемо відлік часу так, що початковий момент \( t_0 = 0 \). Поля \( \Efield \), \( \Bfield \) визначають стан електромагнітного
поля, якщо вони задані в усьому просторі. З’ясуємо, чи дозволяють рівняння Максвелла однозначно передбачати стан поля при \( t > 0 \), якщо цей стан
відомий за \( t = 0 \). Розподіли густини заряду \( \rho(t, \vect{r}) \) та густини струму \( \vect{j}(t, \vect{r}) \) за \( t \geq 0 \) вважаємо
заданими, причому при \( t = 0 \) задаємо поля \( \Efield \), \( \Bfield \), причому вони мають задовольняти рівняння:

\begin{equation}
\Div \Efield = 4\pi\rho \label{eq:2.3.1}
\end{equation}

\begin{equation}
\Div \Bfield = 0 \label{eq:2.3.2}
\end{equation}

Еволюцію полів \( \Efield \) та \( \Bfield \) визначає інша пара рівнянь Максвелла, що містить похідні за часом:

\begin{equation}
\frac{\partial \Efield}{\partial t} = c \Rot \Bfield - 4\pi \cdot \vect{j} \label{eq:2.3.3}
\end{equation}

\begin{equation}
\frac{\partial \Bfield}{\partial t} = -c \Rot \Efield \label{eq:2.3.4}
\end{equation}

З будови системи рівнянь видно, що ми можемо задати електромагнітне поле при \( t = 0 \), яке задовольняє рівнянням (\ref{eq:2.3.1}) та
(\ref{eq:2.3.2}), і розв’язувати потім рівняння (\ref{eq:2.3.3}) та (\ref{eq:2.3.4}) при \( t > 0 \). Однак виникає питання, чи зберігаються при цьому
рівняння (\ref{eq:2.3.1}) та (\ref{eq:2.3.2}) також і при \( t > 0 \)?

Покажемо, що це дійсно так. З рівняння (\ref{eq:2.3.3}) маємо:

\[
\frac{\partial}{\partial t} \Div \Efield = -4\pi \cdot \Div \vect{j} = 4\pi \frac{\partial \rho}{\partial t},
\]

де враховано закон збереження заряду, або:

\[
\frac{\partial}{\partial t} \left\{ \Div \Efield - 4\pi\rho \right\} = 0.
\]

Звідси видно, що співвідношення (\ref{eq:2.3.1}) справедливо при \( t > 0 \), якщо воно має місце при \( t = 0 \).

З рівняння (\ref{eq:2.3.4}) маємо:

\[
\frac{\partial}{\partial t} \Div \Bfield = 0,
\]

звідки видно, що рівняння (\ref{eq:2.3.2}) також зберігається при \( t > 0 \).

\subsection*{Єдиність розв’язків}

Покажемо, що початкові умови, задані при \( t = 0 \) всередині кулі

\begin{equation}
(x - x_0)^2 + (y - y_0)^2 + (z - z_0)^2 \leq c^2 T^2 \label{eq:2.3.5}
\end{equation}
однозначно визначають напруженості поля у точці \( \vect{r}_0 = (x_0, y_0, z_0) \) в момент \( t = T \). Припустимо, що існують два розв’язки рівнянь
Максвелла \( \{\Efield(t, \vect{r}), \Bfield(t, \vect{r})\} \) та \( \{\Efield'(t, \vect{r}), \Bfield'(t, \vect{r})\} \) з однаковими початковими
умовами в області (\ref{eq:2.3.5}) та розглянемо їхню різницю \( \Delta \Efield = \Efield - \Efield' \), \( \Delta \Bfield = \Bfield - \Bfield'
\), яка, очевидно, задовольняє рівняння:

\begin{equation*}
\frac{\partial \Delta \Efield}{\partial t} = c \cdot \Rot(\Delta \Bfield), \quad \frac{\partial \Delta \Bfield}{\partial t} = -c \cdot
\Rot(\Delta \Efield).
\end{equation*}

Обчислюючи похідну з урахуванням цих рівнянь:

\begin{multline}\label{eq:2.3.6},
\frac{1}{8\pi} \frac{\partial}{\partial t} (\Delta \Efield^2 + \Delta \Bfield^2)
= \\ = \frac{c}{4\pi}\left( \Delta\Efield\Rot(\Delta\Bfield - \Delta\Bfield\Rot(\Delta\Efield )\right)
-\frac{c}{4\pi} \Div [\Delta \Efield \times \Delta
\Bfield]
\end{multline}
маємо співвідношення, аналогічне закону збереження енергії в диференціальній формі:

\begin{equation*}
\frac{\partial W}{\partial t} = -\frac{c}{4\pi} \Div [\Delta \Efield \times \Delta \Bfield],
\end{equation*}
де \( W = \frac{1}{8\pi} (\Delta \Efield^2 + \Delta \Bfield^2) \).

Розглянемо невід’ємну величину:

\begin{equation}\label{eq:2.3.7}
    U(t) = \int\limits_{K(t)} W \, dV,
\end{equation}
де областю інтегрування \( K(t) \) є куля з центром у точці \( \{\vect{r}_0\} = \{x_0, y_0, z_0\} \):

\begin{equation*}
    K(t) = \{\vect{r} : |\vect{r} - \vect{r}_0| \leq R(t)\}, \quad \text{де} \quad R(t) = c(T - t), \quad t \leq T.
\end{equation*}

Перейдемо в інтегралі (\ref{eq:2.3.7}) до сферичних координат з центром у \( \vect{r}_0 \):

\begin{equation*}
    U(t) = \int d\Omega \int_0^{R(t)}  \, r^2 W(t, r, \theta, \varphi) dr,
\end{equation*}
де інтегрування по кутовій частині \( d\Omega = \sin \theta \, d\theta \, d\varphi \) виконується по усій одиничній сфері. Враховуючи \eqref{eq:2.3.6},
обчислимо:

\begin{multline}\label{eq:2.3.8}
\frac{dU}{dt} = -c R^2(t) \int d\Omega \, W(t, R(t), \theta, \varphi) + \int_{K(t)} \frac{\partial W}{\partial t} \, dV = \\
= -c R^2 \int d\Omega \, W - \frac{c}{4\pi} \int_{K(t)} \Div [\Delta \Efield \times \Delta \Bfield] \, dV = \\
\text{(за формулою Остроградського-Гаусса)}\\
= -c R^2 \int d\Omega \, W - \frac{c}{4\pi} \int_{\partial K(t)} (d\vect{S} \cdot [\Delta \Efield \times \Delta \Bfield]) = \\
= -c R^2 \int d\Omega \, W - \frac{c}{4\pi} \int d\Omega \, R^2(t) [\Delta \Efield \times \Delta \Bfield] \cdot \vect{n},
\end{multline}
де \( \partial K(t) \) --- поверхня сфери \( r = R(t) \), \( \vect{n} = \vect{r} / r = \{\sin \theta \cos \varphi, \sin \theta \sin \varphi, \cos
\theta\} \) --- одиничний вектор нормалі до цієї поверхні; \( d\vect{S} = \vect{n} R^2 d\Omega \).

Оскільки:

\begin{equation*}
| \left[ \Delta \Efield \times \Delta \Bfield \right]  \cdot \vect{n} | \leq | \Delta \Efield | \cdot | \Delta \Bfield | \leq \frac{\Delta \Efield^2 +
\Delta \Bfield^2}{2} \leq 4\pi W,
\end{equation*}
з \eqref{eq:2.3.8} випливає:

\begin{equation}\label{eq:2.3.9}
\frac{dU}{dt} = -c R^2 \int d\Omega \, W + \frac{c}{4\pi} \int d\Omega \, [\Delta \Efield \times \Delta \Bfield \cdot \vect{n}] \leq 0.
\end{equation}

Якщо початковий момент \( t = 0 \) поля збігаються:

\begin{equation*}
\Delta \Efield(0, \vect{r}) = \Delta \Bfield(0, \vect{r}) \equiv 0,
\end{equation*}

тому \( U(0) = 0 \). За означенням \( U(t) \geq 0 \), тому нерівність (\ref{eq:2.3.9}) означає, що \( U(t) \equiv 0 \), тобто \( \Efield = \Efield'
\), \( \Bfield = \Bfield' \) у будь-якій кулі \( K(t) \), \( t \in [0, T] \).

Таким чином, розв’язки з однаковими початковими умовами співпадають, тобто ці умови, задані в області (\ref{eq:2.3.5}), однозначно задають поле у
будь-якій кулі \( K(t) \), \( 0 \leq t \leq T \).

Як видно з наведених міркувань, зміни початкових умов за \( t = 0 \) поза областю (\ref{eq:2.3.5}) не впливають на поле у точці \( \vect{r}_0 \). Це
очевидний наслідок скінченої швидкості поширення взаємодій. Аналогічно, збурення поля, що відбуваються в момент \( t \) поза кулею \( K(t) \), не
встигають поширитися до точки \( \vect{r}_0 \) за час \( R(t)/c = T - t \). Чим ближче \( t \) до \( T \), тим меншою є область впливу на цю точку.

Природно, що для визначення поля при \( t = T \) в усьому просторі треба задавати початкові умови також в усьому просторі. Поля, що задають у початковий
момент, мають задовольнити рівнянням (\ref{eq:2.3.1}) та (\ref{eq:2.3.2}); тому стан поля та його майбутня еволюція визначаються \emph{чотирма функціями
від трьох просторових змінних}. Наприклад, можна незалежно задати дві компоненти електричного поля та дві --- магнітного.

\section*{Задачі}

%=========================================================
\begin{problem}%
Вектор-потенціал стаціонарного магнітного поля обмеженої системи струмів на великих відстанях можна наближено подати як \( \vect{A}(\vect{r}) =
c^{-1} r^{-3} \int dV \, \vect{j}(\vect{r}')(\vect{r} \cdot \vect{r}') \). Записати цей вираз через магнітний момент \( \vect{m} = (2c)^{-1} \int dV \,
\vect{r} \times \vect{j} \). Відповідь: \( \vect{A}(\vect{r}) = \vect{m} \times \vect{r} / r^3 \).
\end{problem}


%=========================================================
\begin{problem}%
Знайти загальний сферично симетричний розв’язок \( \delta \equiv \delta(k, r) \) рівняння Гельмгольца в вакуумі:
\[
\Delta \delta + k^2 \delta = 0 \quad \text{в області} \quad r \geq R > 0.
\]
Що в цьому розв’язку залишиться після врахування умови випромінювання (тобто за відсутності зовнішніх хвиль, що приходять до центру)?
\end{problem}

%=========================================================
\begin{problem}%
Знайти розв’язок \( \delta \equiv \delta(k, r) \) рівняння Гельмгольца
\begin{equation*}
   \Delta \delta + k^2 \delta = -4\pi\rho
\end{equation*}
у випадку сферичної симетрії \(
\rho \equiv \rho(k, r) \), за умови відсутності зовнішнього випромінювання. Функція \( \rho \equiv \rho(k, r) \) відмінна від нуля лише в області \( r <
R \). Розв’язок подати у вигляді квадратур. Показати, що при \( R \to 0 \) за умови фіксованого \( \int dV \, \rho(k, r) = 4\pi \int_0^R dr \, r^2
\rho(k, r) = q \), матимемо \( \delta(k, r) = q \exp(ikr) / r \).
\end{problem}

