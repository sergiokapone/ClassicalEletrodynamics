% !TeX program = lualatex
% !TeX encoding = utf8
% !TeX spellcheck = uk_UA
% !TeX root =../ClassicalEletrodynamics.tex

%=========================================================
\Opensolutionfile{answer}[\currfilebase/\currfilebase-Answers]
%\Writetofile{answer}{\protect\section*{\nameref*{\currfilebase}}}
\chapter{Потенціали електромагнітного поля}\label{\currfilebase}
%=========================================================


Існує досить багато методів аналізу рівнянь Максвелла \eqref{eq:M1D} -- \eqref{eq:M4D}, серед яких одним з найбільш поширених є введення потенціалів, що
дозволяють дещо
зменшити число невідомих функцій.

З умови соленоїдальності магнітного поля \eqref{eq:M2D} випливає, що існує деяке векторне поле $\vect{A}$, таке, що:
\begin{equation}\label{eq:potB}
	 \Bfield = \Rot\vect{A}
\end{equation}
Підставляючи це в \eqref{eq:M3D}, після елементарних перетворень маємо:
\begin{equation*}
	\Rot \left( \Efield + \frac{1}{c} \frac{\partial \vect{A}}{\partial t}\right)  = 0,
\end{equation*}
звідки випливає існування скалярного поля $\phi$, такого, що
\begin{equation}\label{eq:potE}
	\Efield = -\nabla \phi - \frac{1}{c} \frac{\partial \vect{A}}{\partial t}
\end{equation}

Таким чином, якщо шукати електромагнітне поле у вигляді \eqref{eq:potB}, \eqref{eq:potE}, рівняння \eqref{eq:M2D}, \eqref{eq:M3D} виконуються
автоматично.

%% --------------------------------------------------------
\subsection*{Калібрувальна інваріантність}
%% --------------------------------------------------------

Формули \eqref{eq:potB}, \eqref{eq:potE} не визначають потенціали однозначно. Розглянемо перетворення $(\phi, \vect{A}) \to (\phi', \vect{A}')$:
\begin{equation}\label{eq:A'}
	\vect{A}' = \vect{A} + \nabla \chi
\end{equation}
\begin{equation}\label{eq:phi'}
	\phi' = \phi - \frac{1}{c} \frac{\partial \chi}{\partial t}
\end{equation}
Підставляючи в \eqref{eq:potB}, \eqref{eq:potE}, бачимо, що напруженості електромагнітного
поля виражаються через $\phi'$, $\vect{A}'$ подібно до $\phi$, $\vect{A}$:

\begin{equation}
	\Bfield = \Rot \vect{A}'
\end{equation}
\begin{equation}
	\Efield = -\nabla \phi' - \frac{1}{c} \frac{\partial \vect{A}'}{\partial t}
\end{equation}
Таким чином, потенціали $(\phi, \vect{A})$ містять деякі степені свободи, що ніяк не
впливають на фізичну ситуацію. Цю обставину називають \emph{калібрувальною
інваріантністю} рівнянь поля, а перетворення \eqref{eq:A'}, \eqref{eq:phi'}, або інші, що не
впливають на спостережувані величини $(\Efield, \Bfield)$, називають \emph{калібрувальними
перетвореннями}.


%% --------------------------------------------------------
\subsection*{Калібрувальна умова Лоренца}
%% --------------------------------------------------------

Калібрувальна інваріантність дозволяє накладати додаткові умови на
потенціали $(\phi, \vect{A})$, за допомогою яких можна привести рівняння до більш
зручного вигляду.

Розглянемо умову Лоренца:
\begin{equation}\label{eq:LorenzCondition}
	\Div \vect{A} + \frac{1}{c} \frac{\partial \phi}{\partial t} = 0
\end{equation}
Цю умову можна задовольнити за допомогою перетворень \eqref{eq:A'}, \eqref{eq:phi'}.
Дійсно, припустимо, що:
\begin{equation*}
	\Div \vect{A} + \frac{1}{c} \frac{\partial \phi}{\partial t} = f \neq 0
\end{equation*}
та перейдемо до нових потенціалів $(\phi', \vect{A}')$ за формулами \eqref{eq:A'}, \eqref{eq:phi'}. Тоді:
\begin{equation*}
    \Div \vect{A}' + \frac{1}{c} \frac{\partial \phi'}{\partial t} =
	f - \frac{1}{c^2} \pparttime{\chi} + \nabla^2\chi
\end{equation*}
Підбираючи функцію $\chi$ таким чином, щоб $f = \mdlgwhtsquare\  \chi$
(де $\mdlgwhtsquare\  = \nabla^2 - \frac{1}{c^2} \frac{\partial^2}{\partial t^2}$ – оператор Даламбера), бачимо, що нові потенціали $(\phi',
	\vect{A}')$
задовольняють калібрувальній умові Лоренца \eqref{eq:LorenzCondition}.

Отримаємо рівняння для $(\phi, \vect{A})$, припускаючи тепер, що умова \eqref{eq:LorenzCondition}
виконана. З подальшого буде видно, що розв’язки рівнянь, з якими
матимемо справу, дійсно задовольняють цій умові. З урахуванням
співвідношень \eqref{eq:potE} та \eqref{eq:LorenzCondition} маємо:

\begin{equation*}
    \Div\Efield = \Div\left( -\nabla\phi - \frac1c \parttime{\vect{A}}\right) = - \nabla^2\phi - \frac1c \parttime{}\Div\vect{A} =
    \frac1{c^2}\pparttime{\phi} - \nabla^2\phi.
\end{equation*}
Тоді з рівняння \eqref{eq:M1D} маємо:
\begin{equation}\label{eq:Dphi}
	\mdlgwhtsquare\  \phi = -4\pi \rho
\end{equation}

Підставимо \eqref{eq:potB} та \eqref{eq:potE} в \eqref{eq:M4D}:
\begin{equation*}
    \Rot\Rot\vect{A} = \nabla \Div\vect{A} - \nabla^2\vect{A} = -\frac1c \nabla \parttime\phi - \nabla^2\vect{A}.
\end{equation*}
де враховано умову Лоренца \eqref{eq:LorenzCondition}. Оскільки члени з скалярним потенціалом
$\phi$ в останній формулі скорочуються, отримуємо:
\begin{equation}\label{eq:DA}
	\mdlgwhtsquare\  \vect{A} = -\frac{4\pi}{c} \vect{j}
\end{equation}

Перевіримо, чи сумісні рівняння \eqref{eq:Dphi} та \eqref{eq:DA} з умовою Лоренца \eqref{eq:LorenzCondition}.
Комбінування рівнянь $\frac1c \parttime{} \eqref{eq:Dphi} + \Div\eqref{eq:DA}$ дає:
\begin{equation}
	\mdlgwhtsquare\ \left( \frac1c \parttime{\phi} + \Div\vect{A} \right) = \frac{4\pi}c \left( \parttime{\rho} + \Div\vect{j}\right)  .
\end{equation}
За умови Лоренца права частина дорівнює нулю, тобто закон збереження
заряду є необхідною умовою існування розв’язку. Навпаки, якщо цей закон
виконується, то:
\begin{equation*}
	\mdlgwhtsquare\ \left( \frac1c \parttime{\phi} + \Div\vect{A} \right) = 0
\end{equation*}
Останнє співвідношення, якщо його розглядати як рівняння для:
\begin{equation}
	f = \Div \vect{A} + \frac{1}{c} \frac{\partial \phi}{\partial t}
\end{equation}
само по собі не гарантує $f \equiv 0$, оскільки розв`язок
рівняння:
\begin{equation}\label{eq:Df}
	\mdlgwhtsquare\ f = 0
\end{equation}
не є єдиним (воно має, наприклад, хвильові розв`язки). Але за умови
відсутності зовнішнього випромінювання, коли розглядається обмежена
система зарядів і струмів, рівняння \eqref{eq:Df} має тільки тривіальний розв'язок
$f=0$. Ця ситуація відповідає запізнюючим розв`язкам рівнянь\eqref{eq:Dphi} та \eqref{eq:DA}, що
розглядаються нижче.


%% --------------------------------------------------------
\subsection*{Калібрування Гамільтона}
%% --------------------------------------------------------

Розглянемо іншу калібрувальну умову Гамільтона:
\begin{equation}\label{eq:HamiltonCondition}
	\phi = 0
\end{equation}
Це співвідношення також завжди можна задовольнити за допомогою
калібрувальних перетворень \eqref{eq:A'}, \eqref{eq:phi'}. Тоді за умови \eqref{eq:HamiltonCondition} рівняння
поля \eqref{eq:M1D} перепишеться, з урахуванням \eqref{eq:potE}, так:
\begin{equation}
	-\frac1c \parttime{} (\Div \vect{A}) = -4\pi \rho
\end{equation}
Рівняння \eqref{eq:M4D} перепишеться, з урахуванням \eqref{eq:potB} та \eqref{eq:potE}, так:
\begin{equation}\label{eq:DA_for_Hamilton1}
	\mdlgwhtsquare\ \vect{A}  + \nabla \Div\vect{A} = -\frac{4\pi}{c} \vect{j}
\end{equation}

Перевіримо, чи сумісні рівняння ці рівняння. Застосування
дивергенції до лівої частини останнього рівняння дає:
\begin{equation*}
	\Div(\mdlgwhtsquare\ \vect{A}  - \nabla \Div\vect{A}) = \frac1{c^2} \pparttime{} \Div\vect{A} - \nabla^2\Div\vect{A} + \nabla^2\Div\vect{A},
\end{equation*}
або
\begin{equation}\label{eq:DA_for_Hamilton2}
	-\frac1c \parttime{} \left[ \frac1c \parttime{} (\Div\vect{A})\right] + \frac{4\pi}c \Div\vect{j} = 0.
\end{equation}
Якщо врахувати \eqref{eq:DA_for_Hamilton1}, маємо:
\begin{equation*}
	\Div \vect{J} + \frac{\partial \rho}{\partial t} = 0
\end{equation*}
тобто закон збереження заряду \eqref{eq:charge_conservation_low_diff} є необхідною умовою розв`язку
\eqref{eq:DA_for_Hamilton1}, \eqref{eq:DA_for_Hamilton2} за умови \eqref{eq:HamiltonCondition}.


%% --------------------------------------------------------
\subsection*{Калібрування Кулона}
%% --------------------------------------------------------

Кулонівське калібрування накладає умову:
\begin{equation}\label{eq:ColumbCondition}
	\Div \vect{A} = 0
\end{equation}
також може бути виконана за допомогою підбору калібрувального
перетворення.

За умовою \eqref{eq:ColumbCondition} з \eqref{eq:M1D} маємо рівняння Пуассона для потенціалу~$\phi$:
\begin{equation}\label{eq:ColumbCondition_Puasson}
	\nabla^2 \phi = -4\pi \rho
\end{equation}
аналогічно електростатиці.

З іншого рівняння Максвелла \eqref{eq:M4D} дістаємо:
\begin{equation}\label{eq:ColumbCondition_DA}
	\mdlgwhtsquare\ \vect{A} = \frac{4\pi}c \vect{j} - \frac1c \nabla \parttime{\phi}.
\end{equation}

Для перевірки сумісності \eqref{eq:ColumbCondition_Puasson} та \eqref{eq:ColumbCondition_DA} обчислимо з останнього рівняння з
урахуванням \eqref{eq:ColumbCondition}:
\begin{equation*}
	\mdlgwhtsquare\Div\vect{A} = \frac{4\pi}c \Div\vect{j} - \frac1c \nabla^2 \parttime{\phi} = \frac{4\pi}c\left( \Div\vect{j} +
	\parttime{\phi}\right),
\end{equation*}
де підставлено $\nabla^2\phi$ з \eqref{eq:ColumbCondition_Puasson}. Знову рівняння неперервності --- закон збереження
заряду --- виступає як необхідна умова існування розв’язку при заданій
калібрувальній умові. Навпаки, якщо виконується рівняння неперервності, для
розв’язку \eqref{eq:ColumbCondition_DA} маємо $\mdlgwhtsquare\Div\vect{A} = 0$ і за відповідних граничних умов
дістанемо \eqref{eq:ColumbCondition}.

\Closesolutionfile{answer}

