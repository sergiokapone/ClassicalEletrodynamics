% !TeX program = lualatex
% !TeX encoding = utf8
% !TeX spellcheck = uk_UA
% !BIB program = biber

\documentclass[%
biblatex,
%marginversioninfo
]{ConspectBook}


\begin{document}

\section*{§ 6. Функция Грина для волнового уравнения}

Волновые уравнения (6.37), (6.38) и (6.52) имеют одинаковую структуру
\begin{equation}
\label{eq:wave_equation}
\nabla^2 \psi - \frac{1}{c^2} \frac{\partial^2 \psi}{\partial t^2} = -4\pi f(\vect{r}, t),
\end{equation}
где \( f(\vect{r}, t) \) задает распределение источников, а \( c \) представляет собой скорость распространения волн в пространстве.

Для решения уравнения \eqref{eq:wave_equation}, так же как в электростатике, полезно найти сначала функцию Грина. Поскольку теперь поля зависят и от
времени, функция Грина будет зависеть от переменных \((\vect{r}, \vect{r}', t, t')\) и должна удовлетворять уравнению
\begin{equation}
\label{eq:green_function}
\left( \nabla_{\vect{r}}^2 - \frac{1}{c^2} \frac{\partial^2}{\partial t^2} \right) G(\vect{r}, t; \vect{r}', t') = -4\pi \delta (\vect{r} -
\vect{r}') \delta (t - t').
\end{equation}

Решение уравнения \eqref{eq:wave_equation} в неограниченном пространстве без граничных поверхностей выражается через \( G \) интегралом
\begin{equation}
\label{eq:solution}
\psi(\vect{r}, t) = \int G(\vect{r}, t; \vect{r}', t') f(\vect{r}', t') \, d^3 \vect{r}' \, dt'.
\end{equation}

Нужно, конечно, потребовать, чтобы функция Грина удовлетворяла определенным граничным условиям, которые задаются физическими требованиями.

Основная функция Грина, удовлетворяющая уравнению \eqref{eq:green_function}, зависит только от разностей координат \((\vect{r} - \vect{r}')\)
и времен \((t - t')\). Для нахождения \( G \) представим обе части уравнения \eqref{eq:green_function} в виде интегралов Фурье. Дельта-функцию в правой
части можно представить следующим образом:
\begin{equation}
\label{eq:delta_fourier}
\delta (\vect{r} - \vect{r}') \delta (t - t') = \frac{1}{(2\pi)^4} \int d^3 \vect{k} \int d\omega \, e^{i \vect{k} \cdot (\vect{r} -
\vect{r}')} e^{-i \omega (t - t')}.
\end{equation}

Соответственно запишем функцию \( G \) в виде
\begin{equation}
\label{eq:green_fourier}
G(\vect{r}, t; \vect{r}', t') = \int d^3 \vect{k} \int d\omega \, g(\vect{k}, \omega) e^{i \vect{k} \cdot (\vect{r} - \vect{r}')} e^{-i
\omega (t - t')}.
\end{equation}

\end{document}
