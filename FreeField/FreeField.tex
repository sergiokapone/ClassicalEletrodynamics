% !TeX program = lualatex
% !TeX encoding = utf8
% !TeX spellcheck = uk_UA
% !TeX root =../ClassicalEletrodynamics.tex

%=========================================================
\chapter{Вільне електромагнітне поле}\label{\currfilebase}
%=========================================================


В цьому розділі розглянуто питання, що стосуються вільних електромагнітних полів у вакуумі, тобто за відсутності зарядів та струмів.


%% --------------------------------------------------------
\section{Спектральний розклад і плоскі хвилі}\label{sec:spectral}
%% --------------------------------------------------------


%% --------------------------------------------------------
\subsection*{Зведення до додатних частот}
%% --------------------------------------------------------

Нехай $\phi(t)$ --- будь-яка функція, пов’язана з електромагнітним полем (наприклад, компонента електричного поля), для якої існує образ Фур’є:

\begin{equation}
\tilde{\phi}(\omega) = \frac{1}{\sqrt{2\pi}} \int\limits_{-\infty}^{\infty} dt \, e^{i\omega t} \phi(t)
\label{eq:fourier_transform}
\end{equation}

Обернене перетворення дозволяє обчислити $\phi(t)$ по функції $\tilde{\phi}(\omega)$:

\begin{equation}
\phi(t) = \frac{1}{\sqrt{2\pi}} \int\limits_{-\infty}^{\infty} d\omega \, e^{-i\omega t} \tilde{\phi}(\omega)
\label{eq:inverse_fourier_transform}
\end{equation}

Для функцій $f(t)$, $g(t)$ та їх образів Фур’є $\tilde{f}(\omega)$, $\tilde{g}(\omega)$ має місце рівність Пареєваля:

\begin{equation}
\int\limits_{-\infty}^{\infty} f(t)g^{*}(t)dt = \int\limits_{-\infty}^{\infty} \tilde{f}(\omega)\tilde{g}^{*}(\omega)d\omega
\label{eq:parseval}
\end{equation}

Практично будь-який реальний сигнал можна подати у вигляді суперпозиції монохроматичних сигналів. За допомогою технічних пристроїв можна виділяти,
підсилювати чи послабляти певні ділянки спектру сигналу, що відповідають певній області частот. При цьому формальний запис гармонічних сигналів,
пропорційних $\sim e^{i\omega t}$ та $\sim e^{-i\omega t}$, відповідає однаковій частоті $\omega > 0$. Тому перепишемо усі співвідношення через додатні
значення $\omega$, враховуючи, що фізичні напруженості полів описуються дійсними функціями.

Якщо $\phi(t)$ --- дійсна функція, то:

\begin{equation}\label{eq:conjugate_frequency}
\tilde{\phi}^{*}(\omega) = \frac{1}{\sqrt{2\pi}} \int\limits_{-\infty}^{\infty} dt \, e^{-i\omega t} \phi(t) = \tilde{\phi}(-\omega)
\end{equation}

Завдяки цьому формулу \eqref{eq:inverse_fourier_transform} можна переписати так:

\begin{multline*}
\phi(t) = \frac{1}{\sqrt{2\pi}} \left\{ \int\limits_{0}^{\infty} d\omega \, e^{-i\omega t} \tilde{\phi}(\omega) + \int\limits_{-\infty}^{0} d\omega \,
e^{-i\omega t}
\tilde{\phi}(\omega) \right\} = \\
= \frac{1}{\sqrt{2\pi}} \int\limits_{0}^{\infty} d\omega \left( \hat{\varphi}(\omega) e^{-i\omega t} + \hat{\phi}^{*}(\omega) e^{i\omega t} \right)
\end{multline*}

В цю формулу входять значення \( \tilde{\phi}'(\omega) \) лише з додатними частотами, які можна задавати незалежно на відміну від \( \phi(-\omega) \)
при \(
\omega > 0 \), які пов’язані з \( \phi(\omega) \) формулою \eqref{eq:conjugate_frequency}. Формулу зведення до додатних частот можна переписати так:

\begin{equation}
\phi(t) = \sqrt{\frac{2}{\pi}} \Re \int\limits_{0}^{\infty} d\omega e^{-i\omega t} \tilde{\varphi}(\omega)
\label{eq:positive_frequencies}
\end{equation}

З урахуванням \eqref{eq:conjugate_frequency} для дійсної функції \( f(t) \) можна записати:

\begin{equation}
\int\limits_{-\infty}^{\infty} f^2(t) dt = \int\limits_{-\infty}^{\infty} |\hat{f}(\omega)|^2 d\omega = 2\int\limits_{0}^{\infty} |\hat{f}(\omega)|^2
d\omega
\label{eq:energy_relation}
\end{equation}

%% --------------------------------------------------------
\subsection*{Енергетичні співвідношення}
%% --------------------------------------------------------

Повна енергія, що проходить через поверхню, що оточує об’єм \( \Omega \) за весь час випромінювання:

\begin{multline*}
\mathcal{E} = \int\limits_{-\infty}^{\infty} \oint\limits_{\Omega} \Pi dS dt = \frac{c}{4\pi} \oint\limits_{\Omega} \int\limits_{-\infty}^{\infty}
\left[ \Efield \times \Bfield \right] dtdS = \\
= \frac{c}{4\pi} \oint\limits_{\Omega} dS \left[ \int\limits_{-\infty}^{\infty} \Efield(\omega) \times \Bfield^{*}(\omega) \right] d\omega
\end{multline*}

де \( \Efield(\omega), \Bfield(\omega) \) --- спектральні характеристики дійсних полів \( \Efield(t), \Bfield(t) \).

Аналогічно \eqref{eq:conjugate_frequency}:

\begin{equation*}
\Efield(-\omega) = \Efield^{*}(\omega), \quad \Bfield(-\omega) = \Bfield^{*}(\omega).
\end{equation*}

Тоді, розбиваючи інтеграл по \( d\omega \) на два додатки:

\begin{equation*}
\int\limits_{-\infty}^{\infty} d\omega \left\{ \ldots \right\} = \int\limits_{-\infty}^{\infty} d\omega \left\{ \ldots \right\} +
\int\limits_{-\infty}^{\infty}
d\omega
\left\{ \ldots \right\},
\end{equation*}

кожне з яких є комплексним спряженням іншого, аналогічно \eqref{eq:positive_frequencies}, маємо:

\begin{equation}
\mathcal{E} = \frac{c}{2\pi} \oint_{\Omega} dS \left[ \int\limits_{0}^{\infty} \Efield(\omega) \times \Bfield^{*}(\omega) \right] d\omega
\label{eq:energy_positive_frequencies}
\end{equation}
де фігурують вже тільки додатні частоти.

Якщо експериментальна техніка дає змогу виділяти внесок окремих інтервалів частот, доцільно ввести спектральну густину енергії випромінювання.

\begin{equation}
\frac{d\mathcal{E}}{d\omega} = \frac{c}{2\pi} \int dS \, \Re\left(\Efield(\omega) \times \Bfield^{*}(\omega)\right), \quad \omega > 0
\label{eq:energy_density}
\end{equation}

Тоді з \eqref{eq:energy_positive_frequencies}:

\begin{equation}
\mathcal{E} = \int \frac{dg}{d\omega} d\omega
\label{eq:total_energy_integral}
\end{equation}

%% --------------------------------------------------------
\subsection*{Комплексний формалізм та енергетичні співвідношення для монохроматичних полів}
%% --------------------------------------------------------

В попередніх співвідношеннях комплексні величини \(\Efield\), \(\Bfield\) --- це образи Фур’є відповідних дійсних полів. Доцільно розглядати також
комплексні поля, які є розв’язками рівнянь Максвелла. Завдяки лінійності рівнянь, дійсні та уявні частини комплексних розв’язків також є розв’язками,
але розрахунки з комплексними величинами часто є більш зручними. При розгляді ж спостережних величин або нелінійних (наприклад, енергетичних)
співвідношень слід повернутися до дійсних частин розв’язків, які саме мають фізичний зміст.

Проілюструємо це на прикладі комплексних монохроматичних полів, тобто таких, що залежать від часу як \(\exp(-i\omega t)\), \(\omega \neq 0\). Нехай
електричне та магнітне поля монохроматичними, тобто

\begin{equation}
\Efield_{\text{real}}(t, \vect{r}) = \Efield'_0(\vect{r}) \cos(\omega t + \alpha), \quad \Bfield_{\text{real}}(t, \vect{r}) = \Bfield'_0(\vect{r})
\cos(\omega t + \alpha)
\label{eq:real_fields}
\end{equation}

Ми зіставляємо їм комплексні поля

\begin{equation}
\Efield(t, \vect{r}) = \Efield_0(\vect{r}) e^{-i\alpha} \exp(-i\omega t), \quad \Bfield(t, \vect{r}) = \Bfield_0(\vect{r}) e^{-i\alpha}
\exp(-i\omega t)
\label{eq:complex_fields}
\end{equation}

\begin{equation*}
\Efield_0(\vect{r}) = \Efield'_0(\vect{r}) e^{-i\alpha}, \quad \Bfield_0(\vect{r}) = \Bfield'_0(\vect{r}) e^{-i\alpha},
\end{equation*}
дійсні частини яких збігаються з полями \(\Efield_{\text{real}}\), \(\Bfield_{\text{real}}\).

Вектор Пойнгінга густини потоку енергії виражаємо, як звичайно, через дійсні частини полів:

\begin{multline*}
\vect{\Pi} = \frac{c}{4\pi} \left( \frac{\Efield + \Efield^{*}}{2} \times \frac{\Bfield + \Bfield^{*}}{2} \right) = \\
= \frac{c}{8\pi} \left( \Efield \times \Bfield^{*} + \Efield^{*} \times \Bfield \right) + \frac{c}{16\pi} \left( \Efield \times \Bfield +
\Efield^{*} \times \Bfield^{*} \right).
\end{multline*}

Складові другого доданку осцилюють як \(\exp(\pm 2i\omega t)\); їх середнє за часом дорівнює нулю. Усереднюючи за часом, маємо:

\begin{equation}
\langle \vect{\Pi} \rangle_t = \frac{c}{8\pi} \Re\left(\Efield \times \Bfield^{*}\right)
\label{eq:time_averaged_poynting}
\end{equation}
де символ \(\langle F(t) \rangle_t = \lim_{T \to \infty} \left\{ \frac{1}{T} \int\limits_{-T}^{T} F(t) dt \right\}\) означає усереднення за часом.
Фактично,
\eqref{eq:time_averaged_poynting} --- це вираз для середньої за часом густини потоку енергії через амплітуди монохроматичних електричного та магнітного
полів.

Аналогічним чином легко отримати середнє за часом значення густини енергії електромагнітного поля:

\begin{equation}
\langle W \rangle_t = \frac{\lvert \Efield \rvert^2 + \lvert \Bfield \rvert^2}{16\pi}
\label{eq:energy_density_average}
\end{equation}

Зауважимо, що формули \eqref{eq:time_averaged_poynting}, \eqref{eq:energy_density_average} не змінять свого вигляду у разі суперпозиції монохроматичних
хвиль з різними частотами. Це очевидно, коли врахувати, що \(\langle \exp[i(\omega_1 - \omega_2)t] \rangle_t = 0\), \(\omega_1 \neq \omega_2\).

Розглянемо монохроматичні плоскі хвилі

\begin{equation*}
\Efield = \Efield_0 \exp[i(\vect{k} \cdot \vect{r} - \omega t)], \quad \Bfield = \Bfield_0 \exp[i(\vect{k} \cdot \vect{r} - \omega t)]
\label{eq:monochromatic_waves}
\end{equation*}

Підставимо це в рівняння Максвелла. Маємо, за відсутності зарядів та струмів:

\begin{align*}
\Rot\Efield &= i[\vect{k} \times \Efield] = \frac{i\omega}{c} \Bfield, \\
\Rot\Bfield &= i[\vect{k} \times \Bfield] = -\frac{i\omega}{c} \Efield
\end{align*}
а також умову поперечності:

\begin{align*}
\Div\Efield &= i(\vect{k} \cdot \Efield) = 0, \\
\Div\Bfield &= i(\vect{k} \cdot \Bfield) = 0
\end{align*}

З першого рівняння:

\begin{equation*}
[\vect{k} \times [\vect{k} \times \Efield]] = \vect{k}(\vect{k} \cdot \Efield) - \Efield k^2 = \frac{\omega}{c}[\vect{k} \times \Bfield] =
-\frac{\omega^2}{c^2} \Efield
\end{equation*}

Завдяки умові поперечності маємо:

\begin{equation*}
k^2 = \frac{\omega^2}{c^2}
\label{eq:dispersion_relation}
\end{equation*}
--- зв’язок між хвильовим вектором та частотою, який називають дисперсійним рівнянням.

Таким чином:

\begin{equation*}
\Bfield = [\vect{n} \times \Efield], \quad (\Bfield \cdot \vect{n}) = (\Efield \cdot \vect{n}) = 0, \quad (\Efield \cdot \Bfield) = 0
\label{eq:wave_properties}
\end{equation*}
де \(\vect{n} = \vect{k} / |\vect{k}|\) визначає напрям поширення хвилі.

%% --------------------------------------------------------
\subsection*{Плоскі хвилі у загальному випадку}
%% --------------------------------------------------------

Монохроматичні плоскі хвилі є частковим видом плоских хвиль, що визначаються, як розв’язки рівнянь електродинаміки, параметри яких не
змінюються при переміщеннях вздовж деякої площини --- фронту хвилі, нормаль до якої \(\vect{n}\) не змінюється з часом. Інакше, просторова залежність
цих розв’язків зводиться до залежності від деякої змінної \((\vect{n} \cdot \vect{r})\), \(\vect{n}^2 = 1\). Тоді

\begin{equation*}
\Efield = \Efield(\vect{n} \cdot \vect{r}, t), \quad \Bfield = \Bfield(\vect{n} \cdot \vect{r}, t)
\end{equation*}

Підставляючи в рівняння Максвелла, маємо:

\begin{equation}
\Rot\Efield = \vect{n} \times \frac{\partial \Efield}{\partial (\vect{n} \cdot \vect{r})} = -\frac{1}{c} \frac{\partial \Bfield}{\partial t}
\label{eq:maxwell_rotE}
\end{equation}

\begin{equation}
\Rot\Bfield = \vect{n} \times \frac{\partial \Bfield}{\partial (\vect{n} \cdot \vect{r})} = \frac{1}{c} \frac{\partial \Efield}{\partial t}
\label{eq:maxwell_rotB}
\end{equation}

\begin{equation}
\Div\Efield = \vect{n} \cdot \frac{\partial \Efield}{\partial (\vect{n} \cdot \vect{r})} = 0, \quad \Div\Bfield = \vect{n} \cdot
\frac{\partial \Bfield}{\partial (\vect{n} \cdot \vect{r})} = 0
\label{eq:maxwell_div}
\end{equation}

Обчислимо \(\Rot(\eqref{eq:maxwell_rotE})\):

\begin{equation*}
\Rot(\Rot\Efield) = \grad (\Div\Efield) - \nabla^2 \Efield = -\nabla^2 \Efield
\end{equation*}

З \eqref{eq:maxwell_rotE}:

\begin{equation*}
-\Rot\left( \frac{1}{c} \frac{\partial \Bfield}{\partial t} \right) = -\frac{1}{c^2} \frac{\partial^2 \Efield}{\partial t^2}
\end{equation*}

Тобто:

\begin{equation}
\nabla^2 \Efield = \frac{1}{c^2} \frac{\partial^2 \Efield}{\partial t^2}
\label{eq:wave_equation_E}
\end{equation}

Підставляючи сюди вираз для плоских хвиль \(\Efield = \Efield(\vect{n} \cdot \vect{r}, t)\):

\begin{equation*}
\frac{\partial^2 \Efield}{\partial (\vect{n} \cdot \vect{r})^2} = \frac{1}{c^2} \frac{\partial^2 \Efield}{\partial t^2}
\end{equation*}

Заміна \(U = \vect{n} \cdot \vect{r} - ct\), \(V = \vect{n} \cdot \vect{r} + ct\) приводить до рівняння:

\begin{equation*}
\frac{\partial^2 \Efield}{\partial U \partial V} = 0,
\end{equation*}

загальним розв’язком якого є \(\Efield = \vect{F}_1(U) + \vect{F}_2(V)\), де \(\vect{F}_1\) та \(\vect{F}_2\) --- довільні вектор-функції одної
змінної. Розв’язок є суперпозицією двох хвиль, що рухаються в протилежних напрямках.

Якщо \(\Efield = \Efield(\vect{n} \cdot \vect{r}, t)\), \(\Bfield = \Bfield(\vect{n} \cdot \vect{r}, t)\), з рівняння \eqref{eq:maxwell_rotE} маємо:

\begin{equation*}
\vect{n} \times \frac{\partial \Efield}{\partial (\vect{n} \cdot \vect{r})} = -\frac{1}{c} \frac{\partial \Bfield}{\partial t}
\end{equation*}

Якщо перед фронтом \(\Efield = \Bfield = 0\), звідси \(\Bfield = [\vect{n} \times \Efield]\), як і в плоскій монохроматичній хвилі.

Аналогічно, з \eqref{eq:maxwell_rotB}:

\begin{equation*}
\Efield = [\Bfield \times \vect{n}], \quad (\Bfield \cdot \vect{n}) = (\Efield \cdot \vect{n}) = 0
\end{equation*}


%% --------------------------------------------------------
\section{Випадкові поля випромінювання}
%% --------------------------------------------------------

%% --------------------------------------------------------
\subsection*{Основні характеристики випадкових полів}
%% --------------------------------------------------------

Монохроматичні хвилі являють дуже ідеалізований випадок хвильового поля. Найчастіше в природі ми зустрічаємось із суперпозиціями полів, створених
випадковими реалізаціями випромінювачів (атомів, молекул тощо). Фізичний опис процесу вимірювання цих величин має справу з обчисленням середніх значень
за часом або середніх за деякими просторовими масштабами. Розмір області усереднення в кожному фізичному процесі визначається окремо. Але за певних
широких умов цей спосіб усереднення еквівалентний статистичному усередненню, коли замість одного поля розглядають велику кількість реалізацій ---
ансамбль хвильових полів --- і обчислення проводять за допомогою усереднень за ансамблем\footnote{Стаціонарні процеси, для яких часові середні
збігаються з середніми за ансамблем, називають ергодичними. Дослідник має вирішити, наскільки модель ергодичного процесу адекватна конкретній фізичній
ситуації.}. Цей підхід добре відомий з статистичної механіки. Його можна застосовувати також при обчисленні електричних та магнітних величин, що
вимірюють за допомогою макроскопічних приладів.

Розглянемо спочатку електромагнітне поле, для простоти не звертаючи уваги на ефекти, пов’язані з поляризацією. Тоді це поле можна описувати за допомогою
скалярної функції \(\phi(t)\), що представляє, наприклад, одну з компонент електричного чи магнітного поля, маючи на увазі, що інші компоненти можна
розглянути аналогічно. Будемо користуватися комплексним формалізмом (див.~\ref{sec:spectral}); нагадаємо, що при цьому в виразі для густини енергії
\eqref{eq:energy_density_average} фігурують квадрати модулів полів.

Математичний опис стохастичних полів використовує поняття випадкового процесу. Нагадаємо, що випадковий процес \(\phi(t)\) заданий, якщо для будь-якого
набору \(\{t_1, t_2, \dots, t_N\}\) заданий спільний розподіл ймовірностей

\begin{equation*}
w(\phi_1, \phi_2, \dots, \phi_N; t_1, t_2, \dots, t_N)
\end{equation*}

величин \(\phi(t_1), \phi(t_2), \dots, \phi(t_N)\). Процес називають стаціонарним, якщо спільний розподіл

\begin{equation*}
w(\phi_1, \phi_2, \dots, \phi_N; t_1 + \tau, t_2 + \tau, \dots, t_N + \tau)
\end{equation*}

величин \(\phi(t_1 + \tau), \phi(t_2 + \tau), \dots, \phi(t_N + \tau)\) не залежить від часового зсуву \(\tau\). Знаючи розподіл ймовірностей значень
поля для будь-яких часів \(\{t_1, t_2, \dots, t_N\}\), можна обчислювати середні від лінійних, квадратичних за полем величин тощо. Ці середні в цьому
підрозділі ми позначаємо знаком \(\langle \dots \rangle\). З фізичної точки зору процедуру усереднення можна інтерпретувати, як обчислення
середньоарифметичних величин за даними великої кількості експериментів-реалізацій. Така інтерпретація часто є цілком достатньою для проведення обчислень.

Для стаціонарного процесу середнє \(\langle \phi(t) \rangle = \langle \phi \rangle\). При розгляді випадкових полів випромінювання вважають \(\langle
\phi \rangle = 0\).

Важливою характеристикою випадкового процесу є кореляційна (автокореляційна) функція\footnote{У деяких підручниках у визначенні
\eqref{eq:autocorrelation} від правої частини віднімають \(\langle \phi(t) \rangle \langle \phi^{*}(t + \tau) \rangle\).}:

\begin{equation}
A(\tau) = \langle \phi^{*}(t) \phi(t + \tau) \rangle
\label{eq:autocorrelation}
\end{equation}

Для стаціонарного процесу:

\begin{equation*}
A(-\tau) = \langle \phi^{*}(t) \phi(t - \tau) \rangle = \langle \phi^{*}(t + \tau) \phi(t) \rangle = A^{*}(\tau)
\end{equation*}

не залежить від \(t\).

Розглянемо формально перетворення Фур'є випадкової функції:

\begin{equation*}
\Phi(\omega) = \frac{1}{\sqrt{2\pi}} \int\limits_{-\infty}^{\infty} dt \, e^{i\omega t} \phi(t)
\end{equation*}

Щоб надати змісту цьому невласному інтегралу в разі стаціонарного випадкового процесу, для кожної реалізації можна розглядати вираз:

\begin{equation*}
\Phi_T(\omega) = \frac{1}{\sqrt{2\pi}} \int\limits_{-T}^{T} dt \, e^{i\omega t} \phi(t)
\end{equation*}

Обчислимо для стаціонарного процесу:

\begin{equation*}
\langle \Phi(\omega) \Phi^{*}(\omega') \rangle = \frac{1}{2\pi} \int\limits_{-\infty}^{\infty} dt \, e^{i(\omega - \omega')t}
\int\limits_{-\infty}^{\infty} dt' \,
e^{i\omega(t - t')} \langle \phi^{*}(t') \phi(t) \rangle
\end{equation*}

\begin{equation*}
= \frac{1}{2\pi} \int\limits_{-\infty}^{\infty} dt \, e^{i(\omega - \omega')t} \int\limits_{-\infty}^{\infty} dt' \, e^{i\omega(t - t')} A(t - t')
\end{equation*}

Користуючись відомим співвідношенням:

\begin{equation*}
\lim_{T \to \infty} \int\limits_{-T}^{T} dt \, e^{i(\omega - \omega')t} = 2\pi \delta(\omega - \omega')
\end{equation*}

дістанемо:

\begin{equation}
\langle \Phi(\omega) \Phi^{*}(\omega') \rangle = \delta(\omega - \omega') S(\omega)
\label{eq:spectral_density}
\end{equation}
де

\begin{equation*}
S(\omega) = \int\limits_{-\infty}^{\infty} d\tau \, e^{i\omega \tau} A(\tau)
\end{equation*}
називають спектральною густиною (чи спектром потужності) випадкового процесу. Очевидно

\begin{equation*}
A(\tau) = \frac{1}{2\pi} \int\limits_{-\infty}^{\infty} d\omega \, e^{-i\omega \tau} S(\omega)
\end{equation*}

%% --------------------------------------------------------
\subsection*{Когерентність}
%% --------------------------------------------------------

Для суперпозиції двох хвильових полів \(\phi_1\) та \(\phi_2\) сумарна інтенсивність визначається квадратичною величиною:

\begin{equation*}
I = |\phi_1 + \phi_2|^2 = |\phi_1|^2 + |\phi_2|^2 + 2 \Re(\phi_1 \phi_2^{*})
\end{equation*}

а для середніх:

\begin{equation}
\langle I \rangle = \langle I_1 \rangle + \langle I_2 \rangle + 2 \Re \langle \phi_1 \phi_2^{*} \rangle
\label{eq:intensity}
\end{equation}

де \( I_i = \langle |\phi_i|^2 \rangle \) --- інтенсивності окремих полів, \( i = 1, 2 \).

Розглянемо дві протилежних ситуації --- повної когерентності та некогерентності двох полів.

%% --------------------------------------------------------
\subsection*{Повна когерентність}
%% --------------------------------------------------------

Суперпозиція двох полів, що пов’язані умовою:

\begin{equation*}
\phi_2 = \alpha \phi_1 = |\alpha| \phi_1 e^{i\delta}, \quad \text{де } \delta = \text{Arg} \alpha.
\end{equation*}

В цьому разі:

\begin{equation*}
\langle \phi_1 \phi_2^{*} \rangle = \phi_1 \phi_2^{*} = 2\sqrt{I_1 I_2} \cos \delta
\end{equation*}

Умова зв’язку між \(\phi_1\) та \(\phi_2\) є випадком ідеальної корельованості, або когерентності цих полів. Такі поля можна отримати при розділенні
світлового пучка в інтерферометрі, або, наприклад, у разі двох монохроматичних радіо-джерел, що мають однакову частоту.


%% --------------------------------------------------------
\subsection*{Некогерентність}
%% --------------------------------------------------------

Суперпозиція двох незалежних полів, генерованих двома незалежними випадковими випромінювачами:

\begin{equation*}
\langle \phi_1 \phi_2^{*} \rangle = \langle \phi_1 \rangle \langle \phi_2^{*} \rangle
\end{equation*}

Як правило, для хвильового випадкового поля його середнє значення є нуль:

\begin{equation*}
\langle \phi_1 \rangle = \langle \phi_2 \rangle = 0,
\end{equation*}

тому \(\langle \phi_1 \phi_2^{*} \rangle = 0\), і загальна інтенсивність є сумою інтенсивностей окремих випромінювачів:

\begin{equation*}
I = I_1 + I_2.
\end{equation*}

Випромінювання двох різних природних джерел світла є некогерентним.

%% --------------------------------------------------------
\subsection*{Функція взаємної когерентності}
%% --------------------------------------------------------

Величину \(I' = \langle \phi_1 \phi_2^{*} \rangle\), яку можна вимірювати за допомогою радіофізичних та оптичних пристроїв, називають функцією взаємної
когерентності. Тут \(\phi_1\) та \(\phi_2\) можуть бути двома різними полями, або значеннями одного й того ж поля у різних точках:

\begin{equation*}
\phi_1 = \phi(t_1, \vect{r}_1), \quad \phi_2 = \phi(t_2, \vect{r}_2)
\end{equation*}

Величину

\begin{equation*}
\gamma = \frac{\langle \phi_1 \phi_2^{*} \rangle}{|\phi_1||\phi_2|}, \quad \text{де } |\phi_i| \equiv \sqrt{\langle \phi_i \phi_i^{*} \rangle}, \quad i
= 1, 2,
\end{equation*}

називають степінню взаємної когерентності. Саме ця величина визначає контрастність смуг в інтерференційній картині від двох пучків світла.

Оскільки, за нерівністю Коші-Буняковського\footnote{Для будь-якого комплексного \(\lambda\) величина \(Z = \langle (\phi + \lambda \psi)(\phi^{*} +
\lambda^{*} \psi^{*}) \rangle\) --- невід’ємна. Покладемо \(\lambda = -\langle \psi \phi^{*} \rangle / \langle |\psi|^2 \rangle\), тоді \(Z = \langle
|\phi|^2 \rangle - |\langle \psi \phi^{*} \rangle|^2 / \langle |\psi|^2 \rangle\), звідки й випливає потрібне.}, \(|\langle \phi_1 \phi_2^{*} \rangle|
\leq |\phi_1||\phi_2|\), маємо \(|\gamma| \leq 1\).

\subsection*{Поляризація}

При розгляді поляризації треба зважати на векторну природу польових функцій, зокрема, напруженості електричного поля. Зауважимо, що для фіксованого
напрямку випромінювання досить розглядати лише електричне поле; воно визначає також і вектор напруженості магнітного поля.

Розглянемо поле випромінювання, що поширюється в напрямку осі \(x_3\), з напруженістю електричного поля \(\Efield = (E_1, E_2, 0)\), причому будемо
вважати, що поле є суперпозицією монохроматичних хвиль \(\sim e^{-i\omega t}\) з близькими частотами. Напрямок вектора \(\Efield\), що є ортогональним
напрямку поширення хвилі, визначає поляризацію. Для випадкового поля \(\Efield\) введемо тензор поляризації:

\begin{equation}
\rho_{ij} = \frac{\langle E_i E_j^{*} \rangle}{\langle |E_1|^2 \rangle + \langle |E_2|^2 \rangle}, \quad i, j = 1, 2
\label{eq:polarization_tensor}
\end{equation}

Очевидно, матриця \(\rho\) є ермітовою, а її шпур дорівнює одиниці:

\begin{equation*}
\rho_{ji} = \rho_{ij}^{*}, \quad \rho_{11} + \rho_{22} = 1
\end{equation*}

Для цілком поляризованого випромінювання маємо \(E_i = N_i f(t)\), де \(f\) може бути випадковою функцією, але вектор \(N_i\), що визначає напрямок
поля, є фіксованим. В цьому разі \(\rho_{ij}\) пропорційне \(N_i N_j^{*}\) і:

\begin{equation*}
\det \rho = 0
\end{equation*}

У разі природного світла присутні усі можливі поляризації, що є рівноправними, причому \(\langle E_1 E_2^{*} \rangle = 0\), \(\langle |E_1|^2 \rangle =
\langle |E_2|^2 \rangle\). В цьому разі:

\begin{equation*}
\rho_{ij} = \frac{1}{2} \delta_{ij}
\end{equation*}

В загальному випадку визначник матриці \(\rho\) є:

\begin{equation*}
\det \rho = \rho_{11} \rho_{22} - |\rho_{12}|^2 = \frac{\langle |E_1|^2 \rangle \langle |E_2|^2 \rangle - |\langle E_1 E_2^{*} \rangle|^2}{\left(
\langle |E_1|^2 \rangle + \langle |E_2|^2 \rangle \right)^2}
\end{equation*}

Оскільки \(|\langle E_1 E_2^{*} \rangle|^2 \leq \langle |E_1|^2 \rangle \langle |E_2|^2 \rangle\), цей визначник додатний:

\begin{equation*}
\det \rho \geq 0
\end{equation*}

З іншого боку, очевидно:

\begin{equation*}
\det \rho \leq \frac{1}{4}
\end{equation*}

Оскільки область значень \(\det \rho\) є відрізком \([0, 1/4]\), можна записати:

\begin{equation}
\det \rho = \frac{1}{4} \left( 1 - p^2 \right), \quad 0 \leq p \leq 1
\label{eq:polarization_degree}
\end{equation}

де параметр \(p\) називають ступінню поляризації; він приймає значення від 0 (неполяризоване світло) до 1.

Завдяки властивості ермітовості, власні числа \(\lambda_1, \lambda_2\) матриці \(\rho_{ij}\) є дійсними, причому \(\det \rho = \lambda_1 \lambda_2\),
\(\lambda_1 + \lambda_2 = 1\). Зважаючи на \eqref{eq:polarization_degree}, добуток власних чисел додатний, тобто вони мають один знак. Звідси легко
бачити, що ці числа є додатними. Величину \(\lambda_{\min}/\lambda_{\max}\) називають коефіцієнтом деполяризації.

%% --------------------------------------------------------
\subsection*{Параметри Стокса}
%% --------------------------------------------------------

Виходячи з властивостей тензора поляризації, поданих формулою \eqref{eq:polarization_tensor}, його компоненти \(\rho_{ij}\) можна виразити через три
незалежні дійсні параметри Стокса \(\xi_1, \xi_2, \xi_3\), які приймають значення від -1 до 1:

\begin{equation*}
\rho_{11} = \frac{1}{2} \left( 1 + \xi_3 \right), \quad \rho_{22} = \frac{1}{2} \left( 1 - \xi_3 \right), \quad \rho_{12} = \frac{1}{2} \left( \xi_1 -
i\xi_2 \right) = \rho_{21}^{*}
\end{equation*}

Очевидно:

\begin{equation*}
\det \rho = \frac{1}{4} \left( 1 - \xi_1^2 - \xi_2^2 - \xi_3^2 \right)
\end{equation*}

\begin{equation*}
P = \xi_1^2 + \xi_2^2 + \xi_3^2, \quad 0 \leq \xi_1^2 + \xi_2^2 + \xi_3^2 \leq 1
\end{equation*}

Розглянемо стани поляризації для деяких полів, що відповідають різним параметрам Стокса.

\begin{itemize}

\item Лінійна поляризація під кутом $45^{\circ}$, тобто \(E_1 = \pm E_2\):

\begin{equation*}
\rho_{11} = \rho_{22} = \frac{1}{2} \rightarrow \xi_3 = 0; \quad \text{Im} \, \rho_{12} = 0 \rightarrow \xi_2 = 0;
\end{equation*}

Таким чином:

\begin{equation*}
\xi_1 = \pm 1, \quad \xi_3 = \xi_2 = 0
\end{equation*}



\item  Кутова поляризація \(E_1 = A e^{-i\omega t}, \quad E_2 = \pm i E_1\).

Як видно з поведінки дійсних частин компонент, вектор \(\Efield\) обертається в площині \(X_1 -
X_2\). Маємо:

\begin{equation*}
\rho_{11} = \frac{1}{2}, \quad \rho_{22} = \frac{1}{2}, \quad \rho_{12} = \mp \frac{i}{2}
\end{equation*}

Звідси:

\begin{equation*}
\xi_1 = \xi_3 = 0, \quad \xi_2 = \pm 1
\end{equation*}

\item Лінійна поляризація вздовж однієї з осей \(E_1 \neq 0, E_2 = 0\) або \(E_2 \neq 0, E_1 = 0\):

\begin{equation*}
\rho_{11} = 1, \quad \rho_{22} = 0 \rightarrow \xi_3 = 1,
\end{equation*}

або

\begin{equation*}
\rho_{11} = 0, \quad \rho_{22} = 1 \rightarrow \xi_3 = -1,
\end{equation*}

при цьому \(\rho_{12} = 0 \rightarrow \xi_1 = \xi_2 = 0\).

Таким чином:

\begin{equation*}
\xi_1 = \xi_2 = 0, \quad \xi_3 = \pm 1
\end{equation*}

\end{itemize}

%% --------------------------------------------------------
\section{Співвідношення невизначеностей}
%% --------------------------------------------------------

Міркування цього розділу застосовні до будь-яких процесів, що можна аналізувати за допомогою перетворення Фур’є, зокрема, до хвильових пакетів, які є
суперпозицією плоских хвиль з різними частотами.

Нехай маємо деякий процес, що можна описати за допомогою функції (взагалі кажучи, комплексної) від дійсної змінної \( t \):

\begin{equation*}
\phi(t) = \frac{1}{\sqrt{2\pi}} \int d\omega \cdot e^{-i\omega t} \phi(\omega)
\end{equation*}

Вважатимемо, що \( |\phi(t)| \) та \( |\phi(\omega)| \) досить швидко спадають, якщо відповідно \( |t| \to \infty \) та \( |\omega| \to \infty \).

Введемо величини, що характеризують тривалість процесу, що відповідає функції \( \phi(t) \), та ширину його спектру відповідно до \( \phi(\omega) \).
Середні значення в \( t \)-просторі обчислюватимемо з вагою \( |\phi(t)|^2 \), а в \( \omega \)-просторі --- з вагою \( |\phi(\omega)|^2 \), причому, не
зменшуючи загальності, приймемо умову нормування:

\begin{equation*}
\int dt \cdot |\phi(t)|^2 = 1
\end{equation*}

Звідси, за рівністю Парсеваля, в \( \omega \)-просторі також:

\begin{equation*}
\int d\omega \cdot |\phi(\omega)|^2 = 1
\end{equation*}

Відповідно, середнє значення деякої величини \( F(t) \) у \( t \)-просторі буде:

\begin{equation*}
\langle F \rangle_{\phi} = \int dt \cdot F(t) |\phi(t)|^2,
\end{equation*}
а для \( F(\omega) \) в \( \omega \)-просторі:

\begin{equation*}
\langle F \rangle_{\phi} = \int d\omega \cdot F(\omega) |\phi(\omega)|^2
\end{equation*}

Виберемо відлік часу таким чином, щоби:

\begin{equation*}
\int dt \cdot t |\phi(t)|^2 = 0
\end{equation*}

Тоді тривалість процесу \( \phi(t) \) можна описати середньоквадратичним значенням:

\begin{equation}
\langle \Delta t^2 \rangle_{\phi} = \int dt \cdot t^2 |\phi(t)|^2
\label{eq:time_duration}
\end{equation}

Відповідно, в \( \omega \)-просторі середнє значення частоти є:

\begin{equation*}
\omega_a = \langle \omega \rangle_{\phi} = \int d\omega \cdot \omega |\phi(\omega)|^2,
\end{equation*}
а ширину спектру частот можна описати величиною:

\begin{equation}
\langle \Delta \omega^2 \rangle_{\phi} = \langle (\omega - \omega_a)^2 \rangle_{\phi} = \int d\omega \cdot (\omega - \omega_a)^2 |\phi(\omega)|^2
\label{eq:frequency_width}
\end{equation}

Покажемо, що існує спільне обмеження на ці величини. Для цього перепишемо \eqref{eq:time_duration} через перетворення Фур’є.

Маємо:

\begin{equation*}
t \phi(t) = \frac{1}{\sqrt{2\pi}} \int d\omega \cdot i \frac{d e^{-i\omega t}}{d\omega} \phi(\omega) = \frac{1}{\sqrt{2\pi}} \int d\omega \cdot
e^{-i\omega t} \left[ -i \frac{d\phi}{d\omega} \right],
\end{equation*}
тобто помноження на \( t \) індукує диференціювання у просторі частот. Тоді формулу \eqref{eq:time_duration} можна переписати так:

\begin{equation}
\langle \Delta t^2 \rangle_{\phi} = \int dt \cdot |t \phi(t)|^2 = \int d\omega \left| \frac{d\phi}{d\omega} \right|^2
\label{eq:time_duration_fourier}
\end{equation}

де застосовано рівність Парсеваля.

Для будь-якого дійсного \( x \):

\begin{equation*}
0 \leq \int d\omega \left| x \frac{d\phi}{d\omega} + (\omega - \omega_a) \phi \right|^2 \equiv \int d\omega \left| x \frac{d\phi}{d\omega} + (\omega -
\omega_a) \phi \right| \left[ x \frac{d\phi^{*}}{d\omega} + (\omega - \omega_a) \phi^{*} \right].
\end{equation*}

Звідси, за допомогою \eqref{eq:frequency_width} та \eqref{eq:time_duration_fourier}, маємо:

\begin{equation*}
0 \leq x^2 \langle \Delta t^2 \rangle_{\phi} + \langle \Delta \omega^2 \rangle_{\phi} + x \int d\omega (\omega - \omega_a) \left[ \frac{d\phi}{d\omega}
\phi^{*} + \frac{d\phi^{*}}{d\omega} \phi \right].
\end{equation*}

Вираз у квадратних дужках в останньому доданку є повною похідною, тому інтеграл у правій частині нерівності перетворимо так:

\begin{equation*}
\int d\omega (\omega - \omega_a) \frac{d}{d\omega} (\phi \phi^{*}) = -\int d\omega |\phi|^2 \frac{d}{d\omega} (\omega - \omega_a) = -1.
\end{equation*}

Таким чином:

\begin{equation*}
x^2 \langle \Delta t^2 \rangle_{\phi} + \langle \Delta \omega^2 \rangle_{\phi} - x \geq 0.
\end{equation*}

Це співвідношення має виконуватися для будь-яких \( x \), тому дискримінант квадратного тричлена має бути від’ємний. Звідси маємо нерівність:

\begin{equation}
\langle \Delta t^2 \rangle_{\phi} \langle \Delta \omega^2 \rangle_{\phi} \geq \frac{1}{4},
\label{eq:uncertainty_relation}
\end{equation}
яку називають співвідношенням невизначеностей.

Зауважимо, що в квантовій механіці це дає зв’язок між тривалістю процесу та невизначеністю енергії.

З \eqref{eq:uncertainty_relation} випливає, що сигнал, що близький до монохроматичного, тобто такий, що має малий розкид частот, має бути досить
тривалим. Навпаки, короткочасний імпульсний сигнал має широкий спектр частот. Нерівність, аналогічну \eqref{eq:uncertainty_relation}, можна отримати для
довжини хвильового пакету та ширини інтервалу хвильових векторів у певному напрямі. Квантовий відповідник --- співвідношення невизначеностей для
імпульсу та координати.

%% --------------------------------------------------------
\section{Загальний розв’язок рівнянь вільного поля}
%% --------------------------------------------------------

Далі буде отримано загальний розв’язок рівнянь Максвелла за відсутності джерел (\(\rho = 0\), \(\vect{j} = 0\)). В цьому разі система рівнянь поля є
однорідною і описує вільні електромагнітні хвилі. \textit{Загальний розв’язок} \textit{неоднорідних рівнянь} Максвелла (за наявності зарядів та струмів)
може бути побудований як суперпозиція \textit{загального розв’язку однорідних рівнянь вільного поля} та \textit{часткового розв’язку неоднорідних
рівнянь}.

Скористуємось калібруванням Гамільтона \eqref{eq:HamiltonCondition}:

\begin{equation}
\phi = 0
\label{eq:phi_zero}
\end{equation}

З рівняння \eqref{eq:maxwell_div} за умови \(\rho = 0\) маємо:

\begin{equation*}
\frac{\partial}{\partial t} (\Div\vect{A}) = 0
\end{equation*}

Відкидаючи ненульові розв’язки, які не залежать від часу (що описують статичні поля, а не електромагнітні хвилі), дістанемо:

\begin{equation}
\Div\vect{A} = 0
\label{eq:div_A_zero}
\end{equation}

За цією умовою рівняння \eqref{eq:DA_for_Hamilton1} зводиться до однорідного хвильового рівняння:

\begin{equation}
\frac{1}{c^2} \frac{\partial^2 \vect{A}}{\partial t^2} - \nabla^2 \vect{A} = 0
\label{eq:wave_equation_A}
\end{equation}

Шукатимемо розв’язок рівняння \eqref{eq:wave_equation_A} в усьому просторі за допомогою просторового перетворення Фур’є:

\begin{equation}
\vect{A}_{\vect{k}}(t) = \frac{1}{(2\pi)^{3/2}} \int d^3r \cdot e^{-i\vect{k} \cdot \vect{r}} \vect{A}(t, \vect{r})
\label{eq:fourier_transform_A}
\end{equation}

Обернене перетворення:

\begin{equation}
\vect{A}(t, \vect{r}) = \frac{1}{(2\pi)^{3/2}} \int d^3k \cdot e^{i\vect{k} \cdot \vect{r}} \vect{A}_{\vect{k}}(t)
\label{eq:inverse_fourier_transform_A}
\end{equation}
дозволяє знайти напруженості електромагнітного поля за формулами \eqref{eq:potB} та \eqref{eq:potE} за відомим \(\vect{A}_{\vect{k}}(t)\).

Дія оператора Лапласа:

\begin{equation*}
\nabla^2 \vect{A}(t, \vect{r}) = \frac{1}{(2\pi)^{3/2}} \int d^3k \, \nabla^2 \left( e^{i\vect{k} \cdot \vect{r}} \right) \vect{A}_{\vect{k}}(t) =
\frac{1}{(2\pi)^{3/2}} \int d^3k \, e^{i\vect{k} \cdot \vect{r}} (-k^2) \vect{A}_{\vect{k}}(t)
\end{equation*}
індукує домноження Фур’є-образу на \((-k^2)\). Тому з рівняння \eqref{eq:wave_equation_A} дістаємо:

\begin{equation}
\frac{\partial^2 \vect{A}_{\vect{k}}(t)}{\partial t^2} + \omega_{\vect{k}}^2 \vect{A}_{\vect{k}}(t) = 0
\label{eq:oscillator_equation}
\end{equation}
де \(\omega_{\vect{k}} = ck = c|\vect{k}|\).

Рівняння \eqref{eq:div_A_zero} дає умову поперечності:

\begin{equation}
\vect{k} \cdot \vect{A}_{\vect{k}}(t) = 0
\label{eq:transverse_condition}
\end{equation}

Загальний розв’язок рівняння \eqref{eq:oscillator_equation} запишемо у вигляді:

\begin{equation}
\vect{A}_{\vect{k}}(t) = \vect{C}_1(\vect{k}) e^{i\omega_{\vect{k}} t} + \vect{C}_2(\vect{k}) e^{-i\omega_{\vect{k}} t},
\label{eq:general_solution_Ak}
\end{equation}
де в силу \eqref{eq:transverse_condition}:

\begin{equation*}
\vect{C}_1(\vect{k}) \cdot \vect{k} = \vect{C}_2(\vect{k}) \cdot \vect{k} = 0
\end{equation*}

Звідси розв’язок хвильового рівняння \eqref{eq:wave_equation_A} має вигляд:

\begin{equation}
\vect{A}(t, \vect{r}) = \frac{1}{(2\pi)^{3/2}} \int d^3k \, \left\{ \vect{C}_1(\vect{k}) e^{i\omega_{\vect{k}} t} + \vect{C}_2(\vect{k})
e^{-i\omega_{\vect{k}} t} \right\} e^{i\vect{k} \cdot \vect{r}}
\label{eq:general_solution_A}
\end{equation}

Оскільки вихідне поле \(\vect{A}(t, \vect{r})\) є дійсним, функції \(\vect{C}_1(\vect{k})\) та \(\vect{C}_2(\vect{k})\) не є незалежні. Маємо:

\begin{equation*}
\vect{A}(t, \vect{r}) = \vect{A}^*(t, \vect{r}) = \frac{1}{(2\pi)^{3/2}} \int d^3k \, \left\{ \vect{C}_1^*(\vect{k}) e^{-i\omega_{\vect{k}} t} +
\vect{C}_2^*(\vect{k}) e^{i\omega_{\vect{k}} t} \right\} e^{-i\vect{k} \cdot \vect{r}}
\end{equation*}

Після заміни \(\vect{k} \rightarrow -\vect{k}\) легко отримати:

\begin{equation*}
\vect{A}(t, \vect{r}) = \frac{1}{(2\pi)^{3/2}} \int d^3k \, \left\{ \vect{C}_1^*(-\vect{k}) e^{-i\omega_{\vect{k}} t} + \vect{C}_2^*(-\vect{k})
e^{i\omega_{\vect{k}} t} \right\} e^{i\vect{k} \cdot \vect{r}},
\end{equation*}
де враховано \(\omega(\vect{k}) = c|\vect{k}| = \omega(-\vect{k})\).

Порівнюючи це з \eqref{eq:general_solution_A} і приймаючи до уваги незалежність \(\exp(-i\omega_{\vect{k}} t)\) та \(\exp(i\omega_{\vect{k}} t)\),
отримаємо:

\begin{equation}
\vect{C}_1^*(-\vect{k}) = \vect{C}_2(\vect{k}) \quad \text{або} \quad \vect{C}_1(\vect{k}) = \vect{C}_2^*(-\vect{k}),
\label{eq:C1_C2_relation}
\end{equation}

причому \(\vect{C}_1(\vect{k}) \cdot \vect{k} = 0\).

Подамо цей розв’язок у дещо іншій формі, виділивши лоренц-інваріантні комбінації \(\omega_{\vect{k}} - \vect{k} \cdot \vect{r}\).

У першому доданку в \eqref{eq:general_solution_A} за допомогою заміни \(\vect{k} \rightarrow -\vect{k}\) зробимо очевидні перетворення:

\begin{multline*}
\int \vect{C}_1(\vect{k}) \exp[i(\omega_{\vect{k}} t + \vect{k} \cdot \vect{r})] d^3k = \int \vect{C}_1(-\vect{k}) \exp[i(\omega_{\vect{k}} t - \vect{k}
\cdot \vect{r})] d^3k = \\ =\int \vect{C}_2^*(\vect{k}) \exp[i(\omega_{\vect{k}} t - \vect{k} \cdot \vect{r})] d^3k.
\end{multline*}

Позначаючи \(\vect{a}(\vect{k}) = \vect{C}_2(\vect{k}) = \vect{C}_1^*(-\vect{k})\), маємо представлення розв’язку у вигляді:

\begin{equation}
\vect{A}(t, \vect{r}) = \frac{1}{(2\pi)^{3/2}} \int d^3k \, \left\{ \vect{a}(\vect{k}) e^{-i(\omega_{\vect{k}} t - \vect{k} \cdot \vect{r})} +
\vect{a}^*(\vect{k}) e^{i(\omega_{\vect{k}} t - \vect{k} \cdot \vect{r})} \right\}
\label{eq:A_final_form}
\end{equation}

Формула \eqref{eq:A_final_form} з умовою поперечності \(\vect{k} \cdot \vect{a}(\vect{k}) = 0\) подає загальний розв’язок хвильового рівняння
\eqref{eq:wave_equation_A} з умовою \eqref{eq:div_A_zero}.

Доданок, що містить \(\exp[-i(\omega_{\vect{k}} t - \vect{k} \cdot \vect{r})]\), називають додатно-частотною частиною поля, а комплексно-спряжений йому
доданок --- від’ємно-частотною частиною. Представлення \eqref{eq:A_final_form} називають розбиттям на додатно-частотні та від’ємно-частотні компоненти.

Як бачимо з \eqref{eq:A_final_form}, стан вільного електромагнітного поля визначається, з огляду на умову поперечності \(\vect{k} \cdot \vect{a} = 0\),
двома комплексними функціями від трьох змінних \(\vect{k}\), або --- еквівалентно --- чотирма дійсними функціями від \(\vect{k}\). Кількість цих
незалежних функцій (яка відповідає степеням вільності поля) така ж, як у задачі Коші згідно з п. \ref{sec:Cauchi}.

Кожному \(\vect{k}\) відповідає плоска хвиля \(\vect{a}(\vect{k}) \exp[i(\omega_{\vect{k}} t - \vect{k} \cdot \vect{r})]\) --- частковий розв’язок
рівняння \eqref{eq:wave_equation_A}, причому напрямок вектора \(\vect{a}(\vect{k})\) задає поляризацію.

%% --------------------------------------------------------
\subsection*{Розклад поля на осцилятори}
%% --------------------------------------------------------

За допомогою заміни \(\vect{k} \rightarrow -\vect{k}\) у від’ємно-частотній частині перепишемо \eqref{eq:A_final_form} у вигляді:

\begin{equation}
\vect{A}(t, \vect{r}) = \frac{1}{(2\pi)^{3/2}} \int d^3k \, \left\{ \vect{a}(\vect{k}) e^{-i\omega_{\vect{k}} t} + \vect{a}^*(-\vect{k})
e^{i\omega_{\vect{k}} t} \right\} e^{i\vect{k} \cdot \vect{r}}
\label{eq:A_final_form_modified}
\end{equation}

Згідно з \eqref{eq:potE}, \eqref{eq:potB} та калібрувальною умовою \eqref{eq:phi_zero}:

\begin{equation}
\Efield = -\frac{1}{c} \frac{\partial \vect{A}}{\partial t} = \frac{i}{(2\pi)^{3/2}} \int d^3k \, \omega_{\vect{k}} \left\{ \vect{a}(\vect{k})
e^{-i\omega_{\vect{k}} t} - \vect{a}^*(-\vect{k}) e^{i\omega_{\vect{k}} t} \right\} e^{i\vect{k} \cdot \vect{r}}
\label{eq:E_field_fourier}
\end{equation}

Обчислимо енергію електричного поля за допомогою формули Парсеваля:

\begin{equation}
W_E = \frac{1}{8\pi} \int dV \, \Efield^2 = \frac{1}{8\pi c^2} \int d^3k \, \omega_{\vect{k}}^2 \left\{ \vect{a}^*(\vect{k}) \cdot \vect{a}(\vect{k}) -
\Re \left[ \vect{a}(\vect{k}) \cdot \vect{a}(-\vect{k}) e^{-2i\omega_{\vect{k}} t} \right] \right\}
\label{eq:W_E}
\end{equation}

Оскільки:

\begin{equation*}
\int d^3k \, \vect{a}^*(\vect{k}) \cdot \vect{a}(\vect{k}) = \int d^3k \, \vect{a}^*(-\vect{k}) \cdot \vect{a}(-\vect{k}),
\end{equation*}
маємо:

\begin{equation}
W_E = \frac{1}{4\pi c^2} \int d^3k \, \omega_{\vect{k}}^2 \left\{ \vect{a}^*(\vect{k}) \cdot \vect{a}(\vect{k}) - \text{Re} \left[ \vect{a}(\vect{k})
\cdot \vect{a}(-\vect{k}) e^{-2i\omega_{\vect{k}} t} \right] \right\}
\label{eq:W_E_final}
\end{equation}

Аналогічно:

\begin{equation}
\Bfield = \nabla \times \vect{A} = \frac{1}{(2\pi)^{3/2}} \int d^3k \, \left[ \vect{k} \times \left( \vect{a}(\vect{k}) e^{-i\omega_{\vect{k}} t} +
\vect{a}^*(-\vect{k}) e^{i\omega_{\vect{k}} t} \right) \right] e^{i\vect{k} \cdot \vect{r}}
\label{eq:B_field_fourier}
\end{equation}
а енергія магнітного поля:

\begin{equation}
W_B = \frac{1}{8\pi} \int dV \, \Bfield^2 = \frac{1}{8\pi} \int d^3k \, \left| \vect{k} \times \left( \vect{a}(\vect{k}) e^{-i\omega_{\vect{k}} t} +
\vect{a}^*(-\vect{k}) e^{i\omega_{\vect{k}} t} \right) \right|^2
\label{eq:W_B}
\end{equation}

Оскільки \(\vect{k} \cdot \vect{a}(\vect{k}) = 0\), маємо:

\begin{equation*}
\left| \vect{k} \times \vect{a}(\vect{k}) \right|^2 = k^2 \vect{a}^*(\vect{k}) \cdot \vect{a}(\vect{k}),
\end{equation*}
а також:

\begin{equation*}
\left[ \vect{k} \times \vect{a}(\vect{k}) \right] \cdot \left[ \vect{k} \times \vect{a}(-\vect{k}) \right] = k^2 \vect{a}(\vect{k}) \cdot
\vect{a}(-\vect{k}).
\end{equation*}

Звідси, аналогічно \eqref{eq:W_E_final}:

\begin{equation}
W_B = \frac{1}{4\pi} \int d^3k \, k^2 \left\{ \vect{a}^*(\vect{k}) \cdot \vect{a}(\vect{k}) + \Re \left[ \vect{a}(\vect{k}) \cdot
\vect{a}(-\vect{k}) e^{-2i\omega_{\vect{k}} t} \right] \right\}
\label{eq:W_B_final}
\end{equation}

Оскільки \(\omega_{\vect{k}} = ck\), загальна енергія вільного електромагнітного поля:

\begin{equation}
W = W_E + W_B = \frac{1}{2\pi c^2} \int d^3k \, \omega_{\vect{k}}^2 \vect{a}^*(\vect{k}) \cdot \vect{a}(\vect{k})
\label{eq:total_energy}
\end{equation}

Подамо цей вираз через нові змінні:

\begin{equation}
\vect{P}_{\vect{k}} = \frac{\omega_{\vect{k}}}{2c\sqrt{\pi}} \left[ \vect{a}^*(\vect{k}) e^{i\omega_{\vect{k}} t} + \vect{a}(\vect{k})
e^{-i\omega_{\vect{k}} t} \right]
\label{eq:P_k}
\end{equation}

\begin{equation}
\vect{Q}_{\vect{k}} = \frac{1}{2ic\sqrt{\pi}} \left[ \vect{a}^*(\vect{k}) e^{i\omega_{\vect{k}} t} - \vect{a}(\vect{k}) e^{-i\omega_{\vect{k}} t} \right]
\label{eq:Q_k}
\end{equation}

Звідси:

\begin{equation*}
\vect{a}(\vect{k}) = c\sqrt{\pi} \left[ \frac{\vect{P}_{\vect{k}}}{\omega_{\vect{k}}} - i\vect{Q}_{\vect{k}} \right] e^{i\omega_{\vect{k}} t},
\end{equation*}

\begin{equation*}
\vect{a}^*(\vect{k}) = c\sqrt{\pi} \left[ \frac{\vect{P}_{\vect{k}}}{\omega_{\vect{k}}} + i\vect{Q}_{\vect{k}} \right] e^{-i\omega_{\vect{k}} t}.
\end{equation*}

\begin{equation*}
\vect{P}_{\vect{k}} \cdot \vect{k} = 0, \quad \vect{Q}_{\vect{k}} \cdot \vect{k} = 0
\end{equation*}