% !TeX program = lualatex
% !TeX encoding = utf8
% !TeX spellcheck = uk_UA
% !TeX root =../ClassicalEletrodynamics.tex

%=========================================================
\Opensolutionfile{answer}[\currfilebase/\currfilebase-Answers]
\Writetofile{answer}{\protect\section*{\nameref*{\currfilebase}}}
\chapter{Межі застосовності класичної електродинаміки}\label{\currfilebase}
%=========================================================



%% --------------------------------------------------------
\section{Квантова механіка і електродинаміка. }
%% --------------------------------------------------------


На сучасному рівні знань найбільш фундаментальним є квантовий розгляд фізичних процесів, який і визначає межі застосовності класичної теорії.
Електродинамічна система складається з заряджених частинок та електромагнітного поля; тут необхідно визначити, яка з цих складових (або уся система в
цілому) допускає класичний опис. Широке коло фізичних задач потребує квантового опису руху частинок в класичному електромагнітному полі. Основні зміни,
у порівнянні з класичною механікою, тут стосуються рівнянь~\eqref{eq:Columb_low}, \eqref{eq:Lorentz_force} та інших, пов’язаних із поняттями траєкторії,
сили, другим законом Ньютона, тощо. Перегляд цих понять у дослідженнях атомів та молекул – прерогатива квантової механіки, яка аналізує мікрооб’єкти з
розмірами $10^{-13} \div 10^{-16}$~см. Однак деякі мікропроцеси відзначають властивості твердих тіл і рідин також на макроскопічних масштабах. Хоча рух
заряджених частинок в цих задачах визначається законами квантової механіки, досить часто залишаються незмінними класичні поняття про напруженість
електричного поля та магнітну індукцію; наприклад, в рівнянні Шредінгера для електрона в атомі водню фігурує класичний кулонівський потенціал поля ядра.
Звичайно, що для визначення цих полів ми не завжди можемо прямо скористатися формулою~\eqref{eq:Lorentz_force}, що пов’язана з механікою точкової
частинки. Але зберігається класичний опис електромагнітного поля.


%% --------------------------------------------------------
\section{Квантова будова випромінювання}
%% --------------------------------------------------------

За певних умов треба враховувати квантові властивості самого електромагнітного поля. Вивчення рівноважного електромагнітного випромінювання, а також
фотоелектричних явищ (М.~Планк, 1900; А.~Ейнштейн,1905) привело до висновку, що електромагнітне випромінювання має корпускулярні властивості і може
розглядатися як сукупність окремих квантів-фотонів, з енергією $Е=h\nu$, де $h$ --- стала Планка, $\nu$ --- частота випромінювання. Коли фотонів багато,
можливий класичний опис поля випромінювання. Але в слабких пучках випромінювання рахунок йде на окремі фотони і сучасна техніка дозволяє майже
поодиночну їх реєстрацію. Теоретичну базу для опису процесів, в яких суттєвими є квантові властивості і речовини, і електромагнітного поля, дає квантова
електродинаміка, яка є складовою частиною квантової теорії поля. Квантова електродинаміка передбачає суттєві зміни характеру електродинамічної взаємодії
також в області дуже сильних полів. В електричному полі з напруженістю $E \sim \frac{m_e^2 c^3}{he} = 10^{20}$~В/м необхідно враховувати процеси
народження та знищення електрон–позитронних пар. В цих умовах електромагнітне поле не може розглядатися окремо від електрон– позитронного поля, навіть
при поширенні електромагнітної хвилі в вакуумі. Складна взаємодія цих полів робить нелінійними ефективні рівняння для класичних величин $E$ та $B$;
завдяки цьому стає можливим процес розсіювання фотона фотоном. В цьому розумінні можна говорити про порушення класичного принципу суперпозиції.



%% --------------------------------------------------------
\section{Електродинаміка і гравітація}
%% --------------------------------------------------------

Взаємодію гравітаційного та електромагнітного полів розглядає загальна
теорія відносності. Сильне гравітаційне поле не міняє класичний характер
електричного та магнітного полів, але вносить корективи в рівняння
електродинаміки на фоні викривлення простору-часу. Вплив гравітаційних
ефектів можна оцінити за допомогою параметра $\mu = |U|/c^2$, де $U$ --- порядок
зміни ньютонівського гравітаційного потенціалу в конкретній задачі.
Наприклад, при проходженні променів світла біля Сонця $\mu = 10^{-6}$, відповідний
порядок величини має кут зміщення віддаленого джерела променів, що його
спостерігають з Землі. Гравітаційно-релятивістські ефекти в Сонячній системі
необхідно враховувати для правильної інтерпретації найбільш точних
астрометричних спостережень.


\Closesolutionfile{answer}

